\chapter{Planificación}

\section{Introducción}

Se pretende utilizar una metodología ágil para el desarrollo del proyecto, más concretamente una adaptación de Scrum a las condiciones del mismo.

\bigskip
% El principal motivo de adaptar Scrum es que
% rescribir
Al ser solo una persona desarrollando el proyecto, es imposible seguir algunas fases del marco de trabajo como \textit{Daily Meetings}, entre otros. Las fases que son aplicables son: \textit{Sprint Planning}. No obstante, todos los artefactos se pueden generar.

\bigskip

Otros aspectos de Scrum que se van a aplicar son la utilización de sprints para desarrollar el proyecto, su enfoque iterativo e incremental y las Historias de Usuario.

\bigskip

Además, se van a utilizar herramientas que no son estrictamente de metodologías ágiles, como el diagrama de Gantt, para planificar la cantidad de tiempo que se piensa dedicar a cada sprint o Historia de Usuario, dependiendo del nivel de detalle. 

\bigskip

Por último, cabe recordar que no se pretende seguir Scrum de manera precisa, sino que se piensa utilizar algunos aspectos interesantes del marco, junto a otras herramientas que no son de metodologías ágiles.
% Scrum es un marco de trabajo iterativo e incremental, basado en un conjunto de sprints