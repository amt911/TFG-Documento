% DUDAS
% PUSE EXPLICACION GENERAL, PERO NO SE EXACTAMENTE QUE ES
% esto son datos cambiantes, puedo poner algo inicial, pero seguramente haya que modificarlo al final
% no se ni siquiera el numero de sprints que va a haber!! (tendria que tener una velocidad de equipo)
\chapter{Planificación}

\section{Introducción}

Se pretende utilizar una metodología ágil para el desarrollo del proyecto, más concretamente una adaptación de Scrum a las condiciones del mismo, junto a otras herramientas.

\bigskip
% El principal motivo de adaptar Scrum es que
% rescribir
Al ser solo una persona desarrollando el proyecto, es imposible seguir algunas fases del marco de trabajo como \textit{Daily Meetings}, entre otros. Las fases que son aplicables son: \textit{Sprint Planning}. No obstante, todos los artefactos se pueden generar.

\bigskip

Otros aspectos de Scrum que se van a aplicar son: la utilización de sprints para desarrollar el proyecto, su enfoque iterativo e incremental y las Historias de Usuario.

\bigskip

Además, se van a utilizar herramientas que no son estrictamente de metodologías ágiles, como el diagrama de Gantt, para planificar la cantidad de tiempo que se piensa dedicar a cada sprint o Historia de Usuario, dependiendo del nivel de detalle. 

\bigskip

Por último, cabe recordar que no se pretende seguir Scrum de manera precisa, al no cumplir todos los requisitos para poder usarlo. Solo que se piensa usar algunos aspectos aplicables del marco, junto a otras herramientas que no son exactamente de metodologías ágiles.


\section{Product Backlog}

El \textit{Product Backlog} con las Historias de Usuario, ordenadas por prioridad y con sus Puntos de Historia (PH), son:

\begin{table}[H]
    \centering
    \begin{tabularx}{\textwidth}{| >{\centering\arraybackslash}X | >{\centering\arraybackslash}m{5.5cm} | >{\centering\arraybackslash}X | >{\centering\arraybackslash}X |}
        \hline
        \textbf{Código} & \textbf{Descripción} & \textbf{Prioridad} & \textbf{PH} \\
        \hline
        HU1 & Como usuario, quiero que el sistema simula la conducción de varios pilotos en una carrera de coches, simulando adelantamientos y sorteo de obstáculos. & 1 & 5 \\
        \hline
        HU2 & Como usuario, quiero modificar los ajustes de la simulación antes de comenzar la carrera. & 2 & 3 \\
        \hline
        HU3 & Como usuario, quiero modificar el estado de los pilotos durante la carrera. & 2 & 2 \\
        \hline
        HU4 & Como usuario, quiero ver la posición actual de los pilotos de la carrera en curso. & 3 & 3 \\
        \hline
        HU5 & Como usuario, quiero ver el número de vueltas de la carrera en curso. & 3 & 2 \\ 
        \hline
        HU6 & Como usuario, quiero guardar la configuración de una carrera para su posterior uso. & 4 & 2\\
        \hline
        HU7 & Como usuario, quiero cargar la configuración de una carrera almacenada en un archivo. & 4 & 2 \\
        \hline
        HU8 & Como usuario, quiero pausar y reanudar la simulación en curso. & 5 & 0,5 \\
        \hline
        HU9 & Como usuario, quiero modificar la velocidad de la simulación. & 5 & 0,5 \\
        \hline
    \end{tabularx}
\end{table}
    
\section{Velocidad del equipo}

Esto solo es una estimación de la velocidad que se puede alcanzar para el primer Sprint, en función de los puntos de historia realizados al final del mismo se actualizará la velocidad del equipo.

\bigskip

Algunos aspectos a tener en cuenta son:

\begin{itemize}
    \item Duración de los sprints de \sprintLength, que equivalen a \actualSprintLength, quitando fines de semana.
    \item 1 PH representa un día de trabajo ideal a jornada completa (8 horas), pero suponiendo factores externos serán 5 
    horas reales. Usando las horas reales, 1 PH se completaría en 1,6 días.
    % \item Trabajaré 24 horas semanales.
    \item Hay un total de \projectph en el proyecto.
\end{itemize}

\bigskip

Sabiendo esto, la estimación de la velocidad del equipo para la primera iteración es: $\frac{10 \text{ días/sprint}}{1.6 \text{ días/PH}} = 6.25 \text{ PH/sprint} \simeq \mathbf{7\,PH/sprint}$
% $2 \text{ semanas/sprint} \times 20 \text{ horas/semana} \times 8 \text{ horas/PH} = $

\section{Diseño del Sprint}

Este proyecto se dividirá en \sprintNro sprints, cuya duración será de \sprintLength. A continuación se desglosará qué se realizará en cada uno de ellos.

\subsection{Sprint 1}
\begin{itemize}
    \item Explicación: En este sprint se pretende tener la base de la aplicación en funcionamiento. Como base se entiende a que los coches puedan moverse por el mapa mediante el algoritmo de navegación, pudiendo esquivar obstáculos y adelantarse.
    \item Temporización:
    \item Historias de Usuario abordadas: HU1
\end{itemize}

\section{Diagrama de Gantt}