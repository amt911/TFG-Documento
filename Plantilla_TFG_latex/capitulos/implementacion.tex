\chapter{Implementación}

\section{Introducción}

% rescribir: hablara
En este capítulo se hablará de las distintas herramientas a utilizar para implementar y planificar el proyecto. Además, se hablará de los distintos algoritmos utilizados para realizar la simulación.

\section{Herramientas utilizadas}
% mucho en cuanto
% rescribir
En el desarrollo del proyecto se utiliza \verb|git| para el control de versiones, subiendolo a un repositorio de Github. Además, se hará uso de ramas y de \textit{Pull Requests} para implementar cada una de las historias de usuario, con el objetivo de asemejarse a como se trabaja en la vida real.

\bigskip

% rescribir y quizas quitar esta parte si no se llega a programar en c++
En cuanto a la implementación de la aplicación, en las partes más críticas se utilizará C++, por lo que se utilizará Visual Studio como entorno de desarrollo.

\bigskip

% rescribir
En cuanto a herramientas para la planificación, se hará uso de la herramienta \planApp para llevar un control de las Historias de Usuario que se realizan y pendientes en cada Sprint.

% foto de \planApp

\section{Algoritmos utilizados}
% rescribir
En el control del volante de los vehículos, correspondiente a HU3, se ha utilizado un controlador PID, que consiste en tres componentes: Proporcional, Integral y Derivativa. La ventaja de utilizar un controlador de este tipo frente a decidir el movimiento del volante directamente, es que permite, si está bien calibrado, tener un control más suave y unas correcciones más realistas y con menos oscilaciones.

\bigskip
% consta de ... constantes xdd
El controlador PID consta de 3 constantes que son necesarias de ajustar para que funcione de manera correcta. Existen diversos métodos como el de Ziegler-Nichols\cite{enwiki:1140258750}, consiste en modificar solo la componente proporcional hasta que el coche comience a oscilar de manera continuada, en ese momento se debe calcular la frecuencia a la que oscila para obtener las constantes finales. 

\bigskip

% rescribir
Este método no me dio los resultados deseados, por lo que decidí implementar un algoritmo genético. Consiste en lanzar un conjunto de coches con valores aleatorios al principio que deben intentar llegar a la meta con el menor error posible. Aquellos que consiguen llegar con menos error, tienen más probabilidades de ser elegidos para generaciones futuras. La selección se realiza usando una ruleta\cite{enwiki:1141636554}, que consiste en lanzar una ruleta dividida según las probabilidades de cada coche. Una vez elegidos los coches, se juntan en parejas y mezclan sus constantes para obtener dos hijos cada uno. A continuación se mutan algunos hijos y finalmente se sustituyen los coches no elegidos por los hijos, dejando a los padres también. 
% Todo lo mencionado anteriormente tiene que ver con HU3.

\bigskip

En cuanto al algoritmo de navegación utilizado, he utilizado \finalAlg para obtener la ruta más óptima. Este algoritmo debe ejecutarse en determinados momentos, ya que está pensado principalmente para entornos estáticos.