\chapter{Análisis}

En este capítulo se abordará en más detalle cada Historia de Usuario, indicando sus pruebas de validación y las tareas que pueden surgir de cada una de ellas.

\bigskip

Cada Historia de Usuario se desarrollará en una tabla a continuación:

% \section{Historias de Usuario divididas en subtareas}

\begin{table}[H]
    \begin{tabularx}{\textwidth}{| X | X | X |}
        \hline
        \multicolumn{1}{|l|}{\textbf{Identificador:} HU3} & \multicolumn{2}{l|}{Modificación del estado de los pilotos.}
        \\ \hline
        \multicolumn{3}{|>{\hsize=\dimexpr3\hsize+4\tabcolsep+2\arrayrulewidth\relax\linewidth=\hsize}X|}{\textbf{Descripción:} Como usuario, quiero modificar el estado de los pilotos durante la carrera.}
        \\ \hline
        \textbf{Estimación:} 2 & \textbf{Prioridad:} 2 & \textbf{Entrega:} 2
        \\ \hline
        \multicolumn{3}{|>{\hsize=\dimexpr3\hsize+4\tabcolsep+2\arrayrulewidth\relax\linewidth=\hsize}X|}{\textbf{Pruebas de aceptación:}
        \begin{itemize}
            \item No se podrá elegir un valor fuera del rango del deslizador.
            \item Los cambios en el estado deberán verse reflejados en el comportamiento durante la carrera.
        \end{itemize}
        }
        \\ \hline
        \multicolumn{3}{|>{\hsize=\dimexpr3\hsize+4\tabcolsep+2\arrayrulewidth\relax\linewidth=\hsize}X|}{\textbf{Requisitos relacionados:} RF3, RNF5.}
        \\ \hline
        \multicolumn{3}{|>{\hsize=\dimexpr3\hsize+4\tabcolsep+2\arrayrulewidth\relax\linewidth=\hsize}X|}{\textbf{Dependencias:} HU4.}        
        \\ \hline
        \multicolumn{3}{|>{\hsize=\dimexpr3\hsize+4\tabcolsep+2\arrayrulewidth\relax\linewidth=\hsize}X|}{\textbf{Tareas:}
        \begin{itemize}
            \item Implementar los deslizadores referentes al aguante físico y mental y la agresividad.
        \end{itemize}
        }
        \\ \hline
        \multicolumn{3}{|>{\hsize=\dimexpr3\hsize+4\tabcolsep+2\arrayrulewidth\relax\linewidth=\hsize}X|}{\textbf{Observaciones:}}
        \\ \hline
    \end{tabularx}
\end{table}

\newpage


\begin{table}[H]
    \begin{tabularx}{\textwidth}{| X | X | X |}
        \hline
        \multicolumn{1}{|l|}{\textbf{Identificador:} HU1} & \multicolumn{2}{l|}{Algoritmo de navegación.}
        \\ \hline
        \multicolumn{3}{|>{\hsize=\dimexpr3\hsize+4\tabcolsep+2\arrayrulewidth\relax\linewidth=\hsize}X|}{\textbf{Descripción:} Como usuario, quiero que el sistema simule la conducción de varios pilotos en una carrera de coches, simulando adelantamientos y sorteo de obstáculos.}
        \\ \hline
        \textbf{Estimación:} 5 & \textbf{Prioridad:} 1 & \textbf{Entrega:} 1
        \\ \hline
        \multicolumn{3}{|>{\hsize=\dimexpr3\hsize+4\tabcolsep+2\arrayrulewidth\relax\linewidth=\hsize}X|}{\textbf{Pruebas de aceptación:}
        \begin{itemize}
            \item Se debe poder diferenciar la conducción entre pilotos con estados distintos.
            \item Los pilotos deben ser capaces de esquivar obstáculos en la pista.
            \item La simulación debe poder ejecutarse en un tiempo razonable, de manera que se pueda seguir la carrera en tiempo real.
            \item Los pilotos deben ser capaces de adelantar a otros que sean más lentos sin provocar accidentes.
        \end{itemize}
        }
        \\ \hline
        \multicolumn{3}{|>{\hsize=\dimexpr3\hsize+4\tabcolsep+2\arrayrulewidth\relax\linewidth=\hsize}X|}{\textbf{Requisitos relacionados:} RF6, RNF4.}
        \\ \hline
        \multicolumn{3}{|>{\hsize=\dimexpr3\hsize+4\tabcolsep+2\arrayrulewidth\relax\linewidth=\hsize}X|}{\textbf{Tareas:}
        \begin{itemize}
            \item Implementar la malla de navegación.
            \item Implementar el algoritmo a partir de la malla.
            \item Implementar la lógica de los vehículos, de forma que sean capaces usar esta malla para tomar decisiones.
        \end{itemize}
        }
        \\ \hline
        \multicolumn{3}{|>{\hsize=\dimexpr3\hsize+4\tabcolsep+2\arrayrulewidth\relax\linewidth=\hsize}X|}{\textbf{Observaciones:}}
        \\ \hline
    \end{tabularx}
\end{table}

\begin{table}[H]
    \begin{tabularx}{\textwidth}{| X | X | X |}
        \hline
        \multicolumn{1}{|l|}{\textbf{Identificador:} HU2} & \multicolumn{2}{l|}{Modificación de ajustes antes de la carrera.}
        \\ \hline
        \multicolumn{3}{|>{\hsize=\dimexpr3\hsize+4\tabcolsep+2\arrayrulewidth\relax\linewidth=\hsize}X|}{\textbf{Descripción:} Como usuario, quiero modificar los ajustes de la simulación antes de comenzar la carrera.}
        \\ \hline
        \textbf{Estimación:} 2 & \textbf{Prioridad:} 2 & \textbf{Entrega:} 1
        \\ \hline
        \multicolumn{3}{|>{\hsize=\dimexpr3\hsize+4\tabcolsep+2\arrayrulewidth\relax\linewidth=\hsize}X|}{\textbf{Pruebas de aceptación:}
        \begin{itemize}
            \item No se podrá elegir un número menor de 1 para la cantidad de vueltas.
            \item No se podrá elegir un número menor de 2 coches.
            \item No se podrá dejar el campo del nombre y los apellidos vacíos.
        \end{itemize}
        }
        \\ \hline
        \multicolumn{3}{|>{\hsize=\dimexpr3\hsize+4\tabcolsep+2\arrayrulewidth\relax\linewidth=\hsize}X|}{\textbf{Requisitos relacionados:} RF1, RNF2, RNF3.}
        \\ \hline
        \multicolumn{3}{|>{\hsize=\dimexpr3\hsize+4\tabcolsep+2\arrayrulewidth\relax\linewidth=\hsize}X|}{\textbf{Tareas:}
        \begin{itemize}
            \item Implementar los deslizadores correspondientes al número de vueltas, de coches, hora, aguante y experiencia.
            \item Implementar los campos de texto para el nombre y los apellidos.
            \item Programar la lógica necesaria para que los coches aparezcan en las posiciones correctas.
            \item Programar la lógica de las flechas para elegir a los pilotos.
        \end{itemize}
        }
        \\ \hline
        \multicolumn{3}{|>{\hsize=\dimexpr3\hsize+4\tabcolsep+2\arrayrulewidth\relax\linewidth=\hsize}X|}{\textbf{Observaciones:}}
        \\ \hline
    \end{tabularx}
\end{table}


\begin{table}[H]
    \begin{tabularx}{\textwidth}{| X | X | X |}
        \hline
        \multicolumn{1}{|l|}{\textbf{Identificador:} HU4} & \multicolumn{2}{l|}{Visualización de la información de los pilotos.}
        \\ \hline
        \multicolumn{3}{|>{\hsize=\dimexpr3\hsize+4\tabcolsep+2\arrayrulewidth\relax\linewidth=\hsize}X|}{\textbf{Descripción:} Como usuario, quiero ver la posición actual y el estado de los pilotos de la carrera en curso.}
        \\ \hline
        \textbf{Estimación:} 3 & \textbf{Prioridad:} 3 & \textbf{Entrega:} 2
        \\ \hline
        \multicolumn{3}{|>{\hsize=\dimexpr3\hsize+4\tabcolsep+2\arrayrulewidth\relax\linewidth=\hsize}X|}{\textbf{Pruebas de aceptación:}
        \begin{itemize}
            \item Cuando un piloto cambie posiciones con otro, el marcador debe cambiar.
            \item El cambio del estado del piloto producido por la carrera será reflejado en el deslizador.
        \end{itemize}
        }
        \\ \hline
        \multicolumn{3}{|>{\hsize=\dimexpr3\hsize+4\tabcolsep+2\arrayrulewidth\relax\linewidth=\hsize}X|}{\textbf{Requisitos relacionados:} RF2, RNF3.}
        \\ \hline
        \multicolumn{3}{|>{\hsize=\dimexpr3\hsize+4\tabcolsep+2\arrayrulewidth\relax\linewidth=\hsize}X|}{\textbf{Tareas:}
        \begin{itemize}
            \item Implementar el panel de las posiciones y el estado.
            \item Implementar la lógica asociada al cálculo de la posición general de cada piloto.
            \item Implementar la funcionalidad de mostrar los deslizadores del piloto al pulsar.
        \end{itemize}
        }
        \\ \hline
        \multicolumn{3}{|>{\hsize=\dimexpr3\hsize+4\tabcolsep+2\arrayrulewidth\relax\linewidth=\hsize}X|}{\textbf{Observaciones:}}
        \\ \hline
    \end{tabularx}
\end{table}

\begin{table}[H]
    \begin{tabularx}{\textwidth}{| X | X | X |}
        \hline
        \multicolumn{1}{|l|}{\textbf{Identificador:} HU5} & \multicolumn{2}{l|}{Visualización de las vueltas restantes.}
        \\ \hline
        \multicolumn{3}{|>{\hsize=\dimexpr3\hsize+4\tabcolsep+2\arrayrulewidth\relax\linewidth=\hsize}X|}{\textbf{Descripción:} Como usuario, quiero ver el número de vueltas de la carrera en curso.}
        \\ \hline
        \textbf{Estimación:} 2 & \textbf{Prioridad:} 3 & \textbf{Entrega:} 1
        \\ \hline
        \multicolumn{3}{|>{\hsize=\dimexpr3\hsize+4\tabcolsep+2\arrayrulewidth\relax\linewidth=\hsize}X|}{\textbf{Pruebas de aceptación:}
        \begin{itemize}
            \item El marcador no deberá mostrar como vuelta actual el número 0 ni un número superior al número máximo.
        \end{itemize}
        }
        \\ \hline
        \multicolumn{3}{|>{\hsize=\dimexpr3\hsize+4\tabcolsep+2\arrayrulewidth\relax\linewidth=\hsize}X|}{\textbf{Requisitos relacionados:} RF9, RNF3.}
        \\ \hline
        \multicolumn{3}{|>{\hsize=\dimexpr3\hsize+4\tabcolsep+2\arrayrulewidth\relax\linewidth=\hsize}X|}{\textbf{Tareas:}
        \begin{itemize}
            \item Implementar la lógica para calcular la vuelta actual.
            \item Crear el contador de la interfaz que mostrará las vueltas.
        \end{itemize}
        }
        \\ \hline
        \multicolumn{3}{|>{\hsize=\dimexpr3\hsize+4\tabcolsep+2\arrayrulewidth\relax\linewidth=\hsize}X|}{\textbf{Observaciones:}}
        \\ \hline
    \end{tabularx}
\end{table}


\begin{table}[H]
    \begin{tabularx}{\textwidth}{| X | X | X |}
        \hline
        \multicolumn{1}{|l|}{\textbf{Identificador:} HU6} & \multicolumn{2}{l|}{Exportación de la configuración.}
        \\ \hline
        \multicolumn{3}{|>{\hsize=\dimexpr3\hsize+4\tabcolsep+2\arrayrulewidth\relax\linewidth=\hsize}X|}{\textbf{Descripción:} Como usuario, quiero guardar la configuración de una carrera para su posterior uso.}
        \\ \hline
        \textbf{Estimación:} 2 & \textbf{Prioridad:} 4 & \textbf{Entrega:} 3
        \\ \hline
        \multicolumn{3}{|>{\hsize=\dimexpr3\hsize+4\tabcolsep+2\arrayrulewidth\relax\linewidth=\hsize}X|}{\textbf{Pruebas de aceptación:}}
        \\ \hline
        \multicolumn{3}{|>{\hsize=\dimexpr3\hsize+4\tabcolsep+2\arrayrulewidth\relax\linewidth=\hsize}X|}{\textbf{Requisitos relacionados:} RF7.}
        \\ \hline
        \multicolumn{3}{|>{\hsize=\dimexpr3\hsize+4\tabcolsep+2\arrayrulewidth\relax\linewidth=\hsize}X|}{\textbf{Tareas:}
        \begin{itemize}
            \item Implementar la lógica referente a la escritura de la configuración en un archivo externo.
        \end{itemize}
        }
        \\ \hline
        \multicolumn{3}{|>{\hsize=\dimexpr3\hsize+4\tabcolsep+2\arrayrulewidth\relax\linewidth=\hsize}X|}{\textbf{Observaciones:}}
        \\ \hline
    \end{tabularx}
\end{table}

\begin{table}[H]
    \begin{tabularx}{\textwidth}{| X | X | X |}
        \hline
        \multicolumn{1}{|l|}{\textbf{Identificador:} HU7} & \multicolumn{2}{l|}{Importación de la configuración.}
        \\ \hline
        \multicolumn{3}{|>{\hsize=\dimexpr3\hsize+4\tabcolsep+2\arrayrulewidth\relax\linewidth=\hsize}X|}{\textbf{Descripción:} Como usuario, quiero cargar la configuración de una carrera almacenada en un archivo.}
        \\ \hline
        \textbf{Estimación:} 2 & \textbf{Prioridad:} 4 & \textbf{Entrega:} 3
        \\ \hline
        \multicolumn{3}{|>{\hsize=\dimexpr3\hsize+4\tabcolsep+2\arrayrulewidth\relax\linewidth=\hsize}X|}{\textbf{Pruebas de aceptación:}
        \begin{itemize}
            \item El fichero generado en la fase de exportación debe ser legible por la aplicación.
        \end{itemize}
        }
        \\ \hline
        \multicolumn{3}{|>{\hsize=\dimexpr3\hsize+4\tabcolsep+2\arrayrulewidth\relax\linewidth=\hsize}X|}{\textbf{Requisitos relacionados:} RF8.}
        \\ \hline
        \multicolumn{3}{|>{\hsize=\dimexpr3\hsize+4\tabcolsep+2\arrayrulewidth\relax\linewidth=\hsize}X|}{\textbf{Tareas:}
        \begin{itemize}
            \item Implementar la lógica referente a la modificación de los parámetros por los valores que se encuentren en el archivo.
        \end{itemize}
        }
        \\ \hline
        \multicolumn{3}{|>{\hsize=\dimexpr3\hsize+4\tabcolsep+2\arrayrulewidth\relax\linewidth=\hsize}X|}{\textbf{Observaciones:}}
        \\ \hline
    \end{tabularx}
\end{table}

\newpage

\begin{table}[H]
    \begin{tabularx}{\textwidth}{| X | X | X |}
        \hline
        \multicolumn{1}{|l|}{\textbf{Identificador:} HU8} & \multicolumn{2}{l|}{Pausado y reanudación de la simulación.}
        \\ \hline
        \multicolumn{3}{|>{\hsize=\dimexpr3\hsize+4\tabcolsep+2\arrayrulewidth\relax\linewidth=\hsize}X|}{\textbf{Descripción:} Como usuario, quiero pausar y reanudar la simulación en curso.}
        \\ \hline
        \textbf{Estimación:} 0,5 & \textbf{Prioridad:} 5 & \textbf{Entrega:} 3
        \\ \hline
        \multicolumn{3}{|>{\hsize=\dimexpr3\hsize+4\tabcolsep+2\arrayrulewidth\relax\linewidth=\hsize}X|}{\textbf{Pruebas de aceptación:}
        \begin{itemize}
            \item Al pulsar el botón, debe alternar entre pausado y reanudado.
            \item Al encontrarse la simulación pausada, no se podrá modificar su velocidad.
        \end{itemize}
        }
        \\ \hline
        \multicolumn{3}{|>{\hsize=\dimexpr3\hsize+4\tabcolsep+2\arrayrulewidth\relax\linewidth=\hsize}X|}{\textbf{Requisitos relacionados:} RF5, RNF2.}
        \\ \hline
        \multicolumn{3}{|>{\hsize=\dimexpr3\hsize+4\tabcolsep+2\arrayrulewidth\relax\linewidth=\hsize}X|}{\textbf{Tareas:}
        \begin{itemize}
            \item Incluir el botón en la interfaz.
            \item Implementar la lógica relacionada con el botón y el pausado de la simulación.
        \end{itemize}
        }
        \\ \hline
        \multicolumn{3}{|>{\hsize=\dimexpr3\hsize+4\tabcolsep+2\arrayrulewidth\relax\linewidth=\hsize}X|}{\textbf{Observaciones:}}
        \\ \hline
    \end{tabularx}
\end{table}

\begin{table}[H]
    \begin{tabularx}{\textwidth}{| X | X | X |}
        \hline
        \multicolumn{1}{|l|}{\textbf{Identificador:} HU9} & \multicolumn{2}{l|}{Modificación de la velocidad de la simulación.}
        \\ \hline
        \multicolumn{3}{|>{\hsize=\dimexpr3\hsize+4\tabcolsep+2\arrayrulewidth\relax\linewidth=\hsize}X|}{\textbf{Descripción:} Como usuario, quiero modificar la velocidad de la simulación.}
        \\ \hline
        \textbf{Estimación:} 0,5 & \textbf{Prioridad:} 5 & \textbf{Entrega:} 3
        \\ \hline
        \multicolumn{3}{|>{\hsize=\dimexpr3\hsize+4\tabcolsep+2\arrayrulewidth\relax\linewidth=\hsize}X|}{\textbf{Pruebas de aceptación:}
        \begin{itemize}
            \item Al pulsar el botón, debe alternar entre distintas velocidades de simulación.
            \item Este botón no deberá funcionar si la simulación se encuentra pausada.
        \end{itemize}
        }
        \\ \hline
        \multicolumn{3}{|>{\hsize=\dimexpr3\hsize+4\tabcolsep+2\arrayrulewidth\relax\linewidth=\hsize}X|}{\textbf{Requisitos relacionados:} RF4, RNF2.}
        \\ \hline
        \multicolumn{3}{|>{\hsize=\dimexpr3\hsize+4\tabcolsep+2\arrayrulewidth\relax\linewidth=\hsize}X|}{\textbf{Tareas:}
        \begin{itemize}
            \item Incluir el botón en la interfaz.
            \item Implementar la lógica relacionada con el botón y la modificación de la velocidad de simulación.
        \end{itemize}
        }
        \\ \hline
        \multicolumn{3}{|>{\hsize=\dimexpr3\hsize+4\tabcolsep+2\arrayrulewidth\relax\linewidth=\hsize}X|}{\textbf{Observaciones:}}
        \\ \hline
    \end{tabularx}
\end{table}