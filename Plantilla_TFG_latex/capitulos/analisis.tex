\chapter{Análisis}

En este capítulo se abordará en más detalle cada Historia de Usuario, indicando sus pruebas de validación y las tareas que pueden surgir de cada una de ellas.

\bigskip

Cada Historia de Usuario se desarrollará a continuación en una tabla:

% \section{Historias de Usuario divididas en subtareas}

\begin{table}[H]
    \begin{tabularx}{\textwidth}{| X | X | X |}
        \hline
        \multicolumn{1}{|l|}{\textbf{Identificador:} HU1} & \multicolumn{2}{l|}{Algoritmo de navegación.}
        \\ \hline
        \multicolumn{3}{|>{\hsize=\dimexpr3\hsize+4\tabcolsep+2\arrayrulewidth\relax}X|}{\textbf{Descripción:} Como usuario, quiero que el sistema simule la conducción de varios pilotos en una carrera de coches, simulando adelantamientos y sorteo de obstáculos.}
        \\ \hline
        \textbf{Estimación:} 5 & \textbf{Prioridad:} 1 & \textbf{Entrega:} 1
        \\ \hline
        \multicolumn{3}{|>{\hsize=\dimexpr3\hsize+4\tabcolsep+2\arrayrulewidth\relax}X|}{\textbf{Pruebas de aceptación:}}
        \\
        \multicolumn{3}{|>{\hsize=\dimexpr3\hsize+4\tabcolsep+2\arrayrulewidth\relax}X|}{\tabitem Se debe poder diferenciar la conducción entre pilotos con estados distintos.}
        \\
        \multicolumn{3}{|>{\hsize=\dimexpr3\hsize+4\tabcolsep+2\arrayrulewidth\relax}X|}{\tabitem Los pilotos deben ser capaces de esquivar obstáculos en la pista.}        
        \\
        \multicolumn{3}{|>{\hsize=\dimexpr3\hsize+4\tabcolsep+2\arrayrulewidth\relax}X|}{\tabitem La simulación debe poder ejecutarse en un tiempo razonable, de manera que se pueda seguir la carrera en tiempo real.}
        \\
        \multicolumn{3}{|>{\hsize=\dimexpr3\hsize+4\tabcolsep+2\arrayrulewidth\relax}X|}{\tabitem Los pilotos deben ser capaces de adelantar a otros que sean más lentos sin provocar accidentes.}
        \\ \hline
        \multicolumn{3}{|>{\hsize=\dimexpr3\hsize+4\tabcolsep+2\arrayrulewidth\relax}X|}{\textbf{Requisitos relacionados:} RF6, RNF4}
        \\ \hline
        \multicolumn{3}{|>{\hsize=\dimexpr3\hsize+4\tabcolsep+2\arrayrulewidth\relax}X|}{\textbf{Tareas:}}
        \\
        \multicolumn{3}{|>{\hsize=\dimexpr3\hsize+4\tabcolsep+2\arrayrulewidth\relax}X|}{\tabitem Implementar la malla de navegación.}
        \\
        \multicolumn{3}{|>{\hsize=\dimexpr3\hsize+4\tabcolsep+2\arrayrulewidth\relax}X|}{\tabitem Implementar el algoritmo a partir de la malla.}
        \\
        \multicolumn{3}{|>{\hsize=\dimexpr3\hsize+4\tabcolsep+2\arrayrulewidth\relax}X|}{\tabitem Implementar la lógica de los vehículos, de forma que sean capaces usar esta malla para tomar decisiones.}                
        \\ \hline
        \multicolumn{3}{|>{\hsize=\dimexpr3\hsize+4\tabcolsep+2\arrayrulewidth\relax}X|}{\textbf{Observaciones:}}
        \\ \hline
    \end{tabularx}
\end{table}

\begin{table}[H]
    \begin{tabularx}{\textwidth}{| X | X | X |}
        \hline
        \multicolumn{1}{|l|}{\textbf{Identificador:} HU2} & \multicolumn{2}{l|}{Modificación de ajustes antes de la carrera.}
        \\ \hline
        \multicolumn{3}{|>{\hsize=\dimexpr3\hsize+4\tabcolsep+2\arrayrulewidth\relax}X|}{\textbf{Descripción:} Como usuario, quiero modificar los ajustes de la simulación antes de comenzar la carrera.}
        \\ \hline
        \textbf{Estimación:} 2 & \textbf{Prioridad:} 2 & \textbf{Entrega:} 1
        \\ \hline
        \multicolumn{3}{|>{\hsize=\dimexpr3\hsize+4\tabcolsep+2\arrayrulewidth\relax}X|}{\textbf{Pruebas de aceptación:}}
        \\
        \multicolumn{3}{|>{\hsize=\dimexpr3\hsize+4\tabcolsep+2\arrayrulewidth\relax}X|}{\tabitem No se podrá elegir un número menor de 1 para la cantidad de vueltas.}
        \\
        \multicolumn{3}{|>{\hsize=\dimexpr3\hsize+4\tabcolsep+2\arrayrulewidth\relax}X|}{\tabitem No se podrá elegir un número menor de 2 coches.} 
        \\
        \multicolumn{3}{|>{\hsize=\dimexpr3\hsize+4\tabcolsep+2\arrayrulewidth\relax}X|}{\tabitem No se podrá dejar el campo del nombre y los apellidos vacíos.}               
        \\ \hline
        \multicolumn{3}{|>{\hsize=\dimexpr3\hsize+4\tabcolsep+2\arrayrulewidth\relax}X|}{\textbf{Requisitos relacionados:} RF1, RNF2, RNF3}
        \\ \hline
        \multicolumn{3}{|>{\hsize=\dimexpr3\hsize+4\tabcolsep+2\arrayrulewidth\relax}X|}{\textbf{Tareas:}}
        \\
        \multicolumn{3}{|>{\hsize=\dimexpr3\hsize+4\tabcolsep+2\arrayrulewidth\relax}X|}{\tabitem Implementar los deslizadores correspondientes al número de vueltas, de coches, hora, aguante y experiencia.}
        \\
        \multicolumn{3}{|>{\hsize=\dimexpr3\hsize+4\tabcolsep+2\arrayrulewidth\relax}X|}{\tabitem Implementar los campos de texto para el nombre y los apellidos.}
        \\
        \multicolumn{3}{|>{\hsize=\dimexpr3\hsize+4\tabcolsep+2\arrayrulewidth\relax}X|}{\tabitem Programar la lógica necesaria para que los coches aparezcan en las posiciones correctas.}
        \\
        \multicolumn{3}{|>{\hsize=\dimexpr3\hsize+4\tabcolsep+2\arrayrulewidth\relax}X|}{\tabitem Programar la lógica de las flechas para elegir a los pilotos.}                        
        \\ \hline
        \multicolumn{3}{|>{\hsize=\dimexpr3\hsize+4\tabcolsep+2\arrayrulewidth\relax}X|}{\textbf{Observaciones:}}
        \\ \hline
    \end{tabularx}
\end{table}


\begin{table}[H]
    \begin{tabularx}{\textwidth}{| X | X | X |}
        \hline
        \multicolumn{1}{|l|}{\textbf{Identificador:} HU3} & \multicolumn{2}{l|}{Modificación del estado de los pilotos.}
        \\ \hline
        \multicolumn{3}{|>{\hsize=\dimexpr3\hsize+4\tabcolsep+2\arrayrulewidth\relax}X|}{\textbf{Descripción:} Como usuario, quiero modificar el estado de los pilotos durante la carrera.}
        \\ \hline
        \textbf{Estimación:} 2 & \textbf{Prioridad:} 2 & \textbf{Entrega:} 2
        \\ \hline
        \multicolumn{3}{|>{\hsize=\dimexpr3\hsize+4\tabcolsep+2\arrayrulewidth\relax}X|}{\textbf{Pruebas de aceptación:}}
        \\
        \multicolumn{3}{|>{\hsize=\dimexpr3\hsize+4\tabcolsep+2\arrayrulewidth\relax}X|}{\tabitem No se podrá elegir un valor fuera del rango del deslizador.}
        \\
        \multicolumn{3}{|>{\hsize=\dimexpr3\hsize+4\tabcolsep+2\arrayrulewidth\relax}X|}{\tabitem Los cambios en el estado deberán verse reflejados en el comportamiento durante la carrera.}                    
        \\ \hline
        \multicolumn{3}{|>{\hsize=\dimexpr3\hsize+4\tabcolsep+2\arrayrulewidth\relax}X|}{\textbf{Requisitos relacionados:} RF3, RNF5.}
        \\ \hline
        \multicolumn{3}{|>{\hsize=\dimexpr3\hsize+4\tabcolsep+2\arrayrulewidth\relax}X|}{\textbf{Dependencias:} HU4}        
        \\ \hline
        \multicolumn{3}{|>{\hsize=\dimexpr3\hsize+4\tabcolsep+2\arrayrulewidth\relax}X|}{\textbf{Tareas:}}
        \\
        \multicolumn{3}{|>{\hsize=\dimexpr3\hsize+4\tabcolsep+2\arrayrulewidth\relax}X|}{\tabitem Implementar los deslizadores referentes al aguante físico y mental y la agresividad.}       
        \\ \hline
        \multicolumn{3}{|>{\hsize=\dimexpr3\hsize+4\tabcolsep+2\arrayrulewidth\relax}X|}{\textbf{Observaciones:}}
        \\ \hline
    \end{tabularx}
\end{table}


\begin{table}[H]
    \begin{tabularx}{\textwidth}{| X | X | X |}
        \hline
        \multicolumn{1}{|l|}{\textbf{Identificador:} HU4} & \multicolumn{2}{l|}{Visualización de la información de los pilotos.}
        \\ \hline
        \multicolumn{3}{|>{\hsize=\dimexpr3\hsize+4\tabcolsep+2\arrayrulewidth\relax}X|}{\textbf{Descripción:} Como usuario, quiero ver la posición actual y el estado de los pilotos de la carrera en curso.}
        \\ \hline
        \textbf{Estimación:} 3 & \textbf{Prioridad:} 3 & \textbf{Entrega:} 2
        \\ \hline
        \multicolumn{3}{|>{\hsize=\dimexpr3\hsize+4\tabcolsep+2\arrayrulewidth\relax}X|}{\textbf{Pruebas de aceptación:}}
        \\
        \multicolumn{3}{|>{\hsize=\dimexpr3\hsize+4\tabcolsep+2\arrayrulewidth\relax}X|}{\tabitem Cuando un piloto cambie posiciones con otro, el marcador debe cambiar.}
        \\
        \multicolumn{3}{|>{\hsize=\dimexpr3\hsize+4\tabcolsep+2\arrayrulewidth\relax}X|}{\tabitem El cambio del estado del piloto producido por la carrera será reflejado en el deslizador.}        
        \\ \hline
        \multicolumn{3}{|>{\hsize=\dimexpr3\hsize+4\tabcolsep+2\arrayrulewidth\relax}X|}{\textbf{Requisitos relacionados:} RF2, RNF3}
        \\ \hline
        \multicolumn{3}{|>{\hsize=\dimexpr3\hsize+4\tabcolsep+2\arrayrulewidth\relax}X|}{\textbf{Tareas:}}
        \\
        \multicolumn{3}{|>{\hsize=\dimexpr3\hsize+4\tabcolsep+2\arrayrulewidth\relax}X|}{\tabitem Implementar el panel de las posiciones y el estado.}
        \\
        \multicolumn{3}{|>{\hsize=\dimexpr3\hsize+4\tabcolsep+2\arrayrulewidth\relax}X|}{\tabitem Implementar la lógica asociada al cálculo de la posición general de cada piloto.}        
        \\
        \multicolumn{3}{|>{\hsize=\dimexpr3\hsize+4\tabcolsep+2\arrayrulewidth\relax}X|}{\tabitem Implementar la funcionalidad de mostrar los deslizadores del piloto al pulsar.}         
        \\ \hline
        \multicolumn{3}{|>{\hsize=\dimexpr3\hsize+4\tabcolsep+2\arrayrulewidth\relax}X|}{\textbf{Observaciones:}}
        \\ \hline
    \end{tabularx}
\end{table}

\begin{table}[H]
    \begin{tabularx}{\textwidth}{| X | X | X |}
        \hline
        \multicolumn{1}{|l|}{\textbf{Identificador:} HU5} & \multicolumn{2}{l|}{Visualización de las vueltas restantes.}
        \\ \hline
        \multicolumn{3}{|>{\hsize=\dimexpr3\hsize+4\tabcolsep+2\arrayrulewidth\relax}X|}{\textbf{Descripción:} Como usuario, quiero ver el número de vueltas de la carrera en curso.}
        \\ \hline
        \textbf{Estimación:} 2 & \textbf{Prioridad:} 3 & \textbf{Entrega:} 1
        \\ \hline
        \multicolumn{3}{|>{\hsize=\dimexpr3\hsize+4\tabcolsep+2\arrayrulewidth\relax}X|}{\textbf{Pruebas de aceptación:}}
        \\
        \multicolumn{3}{|>{\hsize=\dimexpr3\hsize+4\tabcolsep+2\arrayrulewidth\relax}X|}{\tabitem El marcador no deberá mostrar como vuelta actual el número 0 ni un número superior al número máximo.}
        \\ \hline
        \multicolumn{3}{|>{\hsize=\dimexpr3\hsize+4\tabcolsep+2\arrayrulewidth\relax}X|}{\textbf{Requisitos relacionados:} RF9, RNF3}
        \\ \hline
        \multicolumn{3}{|>{\hsize=\dimexpr3\hsize+4\tabcolsep+2\arrayrulewidth\relax}X|}{\textbf{Tareas:}}
        \\
        \multicolumn{3}{|>{\hsize=\dimexpr3\hsize+4\tabcolsep+2\arrayrulewidth\relax}X|}{\tabitem Implementar la lógica para calcular la vuelta actual.}      
        \\ \hline
        \multicolumn{3}{|>{\hsize=\dimexpr3\hsize+4\tabcolsep+2\arrayrulewidth\relax}X|}{\textbf{Observaciones:}}
        \\ \hline
    \end{tabularx}
\end{table}


\begin{table}[H]
    \begin{tabularx}{\textwidth}{| X | X | X |}
        \hline
        \multicolumn{1}{|l|}{\textbf{Identificador:} HU6} & \multicolumn{2}{l|}{Exportación de la configuración.}
        \\ \hline
        \multicolumn{3}{|>{\hsize=\dimexpr3\hsize+4\tabcolsep+2\arrayrulewidth\relax}X|}{\textbf{Descripción:} Como usuario, quiero guardar la configuración de una carrera para su posterior uso.}
        \\ \hline
        \textbf{Estimación:} 2 & \textbf{Prioridad:} 4 & \textbf{Entrega:} 3
        \\ \hline
        \multicolumn{3}{|>{\hsize=\dimexpr3\hsize+4\tabcolsep+2\arrayrulewidth\relax}X|}{\textbf{Pruebas de aceptación:}}
        \\ \hline
        \multicolumn{3}{|>{\hsize=\dimexpr3\hsize+4\tabcolsep+2\arrayrulewidth\relax}X|}{\textbf{Requisitos relacionados:} RF7.}
        \\ \hline
        \multicolumn{3}{|>{\hsize=\dimexpr3\hsize+4\tabcolsep+2\arrayrulewidth\relax}X|}{\textbf{Tareas:}}
        \\
        \multicolumn{3}{|>{\hsize=\dimexpr3\hsize+4\tabcolsep+2\arrayrulewidth\relax}X|}{\tabitem Implementar la lógica referente a la escritura de la configuración en un archivo externo.}      
        \\ \hline
        \multicolumn{3}{|>{\hsize=\dimexpr3\hsize+4\tabcolsep+2\arrayrulewidth\relax}X|}{\textbf{Observaciones:}}
        \\ \hline
    \end{tabularx}
\end{table}

\begin{table}[H]
    \begin{tabularx}{\textwidth}{| X | X | X |}
        \hline
        \multicolumn{1}{|l|}{\textbf{Identificador:} HU7} & \multicolumn{2}{l|}{Importación de la configuración.}
        \\ \hline
        \multicolumn{3}{|>{\hsize=\dimexpr3\hsize+4\tabcolsep+2\arrayrulewidth\relax}X|}{\textbf{Descripción:} Como usuario, quiero cargar la configuración de una carrera almacenada en un archivo.}
        \\ \hline
        \textbf{Estimación:} 2 & \textbf{Prioridad:} 4 & \textbf{Entrega:} 3
        \\ \hline
        \multicolumn{3}{|>{\hsize=\dimexpr3\hsize+4\tabcolsep+2\arrayrulewidth\relax}X|}{\textbf{Pruebas de aceptación:}}
        \\
        \multicolumn{3}{|>{\hsize=\dimexpr3\hsize+4\tabcolsep+2\arrayrulewidth\relax}X|}{\tabitem El fichero generado en la fase de exportación debe ser legible por la aplicación.}
        \\ \hline
        \multicolumn{3}{|>{\hsize=\dimexpr3\hsize+4\tabcolsep+2\arrayrulewidth\relax}X|}{\textbf{Requisitos relacionados:} RF8.}
        \\ \hline
        \multicolumn{3}{|>{\hsize=\dimexpr3\hsize+4\tabcolsep+2\arrayrulewidth\relax}X|}{\textbf{Tareas:}}
        \\
        \multicolumn{3}{|>{\hsize=\dimexpr3\hsize+4\tabcolsep+2\arrayrulewidth\relax}X|}{\tabitem Implementar la lógica referente a la modificación de los parámetros por los valores que se encuentren en el archivo.}      
        \\ \hline
        \multicolumn{3}{|>{\hsize=\dimexpr3\hsize+4\tabcolsep+2\arrayrulewidth\relax}X|}{\textbf{Observaciones:}}
        \\ \hline
    \end{tabularx}
\end{table}

\begin{table}[H]
    \begin{tabularx}{\textwidth}{| X | X | X |}
        \hline
        \multicolumn{1}{|l|}{\textbf{Identificador:} HU8} & \multicolumn{2}{l|}{Pausado y reanudación de la simulación.}
        \\ \hline
        \multicolumn{3}{|>{\hsize=\dimexpr3\hsize+4\tabcolsep+2\arrayrulewidth\relax}X|}{\textbf{Descripción:} Como usuario, quiero pausar y reanudar la simulación en curso.}
        \\ \hline
        \textbf{Estimación:} 0,5 & \textbf{Prioridad:} 5 & \textbf{Entrega:} 3
        \\ \hline
        \multicolumn{3}{|>{\hsize=\dimexpr3\hsize+4\tabcolsep+2\arrayrulewidth\relax}X|}{\textbf{Pruebas de aceptación:}}
        \\
        \multicolumn{3}{|>{\hsize=\dimexpr3\hsize+4\tabcolsep+2\arrayrulewidth\relax}X|}{\tabitem Al pulsar el botón, debe alternar entre pausado y reanudado.}
        \\
        \multicolumn{3}{|>{\hsize=\dimexpr3\hsize+4\tabcolsep+2\arrayrulewidth\relax}X|}{\tabitem Al encontrarse la simulación pausada, no se podrá modificar su velocidad.}        
        \\ \hline
        \multicolumn{3}{|>{\hsize=\dimexpr3\hsize+4\tabcolsep+2\arrayrulewidth\relax}X|}{\textbf{Requisitos relacionados:} RF5, RNF2.}
        \\ \hline
        \multicolumn{3}{|>{\hsize=\dimexpr3\hsize+4\tabcolsep+2\arrayrulewidth\relax}X|}{\textbf{Tareas:}}
        \\
        \multicolumn{3}{|>{\hsize=\dimexpr3\hsize+4\tabcolsep+2\arrayrulewidth\relax}X|}{\tabitem Incluir el botón en la interfaz.}      
        \\
        \multicolumn{3}{|>{\hsize=\dimexpr3\hsize+4\tabcolsep+2\arrayrulewidth\relax}X|}{\tabitem Implementar la lógica relacionada con el botón y el pausado de la simulación.}              
        \\ \hline
        \multicolumn{3}{|>{\hsize=\dimexpr3\hsize+4\tabcolsep+2\arrayrulewidth\relax}X|}{\textbf{Observaciones:}}
        \\ \hline
    \end{tabularx}
\end{table}

\begin{table}[H]
    \begin{tabularx}{\textwidth}{| X | X | X |}
        \hline
        \multicolumn{1}{|l|}{\textbf{Identificador:} HU9} & \multicolumn{2}{l|}{Modificación de la velocidad de la simulación.}
        \\ \hline
        \multicolumn{3}{|>{\hsize=\dimexpr3\hsize+4\tabcolsep+2\arrayrulewidth\relax}X|}{\textbf{Descripción:} Como usuario, quiero modificar la velocidad de la simulación.}
        \\ \hline
        \textbf{Estimación:} 0,5 & \textbf{Prioridad:} 5 & \textbf{Entrega:} 3
        \\ \hline
        \multicolumn{3}{|>{\hsize=\dimexpr3\hsize+4\tabcolsep+2\arrayrulewidth\relax}X|}{\textbf{Pruebas de aceptación:}}
        \\
        \multicolumn{3}{|>{\hsize=\dimexpr3\hsize+4\tabcolsep+2\arrayrulewidth\relax}X|}{\tabitem Al pulsar el botón, debe alternar entre distintas velocidades de simulación.}
        \\
        \multicolumn{3}{|>{\hsize=\dimexpr3\hsize+4\tabcolsep+2\arrayrulewidth\relax}X|}{\tabitem Este botón no deberá funcionar si la simulación se encuentra pausada.}
        \\ \hline
        \multicolumn{3}{|>{\hsize=\dimexpr3\hsize+4\tabcolsep+2\arrayrulewidth\relax}X|}{\textbf{Requisitos relacionados:} RF4, RNF2.}
        \\ \hline
        \multicolumn{3}{|>{\hsize=\dimexpr3\hsize+4\tabcolsep+2\arrayrulewidth\relax}X|}{\textbf{Tareas:}}
        \\
        \multicolumn{3}{|>{\hsize=\dimexpr3\hsize+4\tabcolsep+2\arrayrulewidth\relax}X|}{\tabitem Incluir el botón en la interfaz.}      
        \\
        \multicolumn{3}{|>{\hsize=\dimexpr3\hsize+4\tabcolsep+2\arrayrulewidth\relax}X|}{\tabitem Implementar la lógica relacionada con el botón y la modificación de la velocidad de simulación.}              
        \\ \hline
        \multicolumn{3}{|>{\hsize=\dimexpr3\hsize+4\tabcolsep+2\arrayrulewidth\relax}X|}{\textbf{Observaciones:}}
        \\ \hline
    \end{tabularx}
\end{table}