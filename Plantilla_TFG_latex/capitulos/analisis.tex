\chapter{Análisis}

En este capítulo se abordará en más detalle cada Historia de Usuario, indicando sus pruebas de validación y las tareas que pueden surgir de cada una de ellas.

\bigskip

Cabe destacar que el campo de ``Entrega'' indica en qué sprint se va a realizar, mientras que el campo de ``Estimación'' indica cuanto Puntos de Historia (PH) se le han asignado.

\bigskip

Cada Historia de Usuario se desarrollará en una tabla a continuación:

% \section{Historias de Usuario divididas en subtareas}

\begin{table}[H]
    \begin{tabularx}{\textwidth}{| X | X | X |}
        \hline
        \multicolumn{1}{|l|}{\textbf{Identificador:} HU1} & \multicolumn{2}{l|}{Modelado de los coches y circuito.}
        \\ \hline
        \multicolumn{3}{|>{\hsize=\dimexpr3\hsize+4\tabcolsep+2\arrayrulewidth\relax\linewidth=\hsize}X|}{\textbf{Descripción:} Como usuario, quiero tener una representación en 3D de los coches y el circuito.}
        \\ \hline
        \textbf{Estimación:} 3 & \textbf{Prioridad:} 1 & \textbf{Entrega:} 4
        \\ \hline
        \multicolumn{3}{|>{\hsize=\dimexpr3\hsize+4\tabcolsep+2\arrayrulewidth\relax\linewidth=\hsize}X|}{\textbf{Tareas:}
        \begin{itemize}
            \item Modelar el trazado del circuito.
            \item Modelar los vehículos, o en su defecto, obtener modelos ya hechos.
        \end{itemize}
        }
        \\ \hline
        \multicolumn{3}{|>{\hsize=\dimexpr3\hsize+4\tabcolsep+2\arrayrulewidth\relax\linewidth=\hsize}X|}{\textbf{Requisitos relacionados:}}
        \\ \hline
        \multicolumn{3}{|>{\hsize=\dimexpr3\hsize+4\tabcolsep+2\arrayrulewidth\relax\linewidth=\hsize}X|}{\textbf{Pruebas de aceptación:}
        }
        \\ \hline
        \multicolumn{3}{|>{\hsize=\dimexpr3\hsize+4\tabcolsep+2\arrayrulewidth\relax\linewidth=\hsize}X|}{\textbf{Observaciones:}}
        \\ \hline
    \end{tabularx}
\end{table}

\begin{table}[H]
    \begin{tabularx}{\textwidth}{| X | X | X |}
        \hline
        \multicolumn{1}{|l|}{\textbf{Identificador:} HU2} & \multicolumn{2}{l|}{Cálculo de ruta óptima en el circuito.}
        \\ \hline
        \multicolumn{3}{|>{\hsize=\dimexpr3\hsize+4\tabcolsep+2\arrayrulewidth\relax\linewidth=\hsize}X|}{\textbf{Descripción:} Como usuario, quiero que los pilotos calculen la ruta más óptima en el circuito.}
        \\ \hline
        \textbf{Estimación:} 8 & \textbf{Prioridad:} 1 & \textbf{Entrega:} 5
        \\ \hline
        \multicolumn{3}{|>{\hsize=\dimexpr3\hsize+4\tabcolsep+2\arrayrulewidth\relax\linewidth=\hsize}X|}{\textbf{Tareas:}
        \begin{itemize}
            \item Implementar la malla de navegación, para que los coches puedan calcular rutas.
            \item Implementar el algoritmo de navegación, que haga uso de la malla.
            \item Crear los objetos auxiliares, para indicar a la malla la ruta más rápida.
        \end{itemize}
        }
        \\ \hline
        \multicolumn{3}{|>{\hsize=\dimexpr3\hsize+4\tabcolsep+2\arrayrulewidth\relax\linewidth=\hsize}X|}{\textbf{Requisitos relacionados:} RF6, RNF4.}
        \\ \hline
        \multicolumn{3}{|>{\hsize=\dimexpr3\hsize+4\tabcolsep+2\arrayrulewidth\relax\linewidth=\hsize}X|}{\textbf{Pruebas de aceptación:}
        \begin{itemize}
            \item El algoritmo debe calcular una ruta que no dé la vuelta en sentido contrario en el circuito.
            \item El algoritmo debe obtener un resultado en un tiempo razonable.
        \end{itemize}
        }
        \\ \hline
        \multicolumn{3}{|>{\hsize=\dimexpr3\hsize+4\tabcolsep+2\arrayrulewidth\relax\linewidth=\hsize}X|}{\textbf{Observaciones:}}
        \\ \hline
    \end{tabularx}
\end{table}

\begin{table}[H]
    \begin{tabularx}{\textwidth}{| X | X | X |}
        \hline
        \multicolumn{1}{|l|}{\textbf{Identificador:} HU3} & \multicolumn{2}{l|}{Seguimiento de una ruta con el volante.}
        \\ \hline
        \multicolumn{3}{|>{\hsize=\dimexpr3\hsize+4\tabcolsep+2\arrayrulewidth\relax\linewidth=\hsize}X|}{\textbf{Descripción:} Como usuario, quiero que los pilotos sean capaces de seguir una ruta de manera suave.}
        \\ \hline
        \textbf{Estimación:} 5 & \textbf{Prioridad:} 1 & \textbf{Entrega:} 4
        \\ \hline
        \multicolumn{3}{|>{\hsize=\dimexpr3\hsize+4\tabcolsep+2\arrayrulewidth\relax\linewidth=\hsize}X|}{\textbf{Tareas:}
        \begin{itemize}
            \item Implementar el controlador PID para usarlo en el volante.
            \item Implementar un algoritmo para obtener el mejor valor de las constantes.
        \end{itemize}
        }
        \\ \hline
        \multicolumn{3}{|>{\hsize=\dimexpr3\hsize+4\tabcolsep+2\arrayrulewidth\relax\linewidth=\hsize}X|}{\textbf{Requisitos relacionados:} RF6, RNF4.}
        \\ \hline
        \multicolumn{3}{|>{\hsize=\dimexpr3\hsize+4\tabcolsep+2\arrayrulewidth\relax\linewidth=\hsize}X|}{\textbf{Dependencias:}}
        \\ \hline   
        \multicolumn{3}{|>{\hsize=\dimexpr3\hsize+4\tabcolsep+2\arrayrulewidth\relax\linewidth=\hsize}X|}{\textbf{Pruebas de aceptación:}
        \begin{itemize}
            \item La amplitud máxima de las oscilaciones del vehículo en línea recta deben tener un tamaño menor a la anchura del coche.
            \item La amplitud de las oscilaciones debe reducirse con el tiempo.
            \item El coche debe tomar la curvas de manera suave.
        \end{itemize}
        }
        \\ \hline
        \multicolumn{3}{|>{\hsize=\dimexpr3\hsize+4\tabcolsep+2\arrayrulewidth\relax\linewidth=\hsize}X|}{\textbf{Observaciones:} El algoritmo utilizado para ajustar los valores del PID ha sido un algoritmo genético.}
        \\ \hline
    \end{tabularx}
\end{table}

\begin{table}[H]
    \begin{tabularx}{\textwidth}{| X | X | X |}
        \hline
        \multicolumn{1}{|l|}{\textbf{Identificador:} HU4} & \multicolumn{2}{l|}{Adelantamiento entre vehículos.}
        \\ \hline
        \multicolumn{3}{|>{\hsize=\dimexpr3\hsize+4\tabcolsep+2\arrayrulewidth\relax\linewidth=\hsize}X|}{\textbf{Descripción:} Como usuario, quiero que los pilotos sean capaces de adelantarse entre ellos.}
        \\ \hline
        \textbf{Estimación:} 5 & \textbf{Prioridad:} 1 & \textbf{Entrega:} 6
        \\ \hline
        \multicolumn{3}{|>{\hsize=\dimexpr3\hsize+4\tabcolsep+2\arrayrulewidth\relax\linewidth=\hsize}X|}{\textbf{Tareas:}
        \begin{itemize}
            \item Haciendo uso del algoritmo de HU2, implementar la lógica necesaria para saber cuando recalcular la ruta.
            \item Ajustar las colisiones de los vehículos, de forma que tengan espacio para adelantar.
        \end{itemize}
        }
        \\ \hline
        \multicolumn{3}{|>{\hsize=\dimexpr3\hsize+4\tabcolsep+2\arrayrulewidth\relax\linewidth=\hsize}X|}{\textbf{Requisitos relacionados:} RF6, RNF4.}
        \\ \hline
        \multicolumn{3}{|>{\hsize=\dimexpr3\hsize+4\tabcolsep+2\arrayrulewidth\relax\linewidth=\hsize}X|}{\textbf{Dependencias:} HU2.}
        \\ \hline   
        \multicolumn{3}{|>{\hsize=\dimexpr3\hsize+4\tabcolsep+2\arrayrulewidth\relax\linewidth=\hsize}X|}{\textbf{Pruebas de aceptación:}
        \begin{itemize}
            \item El adelantamiento de un coche a otro no debe provocar el accidente de ninguno de los dos.
        \end{itemize}
        }
        \\ \hline
        \multicolumn{3}{|>{\hsize=\dimexpr3\hsize+4\tabcolsep+2\arrayrulewidth\relax\linewidth=\hsize}X|}{\textbf{Observaciones:}}
        \\ \hline
    \end{tabularx}
\end{table}

\begin{table}[H]
    \begin{tabularx}{\textwidth}{| X | X | X |}
        \hline
        \multicolumn{1}{|l|}{\textbf{Identificador:} HU5} & \multicolumn{2}{l|}{Maniobra de recuperación en accidentes.}
        \\ \hline
        \multicolumn{3}{|>{\hsize=\dimexpr3\hsize+4\tabcolsep+2\arrayrulewidth\relax\linewidth=\hsize}X|}{\textbf{Descripción:} Como usuario, quiero que los pilotos sean capaces de recuperarse después de un accidente.}
        \\ \hline
        \textbf{Estimación:} 3 & \textbf{Prioridad:} 1 & \textbf{Entrega:} 6
        \\ \hline
        \multicolumn{3}{|>{\hsize=\dimexpr3\hsize+4\tabcolsep+2\arrayrulewidth\relax\linewidth=\hsize}X|}{\textbf{Tareas:}
        \begin{itemize}
            \item Implementar la lógica para que los pilotos recalculen una ruta que les devuelva a la pista utilizando el algoritmo de navegación.
        \end{itemize}
        }
        \\ \hline
        \multicolumn{3}{|>{\hsize=\dimexpr3\hsize+4\tabcolsep+2\arrayrulewidth\relax\linewidth=\hsize}X|}{\textbf{Requisitos relacionados:} RF6, RNF4.}
        \\ \hline
        \multicolumn{3}{|>{\hsize=\dimexpr3\hsize+4\tabcolsep+2\arrayrulewidth\relax\linewidth=\hsize}X|}{\textbf{Dependencias:} HU2.}
        \\ \hline        
        \multicolumn{3}{|>{\hsize=\dimexpr3\hsize+4\tabcolsep+2\arrayrulewidth\relax\linewidth=\hsize}X|}{\textbf{Pruebas de aceptación:}
        \begin{itemize}
            \item La nueva ruta debe calcularse en un tiempo razonable, de manera que pueda hacerse en tiempo real.
        \end{itemize}
        }
        \\ \hline
        \multicolumn{3}{|>{\hsize=\dimexpr3\hsize+4\tabcolsep+2\arrayrulewidth\relax\linewidth=\hsize}X|}{\textbf{Observaciones:}}
        \\ \hline
    \end{tabularx}
\end{table}

% ANTIGUOS DEL SPRINT 4

\begin{table}[H]
    \begin{tabularx}{\textwidth}{| X | X | X |}
        \hline
        \multicolumn{1}{|l|}{\textbf{Identificador:} HU6} & \multicolumn{2}{l|}{Modificación de ajustes antes de la carrera.}
        \\ \hline
        \multicolumn{3}{|>{\hsize=\dimexpr3\hsize+4\tabcolsep+2\arrayrulewidth\relax\linewidth=\hsize}X|}{\textbf{Descripción:} Como usuario, quiero modificar los ajustes de la simulación antes de comenzar la carrera.}
        \\ \hline
        \textbf{Estimación:} 2 & \textbf{Prioridad:} 2 & \textbf{Entrega:} 7
        \\ \hline
        \multicolumn{3}{|>{\hsize=\dimexpr3\hsize+4\tabcolsep+2\arrayrulewidth\relax\linewidth=\hsize}X|}{\textbf{Tareas:}
        \begin{itemize}
            \item Implementar los deslizadores correspondientes al número de vueltas, de coches, hora, aguante y experiencia.
            \item Implementar los campos de texto para el nombre y los apellidos.
            \item Programar la lógica necesaria para que los coches aparezcan en las posiciones correctas.
            \item Programar la lógica de las flechas para elegir a los pilotos.
        \end{itemize}
        }
        \\ \hline
        \multicolumn{3}{|>{\hsize=\dimexpr3\hsize+4\tabcolsep+2\arrayrulewidth\relax\linewidth=\hsize}X|}{\textbf{Requisitos relacionados:} RF1, RNF2, RNF3.}
        \\ \hline
        \multicolumn{3}{|>{\hsize=\dimexpr3\hsize+4\tabcolsep+2\arrayrulewidth\relax\linewidth=\hsize}X|}{\textbf{Pruebas de aceptación:}
        \begin{itemize}
            % \item No se podrá elegir un número menor de 1 para la cantidad de vueltas.
            % \item No se podrá elegir un número menor de 2 coches.
            \item No se podrá dejar el campo del nombre y los apellidos vacíos.
        \end{itemize}
        }
        \\ \hline
        \multicolumn{3}{|>{\hsize=\dimexpr3\hsize+4\tabcolsep+2\arrayrulewidth\relax\linewidth=\hsize}X|}{\textbf{Observaciones:}}
        \\ \hline
    \end{tabularx}
\end{table}

\begin{table}[H]
    \begin{tabularx}{\textwidth}{| X | X | X |}
        \hline
        \multicolumn{1}{|l|}{\textbf{Identificador:} HU7} & \multicolumn{2}{l|}{Modificación del estado de los pilotos.}
        \\ \hline
        \multicolumn{3}{|>{\hsize=\dimexpr3\hsize+4\tabcolsep+2\arrayrulewidth\relax\linewidth=\hsize}X|}{\textbf{Descripción:} Como usuario, quiero modificar el estado de los pilotos durante la carrera.}
        \\ \hline
        \textbf{Estimación:} 2 & \textbf{Prioridad:} 2 & \textbf{Entrega:} 7
        \\ \hline
        \multicolumn{3}{|>{\hsize=\dimexpr3\hsize+4\tabcolsep+2\arrayrulewidth\relax\linewidth=\hsize}X|}{\textbf{Tareas:}
        \begin{itemize}
            \item Implementar los deslizadores referentes al aguante físico y mental y la agresividad.
        \end{itemize}
        }
        \\ \hline
        \multicolumn{3}{|>{\hsize=\dimexpr3\hsize+4\tabcolsep+2\arrayrulewidth\relax\linewidth=\hsize}X|}{\textbf{Requisitos relacionados:} RF3, RNF5.}
        \\ \hline
        \multicolumn{3}{|>{\hsize=\dimexpr3\hsize+4\tabcolsep+2\arrayrulewidth\relax\linewidth=\hsize}X|}{\textbf{Dependencias:} HU4.}        
        \\ \hline
        \multicolumn{3}{|>{\hsize=\dimexpr3\hsize+4\tabcolsep+2\arrayrulewidth\relax\linewidth=\hsize}X|}{\textbf{Pruebas de aceptación:}
        \begin{itemize}
            \item Los cambios en el estado deberán verse reflejados en el comportamiento del piloto seleccionado durante la carrera.
        \end{itemize}
        }
        \\ \hline
        \multicolumn{3}{|>{\hsize=\dimexpr3\hsize+4\tabcolsep+2\arrayrulewidth\relax\linewidth=\hsize}X|}{\textbf{Observaciones:}}
        \\ \hline
    \end{tabularx}
\end{table}


\begin{table}[H]
    \begin{tabularx}{\textwidth}{| X | X | X |}
        \hline
        \multicolumn{1}{|l|}{\textbf{Identificador:} HU8} & \multicolumn{2}{l|}{Visualización de la información de los pilotos.}
        \\ \hline
        \multicolumn{3}{|>{\hsize=\dimexpr3\hsize+4\tabcolsep+2\arrayrulewidth\relax\linewidth=\hsize}X|}{\textbf{Descripción:} Como usuario, quiero ver la posición actual y el estado de los pilotos de la carrera en curso.}
        \\ \hline
        \textbf{Estimación:} 3 & \textbf{Prioridad:} 3 & \textbf{Entrega:} 7
        \\ \hline
        \multicolumn{3}{|>{\hsize=\dimexpr3\hsize+4\tabcolsep+2\arrayrulewidth\relax\linewidth=\hsize}X|}{\textbf{Tareas:}
        \begin{itemize}
            \item Implementar el panel de las posiciones y el estado.
            \item Implementar la lógica asociada al cálculo de la posición general de cada piloto.
            \item Implementar la funcionalidad de mostrar los deslizadores del piloto al pulsar.
        \end{itemize}
        }
        \\ \hline
        \multicolumn{3}{|>{\hsize=\dimexpr3\hsize+4\tabcolsep+2\arrayrulewidth\relax\linewidth=\hsize}X|}{\textbf{Requisitos relacionados:} RF2, RNF3.}
        \\ \hline
        \multicolumn{3}{|>{\hsize=\dimexpr3\hsize+4\tabcolsep+2\arrayrulewidth\relax\linewidth=\hsize}X|}{\textbf{Pruebas de aceptación:}
        \begin{itemize}
            \item Cuando un piloto cambie posiciones con otro, la lista de posiciones de los pilotos debe cambiar.
            \item El cambio del estado del piloto producido por la carrera será reflejado en el deslizador.
        \end{itemize}
        }
        \\ \hline
        \multicolumn{3}{|>{\hsize=\dimexpr3\hsize+4\tabcolsep+2\arrayrulewidth\relax\linewidth=\hsize}X|}{\textbf{Observaciones:}}
        \\ \hline
    \end{tabularx}
\end{table}

\begin{table}[H]
    \begin{tabularx}{\textwidth}{| X | X | X |}
        \hline
        \multicolumn{1}{|l|}{\textbf{Identificador:} HU9} & \multicolumn{2}{l|}{Visualización de las vueltas restantes.}
        \\ \hline
        \multicolumn{3}{|>{\hsize=\dimexpr3\hsize+4\tabcolsep+2\arrayrulewidth\relax\linewidth=\hsize}X|}{\textbf{Descripción:} Como usuario, quiero ver el número de vueltas de la carrera en curso.}
        \\ \hline
        \textbf{Estimación:} 2 & \textbf{Prioridad:} 3 & \textbf{Entrega:} 8
        \\ \hline
        \multicolumn{3}{|>{\hsize=\dimexpr3\hsize+4\tabcolsep+2\arrayrulewidth\relax\linewidth=\hsize}X|}{\textbf{Tareas:}
        \begin{itemize}
            \item Implementar la lógica para calcular la vuelta actual.
            \item Crear el contador de la interfaz que mostrará las vueltas.
        \end{itemize}
        }
        \\ \hline
        \multicolumn{3}{|>{\hsize=\dimexpr3\hsize+4\tabcolsep+2\arrayrulewidth\relax\linewidth=\hsize}X|}{\textbf{Requisitos relacionados:} RF9, RNF3.}
        \\ \hline
        \multicolumn{3}{|>{\hsize=\dimexpr3\hsize+4\tabcolsep+2\arrayrulewidth\relax\linewidth=\hsize}X|}{\textbf{Pruebas de aceptación:}
        \begin{itemize}
            \item El marcador no deberá mostrar como vuelta actual el número 0 ni un número superior al número máximo.
        \end{itemize}
        }
        \\ \hline
        \multicolumn{3}{|>{\hsize=\dimexpr3\hsize+4\tabcolsep+2\arrayrulewidth\relax\linewidth=\hsize}X|}{\textbf{Observaciones:}}
        \\ \hline
    \end{tabularx}
\end{table}


\begin{table}[H]
    \begin{tabularx}{\textwidth}{| X | X | X |}
        \hline
        \multicolumn{1}{|l|}{\textbf{Identificador:} HU10} & \multicolumn{2}{l|}{Exportación de la configuración.}
        \\ \hline
        \multicolumn{3}{|>{\hsize=\dimexpr3\hsize+4\tabcolsep+2\arrayrulewidth\relax\linewidth=\hsize}X|}{\textbf{Descripción:} Como usuario, quiero guardar la configuración de una carrera para su posterior uso.}
        \\ \hline
        \textbf{Estimación:} 2 & \textbf{Prioridad:} 4 & \textbf{Entrega:} 8
        \\ \hline
        \multicolumn{3}{|>{\hsize=\dimexpr3\hsize+4\tabcolsep+2\arrayrulewidth\relax\linewidth=\hsize}X|}{\textbf{Tareas:}
        \begin{itemize}
            \item Implementar la lógica referente a la escritura de la configuración en un archivo externo.
        \end{itemize}
        }
        \\ \hline
        \multicolumn{3}{|>{\hsize=\dimexpr3\hsize+4\tabcolsep+2\arrayrulewidth\relax\linewidth=\hsize}X|}{\textbf{Requisitos relacionados:} RF7.}
        \\ \hline
        \multicolumn{3}{|>{\hsize=\dimexpr3\hsize+4\tabcolsep+2\arrayrulewidth\relax\linewidth=\hsize}X|}{\textbf{Pruebas de aceptación:}
        \begin{itemize}
            \item Al pulsar el botón se debe producir un fichero externo con la configuración exacta que había en el configurador.
        \end{itemize}
        }
        \\ \hline
        \multicolumn{3}{|>{\hsize=\dimexpr3\hsize+4\tabcolsep+2\arrayrulewidth\relax\linewidth=\hsize}X|}{\textbf{Observaciones:}}
        \\ \hline
    \end{tabularx}
\end{table}

\begin{table}[H]
    \begin{tabularx}{\textwidth}{| X | X | X |}
        \hline
        \multicolumn{1}{|l|}{\textbf{Identificador:} HU11} & \multicolumn{2}{l|}{Importación de la configuración.}
        \\ \hline
        \multicolumn{3}{|>{\hsize=\dimexpr3\hsize+4\tabcolsep+2\arrayrulewidth\relax\linewidth=\hsize}X|}{\textbf{Descripción:} Como usuario, quiero cargar la configuración de una carrera almacenada en un archivo.}
        \\ \hline
        \textbf{Estimación:} 2 & \textbf{Prioridad:} 4 & \textbf{Entrega:} 8
        \\ \hline
        \multicolumn{3}{|>{\hsize=\dimexpr3\hsize+4\tabcolsep+2\arrayrulewidth\relax\linewidth=\hsize}X|}{\textbf{Tareas:}
        \begin{itemize}
            \item Implementar la lógica referente a la modificación de los parámetros por los valores que se encuentren en el archivo.
        \end{itemize}
        }
        \\ \hline
        \multicolumn{3}{|>{\hsize=\dimexpr3\hsize+4\tabcolsep+2\arrayrulewidth\relax\linewidth=\hsize}X|}{\textbf{Requisitos relacionados:} RF8.}
        \\ \hline
        \multicolumn{3}{|>{\hsize=\dimexpr3\hsize+4\tabcolsep+2\arrayrulewidth\relax\linewidth=\hsize}X|}{\textbf{Pruebas de aceptación:}
        \begin{itemize}
            % \item El fichero generado en la fase de exportación debe ser legible por la aplicación.
            \item Ante un fichero válido de configuración, la aplicación lo leerá correctamente y modificará los parámetros de forma adecuada.
        \end{itemize}
        }
        \\ \hline
        \multicolumn{3}{|>{\hsize=\dimexpr3\hsize+4\tabcolsep+2\arrayrulewidth\relax\linewidth=\hsize}X|}{\textbf{Observaciones:}}
        \\ \hline
    \end{tabularx}
\end{table}


\begin{table}[H]
    \begin{tabularx}{\textwidth}{| X | X | X |}
        \hline
        \multicolumn{1}{|l|}{\textbf{Identificador:} HU12} & \multicolumn{2}{l|}{Pausado y reanudación de la simulación.}
        \\ \hline
        \multicolumn{3}{|>{\hsize=\dimexpr3\hsize+4\tabcolsep+2\arrayrulewidth\relax\linewidth=\hsize}X|}{\textbf{Descripción:} Como usuario, quiero pausar y reanudar la simulación en curso.}
        \\ \hline
        \textbf{Estimación:} 0,5 & \textbf{Prioridad:} 4 & \textbf{Entrega:} 8
        \\ \hline
        \multicolumn{3}{|>{\hsize=\dimexpr3\hsize+4\tabcolsep+2\arrayrulewidth\relax\linewidth=\hsize}X|}{\textbf{Tareas:}
        \begin{itemize}
            \item Incluir el botón en la interfaz.
            \item Implementar la lógica relacionada con el botón y el pausado de la simulación.
        \end{itemize}
        }
        \\ \hline
        \multicolumn{3}{|>{\hsize=\dimexpr3\hsize+4\tabcolsep+2\arrayrulewidth\relax\linewidth=\hsize}X|}{\textbf{Requisitos relacionados:} RF5, RNF2.}
        \\ \hline
        \multicolumn{3}{|>{\hsize=\dimexpr3\hsize+4\tabcolsep+2\arrayrulewidth\relax\linewidth=\hsize}X|}{\textbf{Pruebas de aceptación:}
        \begin{itemize}
            \item Al pulsar el botón, debe alternar entre pausado y reanudado.
            \item Al encontrarse la simulación pausada, no se podrá modificar su velocidad.
        \end{itemize}
        }
        \\ \hline
        \multicolumn{3}{|>{\hsize=\dimexpr3\hsize+4\tabcolsep+2\arrayrulewidth\relax\linewidth=\hsize}X|}{\textbf{Observaciones:}}
        \\ \hline
    \end{tabularx}
\end{table}

\begin{table}[H]
    \begin{tabularx}{\textwidth}{| X | X | X |}
        \hline
        \multicolumn{1}{|l|}{\textbf{Identificador:} HU13} & \multicolumn{2}{l|}{Modificación de la velocidad de la simulación.}
        \\ \hline
        \multicolumn{3}{|>{\hsize=\dimexpr3\hsize+4\tabcolsep+2\arrayrulewidth\relax\linewidth=\hsize}X|}{\textbf{Descripción:} Como usuario, quiero modificar la velocidad de la simulación.}
        \\ \hline
        \textbf{Estimación:} 0,5 & \textbf{Prioridad:} 4 & \textbf{Entrega:} 8
        \\ \hline
        \multicolumn{3}{|>{\hsize=\dimexpr3\hsize+4\tabcolsep+2\arrayrulewidth\relax\linewidth=\hsize}X|}{\textbf{Tareas:}
        \begin{itemize}
            \item Incluir el botón en la interfaz.
            \item Implementar la lógica relacionada con el botón y la modificación de la velocidad de simulación.
        \end{itemize}
        }
        \\ \hline
        \multicolumn{3}{|>{\hsize=\dimexpr3\hsize+4\tabcolsep+2\arrayrulewidth\relax\linewidth=\hsize}X|}{\textbf{Requisitos relacionados:} RF4, RNF2.}
        \\ \hline
        \multicolumn{3}{|>{\hsize=\dimexpr3\hsize+4\tabcolsep+2\arrayrulewidth\relax\linewidth=\hsize}X|}{\textbf{Pruebas de aceptación:}
        \begin{itemize}
            \item Al pulsar el botón, debe alternar entre el conjunto de velocidades de simulación existentes.
            \item Este botón no deberá funcionar si la simulación se encuentra pausada.
        \end{itemize}
        }
        \\ \hline
        \multicolumn{3}{|>{\hsize=\dimexpr3\hsize+4\tabcolsep+2\arrayrulewidth\relax\linewidth=\hsize}X|}{\textbf{Observaciones:}}
        \\ \hline
    \end{tabularx}
\end{table}


\begin{table}[H]
    \begin{tabularx}{\textwidth}{| X | X | X |}
        \hline
        \multicolumn{1}{|l|}{\textbf{Identificador:} HU14} & \multicolumn{2}{l|}{Salida de la simulación en curso.}
        \\ \hline
        \multicolumn{3}{|>{\hsize=\dimexpr3\hsize+4\tabcolsep+2\arrayrulewidth\relax\linewidth=\hsize}X|}{\textbf{Descripción:} Como usuario, quiero salir de la simulación que hay en curso.}
        \\ \hline
        \textbf{Estimación:} 0,5 & \textbf{Prioridad:} 4 & \textbf{Entrega:} 8
        \\ \hline
        \multicolumn{3}{|>{\hsize=\dimexpr3\hsize+4\tabcolsep+2\arrayrulewidth\relax\linewidth=\hsize}X|}{\textbf{Tareas:}
        \begin{itemize}
            \item Incluir el botón en la interfaz.
            \item Implementar la lógica relacionada con el botón para poder salir de la simulación.
        \end{itemize}
        }
        \\ \hline
        \multicolumn{3}{|>{\hsize=\dimexpr3\hsize+4\tabcolsep+2\arrayrulewidth\relax\linewidth=\hsize}X|}{\textbf{Requisitos relacionados:} RF10, RNF2.}
        \\ \hline
        \multicolumn{3}{|>{\hsize=\dimexpr3\hsize+4\tabcolsep+2\arrayrulewidth\relax\linewidth=\hsize}X|}{\textbf{Pruebas de aceptación:}
        \begin{itemize}
            \item Al pulsar el botón, debe salir al configurador de la simulación.
        \end{itemize}
        }
        \\ \hline
        \multicolumn{3}{|>{\hsize=\dimexpr3\hsize+4\tabcolsep+2\arrayrulewidth\relax\linewidth=\hsize}X|}{\textbf{Observaciones:}}
        \\ \hline
    \end{tabularx}
\end{table}


\begin{table}[H]
    \begin{tabularx}{\textwidth}{| X | X | X |}
        \hline
        \multicolumn{1}{|l|}{\textbf{Identificador:} HU15} & \multicolumn{2}{l|}{Frenada anticipativa.}
        \\ \hline
        \multicolumn{3}{|>{\hsize=\dimexpr3\hsize+4\tabcolsep+2\arrayrulewidth\relax\linewidth=\hsize}X|}{\textbf{Descripción:} Como usuario, quiero que los vehículos puedan frenar para evitar posibles colisiones.}
        \\ \hline
        \textbf{Estimación:} 2 & \textbf{Prioridad:} 4 & \textbf{Entrega:} 9
        \\ \hline
        \multicolumn{3}{|>{\hsize=\dimexpr3\hsize+4\tabcolsep+2\arrayrulewidth\relax\linewidth=\hsize}X|}{\textbf{Tareas:}
        \begin{itemize}
            \item Implementar la lógica para detectar cuando un vehículo debe frenar.
        \end{itemize}
        }
        \\ \hline
        \multicolumn{3}{|>{\hsize=\dimexpr3\hsize+4\tabcolsep+2\arrayrulewidth\relax\linewidth=\hsize}X|}{\textbf{Requisitos relacionados:} RF6.}
        \\ \hline
        \multicolumn{3}{|>{\hsize=\dimexpr3\hsize+4\tabcolsep+2\arrayrulewidth\relax\linewidth=\hsize}X|}{\textbf{Pruebas de aceptación:}
        \begin{itemize}
            \item Cuando un coche detecta que va a colisionar, debe frenar, evitando el accidente.
        \end{itemize}
        }
        \\ \hline
        \multicolumn{3}{|>{\hsize=\dimexpr3\hsize+4\tabcolsep+2\arrayrulewidth\relax\linewidth=\hsize}X|}{\textbf{Observaciones:}}
        \\ \hline
    \end{tabularx}
\end{table}


\begin{table}[H]
    \begin{tabularx}{\textwidth}{| X | X | X |}
        \hline
        \multicolumn{1}{|l|}{\textbf{Identificador:} HU16} & \multicolumn{2}{l|}{Utilización del rebufo.}
        \\ \hline
        \multicolumn{3}{|>{\hsize=\dimexpr3\hsize+4\tabcolsep+2\arrayrulewidth\relax\linewidth=\hsize}X|}{\textbf{Descripción:} Como usuario, quiero que los vehículos puedan utilizar el rebufo del vehículo de delante.}
        \\ \hline
        \textbf{Estimación:} 1 & \textbf{Prioridad:} 4 & \textbf{Entrega:} 9
        \\ \hline
        \multicolumn{3}{|>{\hsize=\dimexpr3\hsize+4\tabcolsep+2\arrayrulewidth\relax\linewidth=\hsize}X|}{\textbf{Tareas:}
        \begin{itemize}
            \item Implementar la lógica para detectar cuando un vehículo se encuentra detrás de otro.
        \end{itemize}
        }
        \\ \hline
        \multicolumn{3}{|>{\hsize=\dimexpr3\hsize+4\tabcolsep+2\arrayrulewidth\relax\linewidth=\hsize}X|}{\textbf{Requisitos relacionados:} RF6.}
        \\ \hline
        \multicolumn{3}{|>{\hsize=\dimexpr3\hsize+4\tabcolsep+2\arrayrulewidth\relax\linewidth=\hsize}X|}{\textbf{Pruebas de aceptación:}
        \begin{itemize}
            \item Al encontrarse el vehículo de detrás en zona de rebufo, deberá aumentar su aceleración.
        \end{itemize}
        }
        \\ \hline
        \multicolumn{3}{|>{\hsize=\dimexpr3\hsize+4\tabcolsep+2\arrayrulewidth\relax\linewidth=\hsize}X|}{\textbf{Observaciones:}}
        \\ \hline
    \end{tabularx}
\end{table}

\begin{table}[H]
    \begin{tabularx}{\textwidth}{| X | X | X |}
        \hline
        \multicolumn{1}{|l|}{\textbf{Identificador:} HU17} & \multicolumn{2}{l|}{Cambio de la cámara.}
        \\ \hline
        \multicolumn{3}{|>{\hsize=\dimexpr3\hsize+4\tabcolsep+2\arrayrulewidth\relax\linewidth=\hsize}X|}{\textbf{Descripción:} Como usuario, quiero cambiar entre distintas cámaras disponibles.}
        \\ \hline
        \textbf{Estimación:} 3 & \textbf{Prioridad:} 4 & \textbf{Entrega:} 9
        \\ \hline
        \multicolumn{3}{|>{\hsize=\dimexpr3\hsize+4\tabcolsep+2\arrayrulewidth\relax\linewidth=\hsize}X|}{\textbf{Tareas:}
        \begin{itemize}
            \item Implementar la lógica para cambiar entre las distintas cámaras.
            \item Implementar los botones de la interfaz de usuario para poder cambiar de cámara.
        \end{itemize}
        }
        \\ \hline
        \multicolumn{3}{|>{\hsize=\dimexpr3\hsize+4\tabcolsep+2\arrayrulewidth\relax\linewidth=\hsize}X|}{\textbf{Requisitos relacionados:} RF13.}
        \\ \hline
        \multicolumn{3}{|>{\hsize=\dimexpr3\hsize+4\tabcolsep+2\arrayrulewidth\relax\linewidth=\hsize}X|}{\textbf{Pruebas de aceptación:}
        \begin{itemize}
            \item Al pulsar el botón para cambiar de cámara, la visualización deberá cambiar de manera acorde.
        \end{itemize}
        }
        \\ \hline
        \multicolumn{3}{|>{\hsize=\dimexpr3\hsize+4\tabcolsep+2\arrayrulewidth\relax\linewidth=\hsize}X|}{\textbf{Observaciones:}}
        \\ \hline
    \end{tabularx}
\end{table}

\begin{table}[H]
    \begin{tabularx}{\textwidth}{| X | X | X |}
        \hline
        \multicolumn{1}{|l|}{\textbf{Identificador:} HU18} & \multicolumn{2}{l|}{Visualización de las rutas.}
        \\ \hline
        \multicolumn{3}{|>{\hsize=\dimexpr3\hsize+4\tabcolsep+2\arrayrulewidth\relax\linewidth=\hsize}X|}{\textbf{Descripción:} Como usuario, quiero visualizar la ruta generada por los vehículos, de manera opcional.}
        \\ \hline
        \textbf{Estimación:} 2 & \textbf{Prioridad:} 4 & \textbf{Entrega:} 9
        \\ \hline
        \multicolumn{3}{|>{\hsize=\dimexpr3\hsize+4\tabcolsep+2\arrayrulewidth\relax\linewidth=\hsize}X|}{\textbf{Tareas:}
        \begin{itemize}
            \item Implementar la lógica para generar cada uno de los cubos que compone la ruta de los coches.
            \item Implementar la funcionalidad de la interfaz de usuario para poder ocultar y mostrar la visualización de la ruta.
        \end{itemize}
        }
        \\ \hline
        \multicolumn{3}{|>{\hsize=\dimexpr3\hsize+4\tabcolsep+2\arrayrulewidth\relax\linewidth=\hsize}X|}{\textbf{Requisitos relacionados:} RF11.}
        \\ \hline
        \multicolumn{3}{|>{\hsize=\dimexpr3\hsize+4\tabcolsep+2\arrayrulewidth\relax\linewidth=\hsize}X|}{\textbf{Pruebas de aceptación:}
        \begin{itemize}
            \item Al estar el botón pulsado, la ruta de todos los coches deberá aparecer.
            \item Si el botón no está pulsado, la ruta de todos los coches no debe aparecer en el circuito.
        \end{itemize}
        }
        \\ \hline
        \multicolumn{3}{|>{\hsize=\dimexpr3\hsize+4\tabcolsep+2\arrayrulewidth\relax\linewidth=\hsize}X|}{\textbf{Observaciones:}}
        \\ \hline
    \end{tabularx}
\end{table}


\begin{table}[H]
    \begin{tabularx}{\textwidth}{| X | X | X |}
        \hline
        \multicolumn{1}{|l|}{\textbf{Identificador:} HU19} & \multicolumn{2}{l|}{Visualización de los puntos de control.}
        \\ \hline
        \multicolumn{3}{|>{\hsize=\dimexpr3\hsize+4\tabcolsep+2\arrayrulewidth\relax\linewidth=\hsize}X|}{\textbf{Descripción:} Como usuario, quiero visualizar los puntos de control del circuito, de manera opcional.}
        \\ \hline
        \textbf{Estimación:} 0,5 & \textbf{Prioridad:} 4 & \textbf{Entrega:} 9
        \\ \hline
        \multicolumn{3}{|>{\hsize=\dimexpr3\hsize+4\tabcolsep+2\arrayrulewidth\relax\linewidth=\hsize}X|}{\textbf{Tareas:}
        \begin{itemize}
            \item Implementar la lógica en la interfaz para mostrar y ocultar los puntos de control.
        \end{itemize}
        }
        \\ \hline
        \multicolumn{3}{|>{\hsize=\dimexpr3\hsize+4\tabcolsep+2\arrayrulewidth\relax\linewidth=\hsize}X|}{\textbf{Requisitos relacionados:} RF12.}
        \\ \hline
        \multicolumn{3}{|>{\hsize=\dimexpr3\hsize+4\tabcolsep+2\arrayrulewidth\relax\linewidth=\hsize}X|}{\textbf{Pruebas de aceptación:}
        \begin{itemize}
            \item Al estar el botón pulsado, los puntos de control deberán aparecer.
            \item Si el botón no está pulsado, los puntos de control no debe aparecer en el circuito.
        \end{itemize}
        }
        \\ \hline
        \multicolumn{3}{|>{\hsize=\dimexpr3\hsize+4\tabcolsep+2\arrayrulewidth\relax\linewidth=\hsize}X|}{\textbf{Observaciones:}}
        \\ \hline
    \end{tabularx}
\end{table}