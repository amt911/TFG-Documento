% Capitulo conclusiones.

% Que se ha hecho en el proyecto, logros conseguidos. Posibles vias de futuro para trabajo: mejoras o añadir mas funcionalidad.

\chapter{Conclusiones y trabajos futuros}
% Como conclusión, he logrado alcanzar todos los objetivos que había planteado de manera satisfactoria. Aunque he tenido problemas que ralentizaron el desarrollo, como el ajuste correcto de las constantes del controlador PID, problemas de rendimiento en el algoritmo de navegación, fallos al calcular el cambio de posición en el marcador, colisiones extrañas durante la carrera y errores en la apertura de cuadros de diálogo, he podido resolverlos y continuar con la implementación de la aplicación, cumpliendo con los plazos planificados.

% \section{Competencias adquiridas}

Gracias al desarrollo del proyecto he adquirido un mayor conocimiento sobre diversos algoritmos de navegación, especialmente en el algoritmo A*, junto a sus diversas heurísticas y optimizaciones. También he conseguido entender como funcionan y se implementan las colas con prioridad utilizando \textit{Binary Heaps}.

\bigskip
 
He comprendido el funcionamiento de los algoritmos genéticos y he realizado una implementación utilizando \textit{blueprints} para el entrenamiento de las constantes del controlador del volante, obteniendo mejores resultados en cada generación.

\bigskip


% He ampliado mis conocimientos en el manejo de \textit{Unreal Engine} y en la programación mediante \textit{blueprints}. 
En cuanto al manejo de \textit{Unreal Engine} y la programación con \textit{blueprints}, he ampliado mis conocimientos ya existentes.
Asimismo, he aprendido a programar utilizando la API en C++, cuya gestión de memoria es distinta a como se trabaja en C++ nativo. El uso de C++ me ha permitido crear algoritmos más eficientes, así como actores, objetos y la posibilidad de compartir su funcionalidad con otros \textit{blueprints}.

\bigskip

Además, he aplicado conceptos matemáticos y físicos en el desarrollo de funciones para el seguimiento de rutas, cálculo de posiciones entre los vehículos y determinación de la distancia de frenado.

\bigskip

Por último, aunque no era uno de los objetivos, he profundizado mis conocimientos en el programa Blender, especialmente en el modelado de objetos, mapeado UV, generación de texturas normales y \textit{rigging} de los vehículos.  


\section{Trabajos futuros}

He conseguido implementar todas las historias de usuario planificadas, pero en el futuro se podría incluir las siguientes funcionalidades en la aplicación:

\begin{itemize}
    \item Ciclo día/noche modificable de manera similar al estado de los pilotos.
    \item Variabilidad de rendimiento entre vehículos de una misma disciplina.
    \item Nuevas características de los pilotos que afecten a su rendimiento en la carrera.
    \item Posibilidad de conducir uno de los vehículos.
    \item Inclusión de sonido de motor en los vehículos.
\end{itemize}

\bigskip

Además, se podrían mejorar algunas de las siguientes características:

\begin{itemize}
    \item Mejora de rendimiento en términos de fotogramas por segundo durante la simulación.
    \item Reducción del tiempo de lanzamiento de la simulación.
    \item Toma de decisiones basadas en más factores.
    \item Reimplementación en C++ de objetos con gran cantidad de operaciones.
    \item Arreglo de giros erróneos del volante que pueden ocurrir ocasionalmente.
\end{itemize}