\chapter{Conclusiones}
% rescribir
Los resultados obtenidos en el proyecto son satisfactorios, habiendo profundizado mis conocimientos en los diversos algoritmos de navegación, así como en el algoritmo A* y sus diversas heurísticas y optimizaciones y también me ha permitido a entender como funcionan las colas con prioridad utilizando \textit{Binary Heaps}.

\bigskip

También he profundizado en las distintas alternativas para el manejo suave del volante.

\bigskip 
También he conseugido entender e implementar un algoritmo genético básico para el entrenamiento del conrolador del volante, viendo de manera empírica como en cada generación los resultados mejoraban.

\bigskip

% rescribir
Además, he conseguido profundizar mis conocimientos en el manejo de \textit{Unreal Engine} y en la programación utilizando \textit{blueprints}. Asimismo, he aprendido a utilizar la API en C++, permitiéndome crear actores, objetos y poder exponerlos a los \textit{blueprints}.


\bigskip
% rescribir
Por último, si bien no es algo que esperaba hacer, he profundizado mis conocimientos en el programa Blender, así como en el modelado de objetos, mapeado UV, generación de texturas normales y \textit{rigging} de los vehículos.


\section{Trabajos futuros}

He conseguido implementar todas las historias de usuario de manera satisfactoria, pero se podría incluir las siguientes funcionalidades en la aplicación:

\begin{itemize}
    \item Ciclo día/noche modificable de manera similar al estado de los pilotos.
    \item Nuevas características de los pilotos que afecten a su rendimiento en la carrera.
    \item Posibilidad de conducir uno de los vehículos.
    \item Inclusión de sonido de motor en los vehículos.
\end{itemize}

Además, se podrían mejorar los siguientes aspectos:

\begin{itemize}
    \item Aumento del número de coches, manteniendo la fluidez en la simulación.
    \item Mejora de rendimiento en términos de fotogramas por segundo durante la simulación.
    \item Reducción del tiempo de lanzamiento de la simulación.
    \item Toma de decisiones basadas en más factores.
    \item Reimplementación en C++ de objetos con gran cantidad de operaciones.
\end{itemize}