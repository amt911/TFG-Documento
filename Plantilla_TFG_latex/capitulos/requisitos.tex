% DUDAS
% LAS REFERENCIAS DEBERIAN IR EN LAS REFERENCIAS DEL DOCUMENTO EN GENERAL
% LA CALIDAD DEL DOCUMENTO ES PESIMA
% LOS REQUISITOS DE INFORMACION SON NO FUNCIONALES
% HE REPETIDO DEMASIADAS VECES LO QUE HACE MI SIMULADOR
% EN LA MAYORIA DE LAS SECCIONES HE RELLENADO MAS BIEN POCO
% ME PARECEN POCOS REQUISITOS EN GENERAL, PERO NO SE ME OCURREN MAS

% DUDAS TUTORIA
%no se si poner aqui lo de la pagina web. tipo, que no hay suficientes opciones software para el publico general.

\chapter{Especificación de requisitos}

\section{Introducción}

\subsection{Propósito}
%mejorar
El objetivo de esta especificación es definir de manera clara, concisa y precisa las funciones y restricciones que tendrá la aplicación que se desea desarrollar. Al ser un trabajo individual, irá dirigido principalmente a mí, que haré de equipo de desarrollo.

\subsection{Alcance}
%rescribir subsection entera

% repes: carrera
Este sistema se encargará de simular carreras de coches, donde los pilotos tendrán distintas características que variarán dependiendo de las condiciones de la carrera e influirán en su desempeño. El usuario podrá modificar el estado de cada piloto en tiempo real, observando como estos afectan al comportamiento del vehículo.

\bigskip

Además, el software permitirá modificar una serie de ajustes antes de simular la carrera, de acuerdo a las preferencias del usuario. Algunos parámetros son: el número de vehículos, las aptitudes de cada piloto, el tipo de coche y la velocidad máxima. De esta manera, se podrán crear carreras personalizadas de acuerdo a las necesidades del usuario.

\subsection{Definiciones, siglas y abreviaturas}

\begin{itemize}
    \item \textbf{Definiciones: }
        \begin{itemize}
            \item \textbf{Usuario: }Persona que utiliza el sistema para poder obtener los resultados de la carrera personalizada.
        \end{itemize}
\end{itemize}

\subsection{Referencias}

Los siguientes documentos y enlaces se han consultado para crear este capítulo:

\begin{itemize}
    \item Ingeniería del Software. Ejercicio en clase, unidad 3: Requerimientos del Software. LSI (UGR). \url{https://lsi2.ugr.es/~mvega/docis/aluwork/colectivo/Ejercicio%20en%20clase%20version%202}
    \item fusm calidad del software. \url{https://sites.google.com/site/fusmcalidaddelsoftware/proyecto/g-informe-de-especificacion-de-requerimientos/3-requisitos-especificos/3-5-atributos-del-sistema}
\end{itemize}

\subsection{Visión general}
% rescribir
Este capítulo constará de tres secciones: %Introducción, descripción global, y requisitos específicos. 

\bigskip

En esta primera sección se muestra la introducción y la visión general de la especificación de requisitos.

\bigskip

En la sección 2 se proporcionará una descripción general del sistema a construir, con el fin de conocer las funciones principales, datos requeridos, restricciones y otros aspectos relevantes. 

\bigskip

En la sección 3 se definen detalladamente los requisitos que debe cumplir el sistema en el momento de su desarrollo.

\section{Descripción general}
\subsection{Perspectiva del producto}
La aplicación se diseñará como una aplicación gráfica que permitirá a los usuarios simular carreras de coches, permitiendo configurar parámetros antes de la simulación y durante la carrera. El sistema recopilará información sobre la situación de carrera de cada piloto, y utilizará estos datos para simular el desempeño de cada uno en tiempo real.

\bigskip

Los usuarios interactuarán con el sistema a través de una interfaz gráfica con la que podrán modificar todos los ajustes que estimen convenientes.

\subsection{Funciones del producto}

Las funciones principales de la aplicación son las siguientes:

\begin{itemize}
    \item 
    %Configuración de los parámetros antes de la carrera: 
    Otorgará la posibilidad de configurar distintas opciones antes de la carrera como: número de pilotos, aptitudes de cada uno, tipo de vehículo, velocidad máxima y número de vueltas. 
    
    \item Calcular la condición del piloto según las condiciones actuales de la carrera, haciendo que si empeora su estado cometa más errores y si mejora que cometa menos errores, todo en tiempo real.
    
    \item Modificar la condición de los pilotos durante la carrera, con independencia de las condiciones de la misma, pudiendo ver en tiempo real como afecta el cambio producido.
\end{itemize}

Y tendrá otras funciones como:

\begin{itemize}
    \item Pausar la simulación.
    \item Acelerar o reducir la velocidad de simulación.
    \item Cambiar la hora del día.
\end{itemize}

\subsection{Características de los usuarios}

Es recomendable que los usuarios tengan conocimientos básicos de informática, pero no es necesario que sepan como funcionan las carreras, ya que está todo automatizado. 

\subsection{Restricciones}

Las restricciones que tendrá la aplicación son las siguientes:

\begin{itemize}
    \item Una vez pausada la simulación, no se podrá modificar su velocidad.
    \item El número de pilotos en la carrera no puede ser menor o igual a 1.
    \item Las aptitudes de cada piloto antes de la carrera tienen que ser mayores que 0.
    \item La velocidad máxima no podrá ser menor de 150 km/h.
\end{itemize}

\subsection{Suposiciones y dependencias}

Los requisitos descritos pueden estar sujetos a cambio en función de la evolución del proyecto y la posible adición de nuevas funcionalidades. 

Este sistema funciona de forma independiente, por lo que no es necesario comunicarse con otros sistemas externos o instalar ningún programa adicional, excepto el propio simulador.

\subsection{Requerimientos específicos}

\subsubsection{Requisitos funcionales}

\begin{itemize}
    \item RF1.- Modificación de parámetros antes de la carrera: El sistema debe permitir modificar los parámetros antes de ejecutar la simulación.
    \item RF2.- Visualización del estado de los pilotos: El usuario podrá ver el estado de cada piloto durante la carrera, para poder realizar operaciones sobre el mismo, si lo ve necesario.
    \item RF3.- Modificación del estado de los pilotos durante la carrera: El sistema permitirá modificar el estado de los pilotos en tiempo real.
    \item RF4.- Actualización de la velocidad de simulación: El sistema permitirá actualizar la velocidad entre varias opciones predefinidas.
    \item RF5.- Pausado de la simulación: El sistema permitirá pausar y reanudar la simulación que está en curso.
\end{itemize}

\subsubsection{Requisitos de soporte}

\begin{itemize}
    % \item RNF1.- El sistema debe ser ejecutado en un entorno Windows 10 o Windows 11.
    \item RNF1.- El sistema se ejecutará en los sistemas operativos Windows 10 y Windows 11.
\end{itemize}

\subsubsection{Requisitos de usabilidad}

\begin{itemize}
    \item RNF2.- La interfaz deberá ser fácil de usar e intuitiva para los usuarios, de manera que puedan navegar por la misma sin demasiada dificultad.
    \item RNF3.- El tamaño del texto y el estilo de fuente deben ser adecuados para facilitar su lectura.
\end{itemize}

\subsubsection{Requisitos de información}

\begin{itemize}
    \item RI1.- Datos de los pilotos: Nombre, nacionalidad, aguante mental y físico, agresividad y experiencia.
    % \begin{itemize}
    %     \item Nombre
    %     \item Nacionalidad
    %     \item Aguante mental
    %     \item Aguante físico
    %     \item Agresividad
    %     \item Experiencia
    % \end{itemize}
    \item RI2.- Datos del simulador: Velocidad de simulación, hora actual, número de vueltas, vuelta actual y número de vehículos.
\end{itemize}