\chapter{Especificación de requisitos}

\section{Introducción}

\subsection{Propósito}
%mejorar
El objetivo de esta especificación es definir de manera clara, concisa y precisa las funciones y restricciones que tendrá la aplicación que se desea desarrollar. Al ser un trabajo individual, irá dirigido principalmente a mí, que haré de equipo de desarrollo.

\subsection{Alcance}
%rescribir subsection entera

%no se si poner aqui lo de la pagina web. tipo, que no hay suficientes opciones software para el publico general.

Este sistema se encargará de simular carreras de coches, cuyos pilotos tendrán una serie de características que variarán dependiendo de las condiciones de la carrera. También mostrará las condiciones actuales de cada piloto y permitirá que el usuario pueda modificarlas en tiempo real, viendo como afecta a los movimientos del vehículo.

\subsection{Definiciones, siglas y abreviaturas}

\begin{itemize}
    \item \textbf{Definiciones: }
        \begin{itemize}
            \item \textbf{Usuario: }Persona que utiliza el sistema para poder obtener los resultados de la carrera configurada.
        \end{itemize}
\end{itemize}

% \subsection{Referencias} ???

\subsection{Visión general}
% rescribir
Este capítulo constará de tres secciones: Introducción, descripción global, y requisitos específicos. 

\bigskip

En esta primera sección se muestra la introducción y la visión general de la especificación de requisitos.

En la siguiente sección se proporcionará una descripción general del sistema a construir, con el fin de conocer las funciones principales, datos requeridos y restricciones, entre otras cosas. 

En la última sección se definen detalladamente los requisitos que debe cumplir el sistema para ser considerado como terminado.