\chapter*{}
%\thispagestyle{empty}
%\cleardoublepage

%\thispagestyle{empty}

\input{portada/portada_2}



\cleardoublepage
\thispagestyle{empty}

\begin{center}
{\large\bfseries Título del Proyecto: Subtítulo del proyecto}\\
\end{center}
\begin{center}
Andrés Merlo Trujillo\\
\end{center}

%\vspace{0.7cm}
\noindent{\textbf{Palabras clave}: Unreal, Simulación, IA }\\

\vspace{0.7cm}
\noindent{\textbf{Resumen}}\\

% Se pretende desarrollar una aplicación gráfica utilizando el motor gráfico \textit{Unreal Engine} cuyo objetivo es de poder simular carreras de coches. Uno de los objetivos de esta simulación es la de dotar a los pilotos virtuales de la capacidad de tomar decisiones realistas y de cometer errores, todo esto basado en las condiciones de cada piloto durante la carrera, que pueden variar dependiendo de las condiciones externas. El simulador también permitirá la modificación de parámetros antes y durante la carrera. Algunos de estos parámetros son: número de coches, número de vueltas, capacidad de los pilotos y modificación de las condiciones de cada piloto en tiempo real.

Se pretende desarrollar una aplicación gráfica utilizando el motor gráfico \textit{Unreal Engine} cuyo objetivo es de poder simular carreras de coches. Esta simulación dotará a los pilotos virtuales de la capacidad de tomar decisiones realistas y de cometer errores, basándose en las condiciones de cada piloto durante la carrera, que pueden variar dependiendo de las condiciones externas o ser modificadas por el usuario en tiempo real. Asimismo, se podrán configurar ajustes adicionales antes de comenzar la carrera, con el objetivo de hacer la simulación lo más personalizable posible.

\cleardoublepage


\thispagestyle{empty}


\begin{center}
{\large\bfseries Project Title: Project Subtitle}\\
\end{center}
\begin{center}
First name, Family name (student)\\
\end{center}

%\vspace{0.7cm}
\noindent{\textbf{Keywords}: Keyword1, Keyword2, Keyword3, ....}\\

\vspace{0.7cm}
\noindent{\textbf{Abstract}}\\

Write here the abstract in English.

\chapter*{}
\thispagestyle{empty}

\noindent\rule[-1ex]{\textwidth}{2pt}\\[4.5ex]

Yo, \textbf{Andrés Merlo Trujillo}, alumno de la titulación Ingeniería Informática de la \textbf{Escuela Técnica Superior
de Ingenierías Informática y de Telecomunicación de la Universidad de Granada}, con DNI 77147239H, autorizo la
ubicación de la siguiente copia de mi Trabajo Fin de Grado en la biblioteca del centro para que pueda ser
consultada por las personas que lo deseen.

\vspace{6cm}

\noindent Fdo: Andrés Merlo Trujillo

\vspace{2cm}

\begin{flushright}
Granada a X de mes de 2023.
\end{flushright}


\chapter*{}
\thispagestyle{empty}

\noindent\rule[-1ex]{\textwidth}{2pt}\\[4.5ex]

D. \textbf{Luis López Escudero}, Profesor del Área de XXXX del Departamento YYYYY de la Universidad de Granada.

% \vspace{0.5cm}

% D. \textbf{Nombre Apellido1 Apellido2 (tutor2)}, Profesor del Área de XXXX del Departamento YYYY de la Universidad de Granada.


\vspace{0.5cm}

\textbf{Informan:}

\vspace{0.5cm}

Que el presente trabajo, titulado \textit{\textbf{Título del proyecto, Subtítulo del proyecto}},
ha sido realizado bajo su supervisión por \textbf{Andrés Merlo Trujillo}, y autorizamos la defensa de dicho trabajo ante el tribunal
que corresponda.

\vspace{0.5cm}

Y para que conste, expiden y firman el presente informe en Granada a X de mes de 2023 .

\vspace{1cm}

\textbf{Los directores:}

\vspace{5cm}

\noindent \textbf{Luis López Escudero}%\ \ \ \ \ Nombre Apellido1 Apellido2 (tutor2)}

\chapter*{Agradecimientos}
\thispagestyle{empty}

       \vspace{1cm}


Poner aquí agradecimientos...

