\chapter*{}
%\thispagestyle{empty}
%\cleardoublepage

%\thispagestyle{empty}

\input{portada/portada_2}



\cleardoublepage
\thispagestyle{empty}

\begin{center}
{\large\bfseries \myTitle}\\
\end{center}
\begin{center}
Andrés Merlo Trujillo\\
\end{center}

%\vspace{0.7cm}
\noindent{\textbf{Palabras clave}: Unreal, Simulación, IA }\\

\vspace{0.7cm}
\noindent{\textbf{Resumen}}\\

Se pretende desarrollar una aplicación gráfica utilizando el motor gráfico \textit{Unreal Engine}, cuyo objetivo es de poder simular carreras de coches. Esta simulación dotará a los pilotos virtuales de la capacidad de tomar decisiones realistas y de cometer errores, basándose en las condiciones de cada piloto durante la carrera, que pueden variar dependiendo de las condiciones externas o ser modificadas por el usuario en tiempo real. Además, se podrán configurar ajustes adicionales antes de comenzar la carrera, con el objetivo de hacer la simulación lo más personalizable posible.

\cleardoublepage


\thispagestyle{empty}


\begin{center}
{\large\bfseries Car racing simulator in Unreal Engine}\\
\end{center}
\begin{center}
Andrés Merlo Trujillo\\
\end{center}

%\vspace{0.7cm}
\noindent{\textbf{Keywords}: Unreal, Simulation, AI }\\

\vspace{0.7cm}
\noindent{\textbf{Abstract}}\\

The aim is to develop a graphics application using the graphics engine \textit{Unreal Engine}, whose objective is to be able to simulate car races. This simulation will provide virtual pilots with the ability to make realistic decisions and make mistakes, based on the conditions of each pilot during the race, which can vary depending on external conditions or be modified by the user in real time. Furthermore, additional settings can be configured before starting the race, with the aim of making the simulation as customizable as possible.

\chapter*{}
\thispagestyle{empty}

\noindent\rule[-1ex]{\textwidth}{2pt}\\[4.5ex]

Yo, \textbf{Andrés Merlo Trujillo}, alumno de la titulación Ingeniería Informática de la \textbf{Escuela Técnica Superior
de Ingenierías Informática y de Telecomunicación de la Universidad de Granada}, con DNI 77147239H, autorizo la
ubicación de la siguiente copia de mi Trabajo Fin de Grado en la biblioteca del centro para que pueda ser
consultada por las personas que lo deseen.

\vspace{6cm}

\noindent Fdo: Andrés Merlo Trujillo

\vspace{2cm}

\begin{flushright}
Granada a 11 de julio de 2023.
\end{flushright}


\chapter*{}
\thispagestyle{empty}

\noindent\rule[-1ex]{\textwidth}{2pt}\\[4.5ex]

D. \textbf{Luis López Escudero}, Profesor del Área del Departamento de Lenguajes y Sistemas Informáticos de la Universidad de Granada.

\vspace{0.5cm}

D. \textbf{Germán Arroyo Moreno}, Profesor del Área del Departamento de Lenguajes y Sistemas Informáticos de la Universidad de Granada.


\vspace{0.5cm}

\textbf{Informan:}

\vspace{0.5cm}

Que el presente trabajo, titulado \textit{\textbf{\myTitle}},
ha sido realizado bajo su supervisión por \textbf{Andrés Merlo Trujillo}, y autorizamos la defensa de dicho trabajo ante el tribunal
que corresponda.

\vspace{0.5cm}

Y para que conste, expiden y firman el presente informe en Granada a 11 de julio de 2023.

\vspace{1cm}

\textbf{Los directores:}

\vspace{5cm}

\noindent \textbf{Luis López Escudero \ \ \ \ \ Germán Arroyo Moreno}

\chapter*{Agradecimientos}
\thispagestyle{empty}

       \vspace{1cm}


A mi familia y a Luis por brindarme su apoyo durante todo el desarrollo del proyecto.

