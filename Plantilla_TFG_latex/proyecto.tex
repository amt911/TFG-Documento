%PREGUNTAS
%No se si poner ejemplos de software real
%No se si con esto pueden preguntarme por las soluciones software mas avanzadas como F1 MANAGER
%No se si es demasiado poco lo que he escrito
%No se si he puesto demasiados requisitos.
%Tengo la sensacion de que hablo mucho y no digo nada.
%no se si estoy poniendo mi software como el mejor del mundo
% Quizas no he puesto las sificientes fuentes en la itnroudccion
% Arreglar nombre que sale arriba de la linea, que pone indice general, cuando no lo es
\documentclass[a4paper,11pt]{book}
%\documentclass[a4paper,twoside,11pt,titlepage]{book}
\usepackage{listings}
\usepackage[utf8]{inputenc}
\usepackage[spanish]{babel}
\usepackage{float}


\usepackage{xcolor}
\setcounter{secnumdepth}{3}
\setcounter{tocdepth}{3}
% \usepackage[style=list, number=none]{glossary} %
%\usepackage{titlesec}
%\usepackage{pailatino}

\decimalpoint
\usepackage{dcolumn}
\newcolumntype{.}{D{.}{\esperiod}{-1}}
\makeatletter
\addto\shorthandsspanish{\let\esperiod\es@period@code}
\makeatother


%\usepackage[chapter]{algorithm}
\RequirePackage{verbatim}
%\RequirePackage[Glenn]{fncychap}
\usepackage{fancyhdr}
\usepackage{graphicx}
\usepackage{afterpage}

\usepackage{longtable}

\usepackage[pdfborder={000}]{hyperref} %referencia

% ********************************************************************
% Re-usable information
% ********************************************************************
\newcommand{\myTitle}{Título del proyecto\xspace}
\newcommand{\myDegree}{Grado en Ingeniería Informática\xspace}
\newcommand{\myName}{Andrés Merlo Trujillo\xspace}
\newcommand{\myProf}{Luis López Escudero\xspace}
%\newcommand{\myOtherProf}{Nombre Apllido1 Apellido2 (tutor2)\xspace}
%\newcommand{\mySupervisor}{Put name here\xspace}  NO USADO
\newcommand{\myFaculty}{Escuela Técnica Superior de Ingenierías Informática y de
Telecomunicación\xspace}
\newcommand{\myFacultyShort}{E.T.S. de Ingenierías Informática y de
Telecomunicación\xspace}
\newcommand{\myDepartment}{Departamento de Lenguaje y Sistemas Informáticos\xspace}
\newcommand{\myUni}{\protect{Universidad de Granada}\xspace}
\newcommand{\myLocation}{Granada\xspace}
\newcommand{\myTime}{\today\xspace}
\newcommand{\myVersion}{Version 0.1\xspace}


\hypersetup{
pdfauthor = {\myName (andresmerlo@correo.ugr.es)},
pdftitle = {\myTitle},
pdfsubject = {},
pdfkeywords = {Unreal, Simulación, IA },
pdfcreator = {LaTeX con el paquete ....},
pdfproducer = {pdflatex}
}

%\hyphenation{}


%\usepackage{doxygen/doxygen}
%\usepackage{pdfpages}
\usepackage{url}
\usepackage{colortbl,longtable}
\usepackage[stable]{footmisc}
%\usepackage{index}

%\makeindex
%\usepackage[style=long, cols=2,border=plain,toc=true,number=none]{glossary}
% \makeglossary

% Definición de comandos que me son tiles:
%\renewcommand{\indexname}{Índice alfabético}
%\renewcommand{\glossaryname}{Glosario}

\pagestyle{fancy}
\fancyhf{}
\fancyhead[LO]{\leftmark}
\fancyhead[RE]{\rightmark}
\fancyhead[RO,LE]{\textbf{\thepage}}
\renewcommand{\chaptermark}[1]{\markboth{\textbf{#1}}{}}
\renewcommand{\sectionmark}[1]{\markright{\textbf{\thesection. #1}}}

\setlength{\headheight}{1.5\headheight}

\newcommand{\HRule}{\rule{\linewidth}{0.5mm}}
%Definimos los tipos teorema, ejemplo y definición podremos usar estos tipos
%simplemente poniendo \begin{teorema} \end{teorema} ...
\newtheorem{teorema}{Teorema}[chapter]
\newtheorem{ejemplo}{Ejemplo}[chapter]
\newtheorem{definicion}{Definición}[chapter]

\definecolor{gray97}{gray}{.97}
\definecolor{gray75}{gray}{.75}
\definecolor{gray45}{gray}{.45}
\definecolor{gray30}{gray}{.94}

\lstset{ frame=Ltb,
     framerule=0.5pt,
     aboveskip=0.5cm,
     framextopmargin=3pt,
     framexbottommargin=3pt,
     framexleftmargin=0.1cm,
     framesep=0pt,
     rulesep=.4pt,
     backgroundcolor=\color{gray97},
     rulesepcolor=\color{black},
     %
     stringstyle=\ttfamily,
     showstringspaces = false,
     basicstyle=\scriptsize\ttfamily,
     commentstyle=\color{gray45},
     keywordstyle=\bfseries,
     %
     numbers=left,
     numbersep=6pt,
     numberstyle=\tiny,
     numberfirstline = false,
     breaklines=true,
   }
 
% minimizar fragmentado de listados
\lstnewenvironment{listing}[1][]
   {\lstset{#1}\pagebreak[0]}{\pagebreak[0]}

\lstdefinestyle{CodigoC}
   {
	basicstyle=\scriptsize,
	frame=single,
	language=C,
	numbers=left
   }
\lstdefinestyle{CodigoC++}
   {
	basicstyle=\small,
	frame=single,
	backgroundcolor=\color{gray30},
	language=C++,
	numbers=left
   }

 
\lstdefinestyle{Consola}
   {basicstyle=\scriptsize\bf\ttfamily,
    backgroundcolor=\color{gray30},
    frame=single,
    numbers=none
   }


\newcommand{\bigrule}{\titlerule[0.5mm]}


%Para conseguir que en las páginas en blanco no ponga cabecerass
\makeatletter
\def\clearpage{%
  \ifvmode
    \ifnum \@dbltopnum =\m@ne
      \ifdim \pagetotal <\topskip
        \hbox{}
      \fi
    \fi
  \fi
  \newpage
  \thispagestyle{empty}
  \write\m@ne{}
  \vbox{}
  \penalty -\@Mi
}
\makeatother

\usepackage{pdfpages}
\begin{document}
\input{portada/portada}
\chapter*{}
%\thispagestyle{empty}
%\cleardoublepage

%\thispagestyle{empty}

\input{portada/portada_2}



\cleardoublepage
\thispagestyle{empty}

\begin{center}
{\large\bfseries \myTitle}\\
\end{center}
\begin{center}
Andrés Merlo Trujillo\\
\end{center}

%\vspace{0.7cm}
\noindent{\textbf{Palabras clave}: Unreal, Simulación, IA }\\

\vspace{0.7cm}
\noindent{\textbf{Resumen}}\\

Se pretende desarrollar una aplicación gráfica utilizando el motor gráfico \textit{Unreal Engine}, cuyo objetivo es de poder simular carreras de coches. Esta simulación dotará a los pilotos virtuales de la capacidad de tomar decisiones realistas y de cometer errores, basándose en las condiciones de cada piloto durante la carrera, que pueden variar dependiendo de las condiciones externas o ser modificadas por el usuario en tiempo real. Además, se podrán configurar ajustes adicionales antes de comenzar la carrera, con el objetivo de hacer la simulación lo más personalizable posible.

\cleardoublepage


\thispagestyle{empty}


\begin{center}
{\large\bfseries Car racing simulator in Unreal Engine}\\
\end{center}
\begin{center}
Andrés Merlo Trujillo\\
\end{center}

%\vspace{0.7cm}
\noindent{\textbf{Keywords}: Unreal, Simulation, AI }\\

\vspace{0.7cm}
\noindent{\textbf{Abstract}}\\

The aim is to develop a graphics application using the graphics engine \textit{Unreal Engine}, whose objective is to be able to simulate car races. This simulation will provide virtual pilots with the ability to make realistic decisions and make mistakes, based on the conditions of each pilot during the race, which can vary depending on external conditions or be modified by the user in real time. Furthermore, additional settings can be configured before starting the race, with the aim of making the simulation as customizable as possible.

\chapter*{}
\thispagestyle{empty}

\noindent\rule[-1ex]{\textwidth}{2pt}\\[4.5ex]

Yo, \textbf{Andrés Merlo Trujillo}, alumno de la titulación Ingeniería Informática de la \textbf{Escuela Técnica Superior
de Ingenierías Informática y de Telecomunicación de la Universidad de Granada}, con DNI 77147239H, autorizo la
ubicación de la siguiente copia de mi Trabajo Fin de Grado en la biblioteca del centro para que pueda ser
consultada por las personas que lo deseen.

\vspace{6cm}

\noindent Fdo: Andrés Merlo Trujillo

\vspace{2cm}

\begin{flushright}
Granada a 11 de julio de 2023.
\end{flushright}


\chapter*{}
\thispagestyle{empty}

\noindent\rule[-1ex]{\textwidth}{2pt}\\[4.5ex]

D. \textbf{Luis López Escudero}, Profesor del Área del Departamento de Lenguajes y Sistemas Informáticos de la Universidad de Granada.

\vspace{0.5cm}

D. \textbf{Germán Arroyo Moreno}, Profesor del Área del Departamento de Lenguajes y Sistemas Informáticos de la Universidad de Granada.


\vspace{0.5cm}

\textbf{Informan:}

\vspace{0.5cm}

Que el presente trabajo, titulado \textit{\textbf{\myTitle}},
ha sido realizado bajo su supervisión por \textbf{Andrés Merlo Trujillo}, y autorizamos la defensa de dicho trabajo ante el tribunal
que corresponda.

\vspace{0.5cm}

Y para que conste, expiden y firman el presente informe en Granada a 11 de julio de 2023.

\vspace{1cm}

\textbf{Los directores:}

\vspace{5cm}

\noindent \textbf{Luis López Escudero \ \ \ \ \ Germán Arroyo Moreno}

\chapter*{Agradecimientos}
\thispagestyle{empty}

       \vspace{1cm}


A mi familia y a Luis por brindarme su apoyo durante todo el desarrollo del proyecto.



%\frontmatter
\tableofcontents
\newpage
%\listoffigures
%\listoftables
%
%\mainmatter
% \setlength{\parskip}{5pt}
\chapter{Introducción}
\section{Introducción y Motivación}
%PALABRAS REPETIDAS, REVISAR
%gran cantidad
%no se repite, pero "cosas" queda muy feo

% \textcolor{red}{El mundo del motor es un deporte con una gran cantidad de seguidores y en el que se invierte una gran cantidad de dinero en tecnologías de todo tipo para poder tener una ventaja competitiva, incluyendo simuladores. Estos simuladores permiten a los equipos de carreras realizar un amplio abanico de cosas: entrenar a los pilotos antes de la temporada, modificar parámetros para exprimir el máximo rendimiento del vehículo o simular carreras para analizar cuál será la estrategia que les pueda llevar a la victoria. }

\textcolor{orange}{El mundo del motor es un deporte con una amplia base de seguidores y en el que se invierte una gran cantidad de dinero en simuladores de todo tipo para poder tener una ventaja competitiva. Estos simuladores permiten a los equipos de carreras realizar una extensa variedad de actividades: entrenar a los pilotos antes y durante la temporada, modificar parámetros para lograr el máximo rendimiento del vehículo o simular carreras para analizar cuál será la estrategia más adecuada para alcanzar la victoria.}

\bigskip

%PALABRAS REPES
%mucho poder
%como

\textcolor{orange}{Este proyecto se centrará en la simulación de gestión de carreras, sector en el que existen pocas opciones para el público general y que es cada vez más demandado para aprender la manera en que las distintas estrategias pueden afectar al desenlace de una carrera y como diferentes situaciones influyen en la capacidad de los pilotos. Estos factores pueden afectar a la tasa de errores producidos durante la carrera.}

\bigskip

%mejorar wording eleccion de varios coches
%mejorar el por ejemplo, que solo es uno y parece que puede haber mas
\textcolor{orange}{Se añadirán ciertas características procedentes de videojuegos con elementos de simulación a la aplicación gráfica, como la posibilidad de elegir entre varios tipos de vehículos (\textit{Gran Turismo B-Spec}), por mencionar un ejemplo.}

\bigskip

%quizas no repetir lo de la api, se sobreentiende con lo otro
\textcolor{orange}{La motivación detrás de realizar esta aplicación es profundizar en los conocimientos previamente adquiridos en otras asignaturas del manejo de \textit{Unreal Engine} y en el aprendizaje de su API, enfocándolo al ámbito de las carreras y de los vehículos autónomos, área que me resulta muy interesante. Asimismo, se busca profundizar en el estudio de los distintos algoritmos de navegación para los vehículos del simulador, intentando escoger el que mejor se adapte a las necesidades del proyecto y que ofrezca buenos resultados.}

\section{Objetivos}

\textcolor{red}{En este proyecto se pretende desarrollar un simulador de gestión de carreras multidisciplinar y personalizable, en el que los pilotos tendrán un conjunto de capacidades que serán modificadas por las condiciones de la carrera o por el usuario en tiempo real, haciendo que el piloto pueda cometer más errores o menos si su estado mejora. También existirá la opción de poder modificar la velocidad de simulación, con el objetivo de obtener los resultados de manera más rápida.}

\bigskip

\textcolor{red}{Además, la aplicación permitirá modificar diversos parámetros antes de la carrera, con el propósito de personalizar la simulación lo máximo posible. Cabe destacar que estos parámetros no podrán ser cambiados de nuevo, debido a que no tendría demasiado sentido y podrían hacer que los resultados de la simulación no sean del todo precisos.}

% Como se ha mencionado en el apartado anterior, el objetivo principal de este proyecto es de ofrecer un simulador de carreras multidisciplinar y con un grado alto de personalización en las variables de cada piloto.

\bigskip

% Además, el proyecto debe cumplir los siguientes objetivos mínimos relativos a cambios antes de ejecutar la simulación:

\textcolor{red}{Siendo más específicos, los objetivos mínimos relativos a alteraciones antes de ejecutar la simulación son:}

\begin{itemize}
   \item Cambio de número de pilotos.
   \item Cambio de número de vueltas.
   \item Cambios de las aptitudes de los pilotos; es decir, el nivel máximo de cada condición que puede alcanzar.
   \item Cambio de la velocidad máxima de los vehículos.
   \item Permitir (opcionalmente) que haya variación de prestaciones entre vehículos.
   \item Cambio del tipo de vehículo.
\end{itemize}

\bigskip

\textcolor{red}{En cuanto a cambios durante la simulación, debe cumplir estos objetivos:}

\begin{itemize}
   \item Seleccionar piloto de la lista de pilotos.
   \item Obtener las condiciones actuales del piloto seleccionado.
   \item Modificar las condiciones actuales del piloto seleccionado.
   \item Cambiar la velocidad de simulación.
\end{itemize}

Conviene resaltar que todos estos objetivos y algunos opcionales serán explicados en detalle en apartados siguientes.

\section{Estado del arte}
Como ya se ha descrito anteriormente, este proyecto hará uso de \textit{Unreal Engine} como motor gráfico y \textit{framework} para la realización del simulador especificado. En cuanto a la forma en la que los pilotos se moverán por el circuito, he decidido usar el algoritmo de navegación \textbf{PONER AQUÍ EL ALGORITMO DE NAVEGACIÓN}.

\bigskip

A continuación, voy a explicar las distintas opciones que hay disponibles en motores gráficos y en algoritmos de navegación y explicar el motivo de la elección realizada. Separaré esto en dos subsecciones.

\subsection{Motor gráfico}
En el mercado hay gran variedad de motores gráficos de videojuegos, los más conocidos son los siguientes:

\subsubsection{Unreal Engine}

\begin{figure}[H]
   \centering
   \includegraphics[width=0.2\textwidth]{imagenes/UE_LOGO.png}
   \caption{Logotipo de \textit{Unreal Engine}.}
   \vspace{10pt}
   \footnotesize{Fuente: \url{https://es.wikipedia.org/wiki/Unreal_Engine}}
\end{figure}

Unreal Engine \cite{unreal} es una plataforma de desarrollo que permite, entre otras cosas: el desarrollo de videojuegos, creación de contenido para programas y películas, simulación, etc. Posee un soporte para un gran número de plataformas, tanto consolas como móviles.

\bigskip

Unreal incluye \textit{Blueprint Visual Scripting}, el cual es un sistema de \textit{scripting} visual que hace uso de nodos para programar las distintas partes del sistema. Permite programar de manera visual sin la necesidad de conocer la \textit{API} de C++.

\bigskip

Además, a partir de la versión 5.0, Unreal incluye diversas tecnologías como: 

\begin{itemize}
   \item TSR \textit{(Temporal Super Resolution)}, mejorando el rendimiento haciendo que el juego se renderice internamente a una resolución menor y luego sea reescalado al tamaño deseado.
   \item Nanite, que permite tener objetos con una gran cantidad de polígonos en la pantalla, 
   \item Lumen, que genera una iluminación más realista. %[https://www.unrealengine.com/en-US/unreal-engine-5]
\end{itemize}

% Todo esto permite tener un resultado a nivel gráfico muy convincente sin necesitar dedicar demasiado trabajo en este apartado y sin requerir un hardware demasiado potente.
Estas tecnologías permiten obtener un resultado gráfico muy convincente, partiendo de un modelo correctamente modelado y texturizado. Asimismo, la inclusión de \textit{TSR} posibilita escalar la aplicación a resoluciones mayores de lo que el hardware es capaz de ejecutar.

\bigskip

Unreal también incluye una tienda con una gran cantidad de recursos, permitiendo simplificar mucho el desarrollo de videojuegos.

\bigskip

Sin embargo, no tiene una licencia de código abierto, pero su código fuente es accesible a través de un repositorio de GitHub. %[https://forums.unrealengine.com/t/is-unreal-open-source/473823]

\subsubsection{Unity}

\begin{figure}[H]
   \centering
   \includegraphics[width=0.4\textwidth]{imagenes/UNITY_LOGO.png}
   \caption{Logotipo de \textit{Unity}.}
   \vspace{10pt}
   \footnotesize{Fuente: \url{https://es.wikipedia.org/wiki/Unity_(motor_de_videojuego)}}
\end{figure}

Unity \cite{unity} es una plataforma de desarrollo que permite: desarrollar videojuegos, visualizar construcciones en el ámbito de la arquitectura, para uso en la industria automotriz y para la creación de películas y series.

\bigskip

Entre sus características se encuentra:

\begin{itemize}
   \item Uso de C\# como lenguaje de programación.
   \item Motor de físicas 2D separado del de físicas en 3D.
   \item Posee un \textit{Scriptable Rendering Pipeline} altamente personalizable, permitiendo mediante el uso de scripts escritos en C\# personalizar sombras, iluminación, oclusión ambiental, entre otras.
   \item Soporte para un gran número de plataformas, incluyendo consolas y móviles, aparte de Windows, macOS y Linux.
   \item Incluye un lenguaje de programación visual, similar a \textit{Unreal} que permite programar sin necesitar conocer la \textit{API}.
\end{itemize}
   
Además, Unity tiene varias versiones del programa, incluyendo uno gratuito y los demás de pago, con prioridad en la asistencia y un mayor abanico de herramientas.

\subsubsection{Godot}

\begin{figure}[H]
   \centering
   \includegraphics[width=0.4\textwidth]{imagenes/GODOT_LOGO.png}
   \caption{Logotipo de \textit{Godot}.}
   \vspace{10pt}
   \footnotesize{Fuente: \url{https://es.wikipedia.org/wiki/Godot}}
\end{figure}

Godot \cite{godot} es un motor gráfico gratuito y multiplataforma de código abierto que permite crear videojuegos en 2D y 3D. 

\bigskip

Entre las características más notables se encuentran las siguientes:

\begin{itemize}
   \item Motor gráfico 2D dedicado. 
   \item Programación usando GDScript, C\# o C++ de manera oficial. No obstante, se puede programar usando otro lenguaje como Rust, Python o JavaScript. 
   \item Soporte para un gran número de plataformas, incluyendo móviles y consolas.
   \item Incluye un motor de físicas (solo para el motor gráfico 3D), aunque es algo más simple que el de sus competidores.
\end{itemize}

\bigskip 

A partir de la versión 4, Godot ha dejado de incluir el lenguaje visual, debido a su falta de uso, por lo que solo es posible utilizar lenguajes de programación convencionales \cite{godot-no-visual}.

\bigskip

Además, Godot tiene una biblioteca con una gran cantidad de recursos para utilizar durante el desarrollo de un videojuego.

% CryEngine:

\subsubsection{Elección final}

Al final, he decidido utilizar Unreal Engine debido a que la versión gratuita incluye toda la funcionalidad, ofrece unos buenos resultados a nivel visual sin requerir demasiado esfuerzo, es el más potente en cuanto a programación visual y es con el que más familiarizado estoy de todas las opciones citadas anteriormente.

\subsection{Algoritmo de navegación}
En cuanto a algoritmos de navegación para los vehículos, los más destacados son:

\begin{itemize}
   \item \textbf{Aprendizaje por refuerzo:} Es una técnica de aprendizaje automático en la que el agente aprende a tomar decisiones en un entorno mediante la interacción a través de acciones %(en este caso es: acelerador, freno y volante)
   y recibiendo una recompensa en caso de acercarse al objetivo deseado o de hacerlo bien. 
   
   Es importante obtener un equilibrio entre exploración y explotación, para que el agente sea capaz de actuar ante situaciones desconocidas de manera correcta. En caso de estar desequilibrado, puede darse la situación en el que se produzca un máximo local y no sea capaz de descubrir la mejor estrategia.

   
   Para evitar esto, se suelen utilizar métodos de equilibrio. Uno de ellos es el denominado \textit{epsilon-greedy}, que permite controlar la cantidad de exploración mediante un parámetro \textit{epsilon}, que determina la aleatoriedad en la selección de acciones \cite{10.1007/978-3-642-24455-1_33}, permitiendo elegir en función de dicho parámetro una acción aleatoria o la mejor hasta el momento.

   Una desventaja que tiene es la necesidad de entrenar el modelo, lo cual puede ser un proceso largo y complicado, pero con el suficiente tiempo y con un conjunto de recompensas y castigos bien realizado, es una excelente técnica para resolver este tipo de problema.

   \textcolor{orange}{El algoritmo \textit{Q-Learning} es uno de los más utilizados para espacios discretos. Consiste en una función Q, que permite al agente estimar el valor esperado de una acción dado un estado. Dicha función se implementa mediante una tabla denominada \textit{Q-Values}, en la que se almacena el valor esperado de las recompensas para cada estado y acción posible. Para evitar problemas de equilibrio, se utilizan métodos como el \textit{epsilon-greedy}, ampliamente utilizado y que ofrece buenos resultados.}

   No obstante, si el entorno de aprendizaje es continuo, una tabla de \textit{Q-Values} no es adecuada debido a la gran cantidad de posibles valores y el alto costo computacional en procesamiento y memoria. En su lugar, se puede sustituir la tabla por una red neuronal, como el algoritmo \textit{Deep Q-Network} (DQN), para trabajar en este tipo de espacios \cite{coulom:tel-00003985}.

   \textcolor{orange}{Para este proyecto, se podría implementar el algoritmo DQN. La red neuronal tendría como entrada varios sensores de colisión y de límites de pista, y como salida la cantidad de acelerador, freno y giro del volante.}

   \item \textbf{Mediante reglas:} Otra forma de resolver este problema es mediante reglas. Son un conjunto de acciones que se aplican cuando se cumple una determinada condición y suelen estar ordenadas por prioridad. Pueden ser representadas mediante un árbol, donde cada acción se encuentra en una hoja y las preguntas o condiciones que las activan son las ramas (aristas). La prioridad de las acciones se establece según su posición en el árbol, siendo las más importantes las que se encuentran en niveles superiores.
   
   Esto tiene la ventaja de ser más sencillo de implementar que el de aprendizaje por refuerzo, pero suele tener peores resultados, al poder darse el caso de no haber priorizado bien las reglas o no haber tenido en cuenta algún caso especial.


   Aplicado a este proyecto, se podría diseñar como un conjunto de funciones que manejen cada posible control del vehículo y se ejecuten de manera periódica. De esta forma, dependiendo del entorno y del estado del piloto, se aplicarán distintas salidas para guiar el vehículo por el circuito.

   \item \textbf{Máquina de estados:} Este algoritmo implementa una máquina de estados finitos mediante una estructura de datos, donde cada nodo representa un estado en el que se encuentra el piloto y cada arco representa la transición a otro estado. 
   
   Una de las ventajas que tiene es que suele ser más simple describir el comportamiento en agentes más complejos y además, suelen ser más fáciles de modificar que los basados en reglas.

   En este proyecto se podrían implementar estados como: ``Seguir línea'', ``Adelantar'', ``Comienzo de carrera'' y ``Esquivar obstáculo''. Dependiendo del entorno y del estado del piloto, el agente puede pasar a otro estado más adecuado.

   \item \textbf{Algoritmo A* (A Estrella)}: Algoritmo de búsqueda de caminos en grafos que se utiliza para encontrar la ruta óptima. Utiliza una función heurística para guiar la búsqueda hacia la ruta óptima y es más eficiente que otros algoritmos similares, como el de Dijkstra.
   
   Si se desea aplica este algoritmo al proyecto, es necesario discretizar el espacio, lo que puede provocar una pérdida de precisión. Además, se debería aplicar periódicamente en todos los vehículos de la simulación, lo que puede ser computacionalmente costoso.

   No obstante, una ventaja significativa del algoritmo es que se puede incluir el estado del piloto en la función heurística, lo que facilita su integración en la aplicación. También se puede aplicar una función de suavizado a la trayectoria resultante para mejorar su precisión. Dado que el proyecto no se centra en la eficiencia computacional, el algoritmo A* podría ser considerado como una opción viable para la navegación por el circuito.
\end{itemize}
%\input{capitulos/01_Introduccion}
%
%\input{capitulos/02_EspecificacionRequisitos}
%
%\input{capitulos/03_Planificacion}
%
%\input{capitulos/04_Analisis}
%
%\input{capitulos/05_Diseno}
%
%\input{capitulos/06_Implementacion}
%
%\input{capitulos/07_Pruebas}
%
%\input{capitulos/08_Conclusiones}
%
%%\chapter{Conclusiones y Trabajos Futuros}
%
%
%%\nocite{*}
\bibliography{bibliografia/bibliografia}\addcontentsline{toc}{chapter}{Bibliografía}
% \bibliographystyle{miunsrturl}
\bibliographystyle{plainurl}
%
%\appendix
%\input{apendices/manual_usuario/manual_usuario}
%%\input{apendices/paper/paper}
%\input{glosario/entradas_glosario}
% \addcontentsline{toc}{chapter}{Glosario}
% \printglossary
\chapter*{}
\thispagestyle{empty}

\end{document}
