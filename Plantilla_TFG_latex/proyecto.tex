%PREGUNTAS
%No se si poner ejemplos de software real
%No se si con esto pueden preguntarme por las soluciones software mas avanzadas como F1 MANAGER
%No se si es demasiado poco lo que he escrito
%No se si he puesto demasiados requisitos.
%Tengo la sensacion de que hablo mucho y no digo nada.

\documentclass[a4paper,11pt]{book}
%\documentclass[a4paper,twoside,11pt,titlepage]{book}
\usepackage{listings}
\usepackage[utf8]{inputenc}
\usepackage[spanish]{babel}

% \usepackage[style=list, number=none]{glossary} %
%\usepackage{titlesec}
%\usepackage{pailatino}

\decimalpoint
\usepackage{dcolumn}
\newcolumntype{.}{D{.}{\esperiod}{-1}}
\makeatletter
\addto\shorthandsspanish{\let\esperiod\es@period@code}
\makeatother


%\usepackage[chapter]{algorithm}
\RequirePackage{verbatim}
%\RequirePackage[Glenn]{fncychap}
\usepackage{fancyhdr}
\usepackage{graphicx}
\usepackage{afterpage}

\usepackage{longtable}

\usepackage[pdfborder={000}]{hyperref} %referencia

% ********************************************************************
% Re-usable information
% ********************************************************************
\newcommand{\myTitle}{Título del proyecto\xspace}
\newcommand{\myDegree}{Grado en Ingeniería Informática\xspace}
\newcommand{\myName}{Andrés Merlo Trujillo\xspace}
\newcommand{\myProf}{Luis López Escudero\xspace}
%\newcommand{\myOtherProf}{Nombre Apllido1 Apellido2 (tutor2)\xspace}
%\newcommand{\mySupervisor}{Put name here\xspace}  NO USADO
\newcommand{\myFaculty}{Escuela Técnica Superior de Ingenierías Informática y de
Telecomunicación\xspace}
\newcommand{\myFacultyShort}{E.T.S. de Ingenierías Informática y de
Telecomunicación\xspace}
\newcommand{\myDepartment}{Departamento de Lenguaje y Sistemas Informáticos\xspace}
\newcommand{\myUni}{\protect{Universidad de Granada}\xspace}
\newcommand{\myLocation}{Granada\xspace}
\newcommand{\myTime}{\today\xspace}
\newcommand{\myVersion}{Version 0.1\xspace}


\hypersetup{
pdfauthor = {\myName (andresmerlo@correo.ugr.es)},
pdftitle = {\myTitle},
pdfsubject = {},
pdfkeywords = {unreal, simulador, ia, },
pdfcreator = {LaTeX con el paquete ....},
pdfproducer = {pdflatex}
}

%\hyphenation{}


%\usepackage{doxygen/doxygen}
%\usepackage{pdfpages}
\usepackage{url}
\usepackage{colortbl,longtable}
\usepackage[stable]{footmisc}
%\usepackage{index}

%\makeindex
%\usepackage[style=long, cols=2,border=plain,toc=true,number=none]{glossary}
% \makeglossary

% Definición de comandos que me son tiles:
%\renewcommand{\indexname}{Índice alfabético}
%\renewcommand{\glossaryname}{Glosario}

\pagestyle{fancy}
\fancyhf{}
\fancyhead[LO]{\leftmark}
\fancyhead[RE]{\rightmark}
\fancyhead[RO,LE]{\textbf{\thepage}}
\renewcommand{\chaptermark}[1]{\markboth{\textbf{#1}}{}}
\renewcommand{\sectionmark}[1]{\markright{\textbf{\thesection. #1}}}

\setlength{\headheight}{1.5\headheight}

\newcommand{\HRule}{\rule{\linewidth}{0.5mm}}
%Definimos los tipos teorema, ejemplo y definición podremos usar estos tipos
%simplemente poniendo \begin{teorema} \end{teorema} ...
\newtheorem{teorema}{Teorema}[chapter]
\newtheorem{ejemplo}{Ejemplo}[chapter]
\newtheorem{definicion}{Definición}[chapter]

\definecolor{gray97}{gray}{.97}
\definecolor{gray75}{gray}{.75}
\definecolor{gray45}{gray}{.45}
\definecolor{gray30}{gray}{.94}

\lstset{ frame=Ltb,
     framerule=0.5pt,
     aboveskip=0.5cm,
     framextopmargin=3pt,
     framexbottommargin=3pt,
     framexleftmargin=0.1cm,
     framesep=0pt,
     rulesep=.4pt,
     backgroundcolor=\color{gray97},
     rulesepcolor=\color{black},
     %
     stringstyle=\ttfamily,
     showstringspaces = false,
     basicstyle=\scriptsize\ttfamily,
     commentstyle=\color{gray45},
     keywordstyle=\bfseries,
     %
     numbers=left,
     numbersep=6pt,
     numberstyle=\tiny,
     numberfirstline = false,
     breaklines=true,
   }
 
% minimizar fragmentado de listados
\lstnewenvironment{listing}[1][]
   {\lstset{#1}\pagebreak[0]}{\pagebreak[0]}

\lstdefinestyle{CodigoC}
   {
	basicstyle=\scriptsize,
	frame=single,
	language=C,
	numbers=left
   }
\lstdefinestyle{CodigoC++}
   {
	basicstyle=\small,
	frame=single,
	backgroundcolor=\color{gray30},
	language=C++,
	numbers=left
   }

 
\lstdefinestyle{Consola}
   {basicstyle=\scriptsize\bf\ttfamily,
    backgroundcolor=\color{gray30},
    frame=single,
    numbers=none
   }


\newcommand{\bigrule}{\titlerule[0.5mm]}


%Para conseguir que en las páginas en blanco no ponga cabecerass
\makeatletter
\def\clearpage{%
  \ifvmode
    \ifnum \@dbltopnum =\m@ne
      \ifdim \pagetotal <\topskip
        \hbox{}
      \fi
    \fi
  \fi
  \newpage
  \thispagestyle{empty}
  \write\m@ne{}
  \vbox{}
  \penalty -\@Mi
}
\makeatother

\usepackage{pdfpages}
\begin{document}
\input{portada/portada}
\chapter*{}
%\thispagestyle{empty}
%\cleardoublepage

%\thispagestyle{empty}

\input{portada/portada_2}



\cleardoublepage
\thispagestyle{empty}

\begin{center}
{\large\bfseries \myTitle}\\
\end{center}
\begin{center}
Andrés Merlo Trujillo\\
\end{center}

%\vspace{0.7cm}
\noindent{\textbf{Palabras clave}: Unreal, Simulación, IA }\\

\vspace{0.7cm}
\noindent{\textbf{Resumen}}\\

Se pretende desarrollar una aplicación gráfica utilizando el motor gráfico \textit{Unreal Engine}, cuyo objetivo es de poder simular carreras de coches. Esta simulación dotará a los pilotos virtuales de la capacidad de tomar decisiones realistas y de cometer errores, basándose en las condiciones de cada piloto durante la carrera, que pueden variar dependiendo de las condiciones externas o ser modificadas por el usuario en tiempo real. Además, se podrán configurar ajustes adicionales antes de comenzar la carrera, con el objetivo de hacer la simulación lo más personalizable posible.

\cleardoublepage


\thispagestyle{empty}


\begin{center}
{\large\bfseries Car racing simulator in Unreal Engine}\\
\end{center}
\begin{center}
Andrés Merlo Trujillo\\
\end{center}

%\vspace{0.7cm}
\noindent{\textbf{Keywords}: Unreal, Simulation, AI }\\

\vspace{0.7cm}
\noindent{\textbf{Abstract}}\\

The aim is to develop a graphics application using the graphics engine \textit{Unreal Engine}, whose objective is to be able to simulate car races. This simulation will provide virtual pilots with the ability to make realistic decisions and make mistakes, based on the conditions of each pilot during the race, which can vary depending on external conditions or be modified by the user in real time. Furthermore, additional settings can be configured before starting the race, with the aim of making the simulation as customizable as possible.

\chapter*{}
\thispagestyle{empty}

\noindent\rule[-1ex]{\textwidth}{2pt}\\[4.5ex]

Yo, \textbf{Andrés Merlo Trujillo}, alumno de la titulación Ingeniería Informática de la \textbf{Escuela Técnica Superior
de Ingenierías Informática y de Telecomunicación de la Universidad de Granada}, con DNI 77147239H, autorizo la
ubicación de la siguiente copia de mi Trabajo Fin de Grado en la biblioteca del centro para que pueda ser
consultada por las personas que lo deseen.

\vspace{6cm}

\noindent Fdo: Andrés Merlo Trujillo

\vspace{2cm}

\begin{flushright}
Granada a 11 de julio de 2023.
\end{flushright}


\chapter*{}
\thispagestyle{empty}

\noindent\rule[-1ex]{\textwidth}{2pt}\\[4.5ex]

D. \textbf{Luis López Escudero}, Profesor del Área del Departamento de Lenguajes y Sistemas Informáticos de la Universidad de Granada.

\vspace{0.5cm}

D. \textbf{Germán Arroyo Moreno}, Profesor del Área del Departamento de Lenguajes y Sistemas Informáticos de la Universidad de Granada.


\vspace{0.5cm}

\textbf{Informan:}

\vspace{0.5cm}

Que el presente trabajo, titulado \textit{\textbf{\myTitle}},
ha sido realizado bajo su supervisión por \textbf{Andrés Merlo Trujillo}, y autorizamos la defensa de dicho trabajo ante el tribunal
que corresponda.

\vspace{0.5cm}

Y para que conste, expiden y firman el presente informe en Granada a 11 de julio de 2023.

\vspace{1cm}

\textbf{Los directores:}

\vspace{5cm}

\noindent \textbf{Luis López Escudero \ \ \ \ \ Germán Arroyo Moreno}

\chapter*{Agradecimientos}
\thispagestyle{empty}

       \vspace{1cm}


A mi familia y a Luis por brindarme su apoyo durante todo el desarrollo del proyecto.



%\frontmatter
\tableofcontents
\newpage
%\listoffigures
%\listoftables
%
%\mainmatter
% \setlength{\parskip}{5pt}

\section{Introducción y Motivación}

El mundo del motor siempre ha sido un deporte que despierta la curiosidad en muchas personas a lo largo del mundo, ya sea por la innovación tecnológica continua que se produce cada año en los vehículos, por sentir la adrenalina de conducir un coche de carreras o incluso por la satisfacción personal de ser capaz de dar órdenes a un piloto para llevarlo a la victoria, seleccionando las mejores estrategias en cada momento. Este proyecto se centrará en esto último.

\bigskip

Gracias al avance de la informática, se puede obtener más información de diversos factores en tiempo real, tales como: estado general del vehículo, condiciones del piloto e incluso estado de la pista. Con todos estos datos, se puede tomar la mejor decisión para llevar el piloto y el equipo a la victoria. Además, son cada vez más los entusiastas que recurren al uso de simuladores para poder probar de primera mano estas experiencias.

\bigskip

No obstante, el número de simuladores de gestión de carreras es relativamente pequeño y poco variado, normalmente centrándose en ciertas disciplinas concretas, sin permitir la inclusión de más tipos. Aparte de esto, las aplicaciones existentes suelen ser relativamente caras. 

\bigskip
%No se si poner ejemplos de software real
%No se si con esto pueden preguntarme por las soluciones software mas avanzadas como F1 MANAGER
%No se si es demasiado poco lo que he escrito
Un ejemplo de aplicaciones existentes es la de \textit{Gran Turismo B-Spec}, que permite dar órdenes a un piloto de la carrera y dependiendo de la frecuencia de dichas órdenes y la posición en la parrilla, pueden variar sus parámetros de aguante físico y mental.

\bigskip

En este proyecto se pretende ofrecer un simulador relativamente realista con diversos parámetros, incluyendo algunos similares a los mencionados anteriormente, y que además ofrece la posibilidad de modificar los parámetros de todos los pilotos en tiempo real, aparte de ser modificados por el entorno de la carrera.

\bigskip

Además, la aplicación permitirá ajustar las aptitudes, el número de pilotos, la velocidad máxima de los coches, el número de los mismos y el número de vueltas y la hora del día a la que se va a disputar la carrera.

\bigskip

Por último, cabe decir que el \textit{software} permitirá también modificar otros aspectos ajenos a los que puedan modificar el resultado de la carrera, tales como la velocidad de simulación.

\section{otra}
\subsection{title}
\subsubsection{asd}

%\input{capitulos/01_Introduccion}
%
%\input{capitulos/02_EspecificacionRequisitos}
%
%\input{capitulos/03_Planificacion}
%
%\input{capitulos/04_Analisis}
%
%\input{capitulos/05_Diseno}
%
%\input{capitulos/06_Implementacion}
%
%\input{capitulos/07_Pruebas}
%
%\input{capitulos/08_Conclusiones}
%
%%\chapter{Conclusiones y Trabajos Futuros}
%
%
%%\nocite{*}
%\bibliography{bibliografia/bibliografia}\addcontentsline{toc}{chapter}{Bibliografía}
%\bibliographystyle{miunsrturl}
%
%\appendix
%\input{apendices/manual_usuario/manual_usuario}
%%\input{apendices/paper/paper}
%\input{glosario/entradas_glosario}
% \addcontentsline{toc}{chapter}{Glosario}
% \printglossary
\chapter*{}
\thispagestyle{empty}

\end{document}
