\documentclass[a4paper,11pt]{book}
%\documentclass[a4paper,twoside,11pt,titlepage]{book}
\usepackage{subcaption}
% \usepackage[final]{listings}     % COMENTAR ESTE CUANDO SE QUITE DRAFT
% \usepackage{listings}

\usepackage[labelfont=bf]{caption}

\usepackage{minted}
\usepackage[utf8]{inputenc}
\usepackage[spanish]{babel}
\usepackage{float}
\usepackage{tabularx}
\renewcommand\tabularxcolumn[1]{m{#1}}
\usepackage{amsmath}
\usepackage[export]{adjustbox}


\usepackage{mwe}     % para poner imagenes dummy

% Parte de las tablas
\usepackage{tabularray}

\newcommand{\tabitem}{~~\llap{\textbullet}~~}

% GANTT
\usepackage[spanish]{translator}

\usepackage{adjustbox}
\usepackage{pgfgantt}
\def\pgfcalendarweekdayletter#1{%
\ifcase#1L\or M\or X\or J\or V\or S\or D\fi%
}

% FIN GANTT

\usepackage{xcolor}      % PARA PONER COLOR AL TEXTO
\definecolor{Silver}{rgb}{0.752,0.752,0.752}
\definecolor{Alto}{rgb}{0.874,0.874,0.874}
\definecolor{LightGray}{gray}{0.9}

\usepackage{xspace}

\setcounter{secnumdepth}{3}    % si se quiere poner subsubsections en el indice
\setcounter{tocdepth}{3}       % si se quiere poner subsubsections en el indice

% \usepackage[style=list, number=none]{glossary} %
%\usepackage{titlesec}
%\usepackage{pailatino}

\decimalpoint
\usepackage{dcolumn}
\newcolumntype{.}{D{.}{\esperiod}{-1}}
\makeatletter
\addto\shorthandsspanish{\let\esperiod\es@period@code}
\makeatother


%\usepackage[chapter]{algorithm}
\RequirePackage{verbatim}
%\RequirePackage[Glenn]{fncychap}
\usepackage{fancyhdr}
\usepackage{graphicx}
\usepackage{afterpage}

\usepackage{longtable}

\usepackage[pdfborder={000}]{hyperref} %referencia
\usepackage{xurl}

% ********************************************************************
% Re-usable information
% ********************************************************************
\newcommand{\myTitle}{Simulador de carreras de coches en Unreal Engine\xspace}
\newcommand{\myDegree}{Grado en Ingeniería Informática\xspace}
\newcommand{\myName}{Andrés Merlo Trujillo\xspace}
\newcommand{\myProf}{Luis López Escudero\xspace}
%\newcommand{\myOtherProf}{Nombre Apllido1 Apellido2 (tutor2)\xspace}
%\newcommand{\mySupervisor}{Put name here\xspace}  NO USADO
\newcommand{\myFaculty}{Escuela Técnica Superior de Ingenierías Informática y de
Telecomunicación\xspace}
\newcommand{\myFacultyShort}{E.T.S. de Ingenierías Informática y de
Telecomunicación\xspace}
\newcommand{\myDepartment}{Departamento de Lenguaje y Sistemas Informáticos\xspace}
\newcommand{\myUni}{\protect{Universidad de Granada}\xspace}
\newcommand{\myLocation}{Granada\xspace}
\newcommand{\myTime}{\today\xspace}
\newcommand{\myVersion}{Version 0.1\xspace}
\newcommand{\sprintNro}{6\xspace}
\newcommand{\docSprints}{3\xspace}
\newcommand{\totalSprints}{9\xspace}
\newcommand{\sprintLength}{2 semanas\xspace}
\newcommand{\actualSprintLength}{10 días\xspace}
\newcommand{\projectph}{46,5 PH\xspace}
% \newcommand{\finalAlg}{\textbf{PONER AQUÍ EL ALGORITMO DE NAVEGACIÓN}\xspace}
\newcommand{\finalAlg}{A*\xspace}
\newcommand{\planApp}{Trello\cite{trello}\xspace}
\newcommand{\cubeSize}{1 metro\xspace}
\newcommand{\gridSize}{250x250\xspace}


\hypersetup{
pdfauthor = {\myName (andresmerlo@correo.ugr.es)},
pdftitle = {\myTitle},
pdfsubject = {},
pdfkeywords = {Unreal, Simulación, IA },
pdfcreator = {LaTeX con el paquete ....},
pdfproducer = {pdflatex}
}

%\hyphenation{}


%\usepackage{doxygen/doxygen}
%\usepackage{pdfpages}
\usepackage{url}
\usepackage{colortbl,longtable}
\usepackage[stable]{footmisc}
%\usepackage{index}

%\makeindex
%\usepackage[style=long, cols=2,border=plain,toc=true,number=none]{glossary}
% \makeglossary


\pagestyle{fancy}
\fancyhf{}
\fancyhead[LO]{\leftmark}
\fancyhead[RE]{\rightmark}
\fancyhead[RO,LE]{\textbf{\thepage}}
\renewcommand{\chaptermark}[1]{\markboth{\textbf{#1}}{}}
\renewcommand{\sectionmark}[1]{\markright{\textbf{\thesection. #1}}}

\setlength{\headheight}{1.5\headheight}

\newcommand{\HRule}{\rule{\linewidth}{0.5mm}}
%Definimos los tipos teorema, ejemplo y definición podremos usar estos tipos
%simplemente poniendo \begin{teorema} \end{teorema} ...
\newtheorem{teorema}{Teorema}[chapter]
\newtheorem{ejemplo}{Ejemplo}[chapter]
\newtheorem{definicion}{Definición}[chapter]

\definecolor{gray97}{gray}{.97}
\definecolor{gray75}{gray}{.75}
\definecolor{gray45}{gray}{.45}
\definecolor{gray30}{gray}{.94}

\newcommand{\bigrule}{\titlerule[0.5mm]}


%Para conseguir que en las páginas en blanco no ponga cabecerass
\makeatletter
\def\clearpage{%
  \ifvmode
    \ifnum \@dbltopnum =\m@ne
      \ifdim \pagetotal <\topskip
        \hbox{}
      \fi
    \fi
  \fi
  \newpage
  \thispagestyle{empty}
  \write\m@ne{}
  \vbox{}
  \penalty -\@Mi
}
\makeatother

\usepackage{pdfpages}
\begin{document}
\input{portada/portada}
\chapter*{}
%\thispagestyle{empty}
%\cleardoublepage

%\thispagestyle{empty}

\input{portada/portada_2}



\cleardoublepage
\thispagestyle{empty}

\begin{center}
{\large\bfseries \myTitle}\\
\end{center}
\begin{center}
Andrés Merlo Trujillo\\
\end{center}

%\vspace{0.7cm}
\noindent{\textbf{Palabras clave}: Unreal, Simulación, IA }\\

\vspace{0.7cm}
\noindent{\textbf{Resumen}}\\

Se pretende desarrollar una aplicación gráfica utilizando el motor gráfico \textit{Unreal Engine}, cuyo objetivo es de poder simular carreras de coches. Esta simulación dotará a los pilotos virtuales de la capacidad de tomar decisiones realistas y de cometer errores, basándose en las condiciones de cada piloto durante la carrera, que pueden variar dependiendo de las condiciones externas o ser modificadas por el usuario en tiempo real. Además, se podrán configurar ajustes adicionales antes de comenzar la carrera, con el objetivo de hacer la simulación lo más personalizable posible.

\cleardoublepage


\thispagestyle{empty}


\begin{center}
{\large\bfseries Car racing simulator in Unreal Engine}\\
\end{center}
\begin{center}
Andrés Merlo Trujillo\\
\end{center}

%\vspace{0.7cm}
\noindent{\textbf{Keywords}: Unreal, Simulation, AI }\\

\vspace{0.7cm}
\noindent{\textbf{Abstract}}\\

The aim is to develop a graphics application using the graphics engine \textit{Unreal Engine}, whose objective is to be able to simulate car races. This simulation will provide virtual pilots with the ability to make realistic decisions and make mistakes, based on the conditions of each pilot during the race, which can vary depending on external conditions or be modified by the user in real time. Furthermore, additional settings can be configured before starting the race, with the aim of making the simulation as customizable as possible.

\chapter*{}
\thispagestyle{empty}

\noindent\rule[-1ex]{\textwidth}{2pt}\\[4.5ex]

Yo, \textbf{Andrés Merlo Trujillo}, alumno de la titulación Ingeniería Informática de la \textbf{Escuela Técnica Superior
de Ingenierías Informática y de Telecomunicación de la Universidad de Granada}, con DNI 77147239H, autorizo la
ubicación de la siguiente copia de mi Trabajo Fin de Grado en la biblioteca del centro para que pueda ser
consultada por las personas que lo deseen.

\vspace{6cm}

\noindent Fdo: Andrés Merlo Trujillo

\vspace{2cm}

\begin{flushright}
Granada a 11 de julio de 2023.
\end{flushright}


\chapter*{}
\thispagestyle{empty}

\noindent\rule[-1ex]{\textwidth}{2pt}\\[4.5ex]

D. \textbf{Luis López Escudero}, Profesor del Área del Departamento de Lenguajes y Sistemas Informáticos de la Universidad de Granada.

\vspace{0.5cm}

D. \textbf{Germán Arroyo Moreno}, Profesor del Área del Departamento de Lenguajes y Sistemas Informáticos de la Universidad de Granada.


\vspace{0.5cm}

\textbf{Informan:}

\vspace{0.5cm}

Que el presente trabajo, titulado \textit{\textbf{\myTitle}},
ha sido realizado bajo su supervisión por \textbf{Andrés Merlo Trujillo}, y autorizamos la defensa de dicho trabajo ante el tribunal
que corresponda.

\vspace{0.5cm}

Y para que conste, expiden y firman el presente informe en Granada a 11 de julio de 2023.

\vspace{1cm}

\textbf{Los directores:}

\vspace{5cm}

\noindent \textbf{Luis López Escudero \ \ \ \ \ Germán Arroyo Moreno}

\chapter*{Agradecimientos}
\thispagestyle{empty}

       \vspace{1cm}


A mi familia y a Luis por brindarme su apoyo durante todo el desarrollo del proyecto.



%\frontmatter
\tableofcontents
\newpage
%\listoffigures
%\listoftables
%
%\mainmatter
% \setlength{\parskip}{5pt}

% ################################ MIS CAPITULOS ################################

\chapter{Introducción}
\section{Introducción y Motivación}
%PALABRAS REPETIDAS, REVISAR
%gran cantidad
%no se repite, pero "cosas" queda muy feo

El mundo del motor es un deporte con una amplia base de seguidores y en el que se invierte una gran cantidad de dinero en simuladores de todo tipo para poder tener una ventaja competitiva. Estos simuladores permiten a los equipos de carreras realizar una extensa variedad de actividades: entrenar a los pilotos antes y durante la temporada, modificar parámetros para lograr el máximo rendimiento del vehículo o simular carreras para analizar cuál será la estrategia más adecuada para alcanzar la victoria.

\bigskip

El proyecto se centrará en la simulación de gestión de carreras, sector donde existen pocas opciones para el público general y que es cada vez más demandado para aprender la manera en que las distintas estrategias pueden afectar al desenlace de una carrera y como diferentes situaciones influyen en la capacidad de los pilotos. Estos factores pueden afectar a la tasa de errores producidos durante la carrera.

% \bigskip

% Se añadirán ciertas características procedentes de videojuegos con elementos de simulación a la aplicación gráfica, como la posibilidad de elegir entre varios tipos de vehículos y algunas de las aptitudes de los pilotos, ambas procedentes de \textit{Gran Turismo B-Spec}

\bigskip


La motivación detrás de realizar este programa es profundizar en los conocimientos previamente adquiridos en otras asignaturas del manejo de \textit{Unreal Engine} y en el aprendizaje de su API, enfocándolo al ámbito de las carreras y de los vehículos autónomos, área que me resulta muy interesante. Asimismo, se busca profundizar en el estudio de los distintos algoritmos de navegación para los vehículos del simulador, intentando escoger el que mejor se adapte a las necesidades del proyecto y que ofrezca buenos resultados.

\newpage

\section{Objetivos}

En este proyecto se pretende desarrollar un simulador de gestión de carreras multidisciplinar y personalizable, cuyos pilotos tendrán un conjunto de capacidades que serán modificadas por las condiciones de la carrera o por el usuario en tiempo real, haciendo que el piloto pueda cometer más errores o menos si su estado mejora. También existirá la opción de poder modificar la velocidad de simulación, con el objetivo de obtener los resultados de manera más rápida.

\bigskip

Además, la aplicación permitirá modificar diversos parámetros antes de la carrera, con el propósito de personalizar la simulación lo máximo posible. Cabe destacar que estos parámetros no podrán ser cambiados de nuevo, debido a que no tendría demasiado sentido y podrían hacer que los resultados finales no fueran del todo precisos.


\bigskip


Siendo más específicos, los objetivos mínimos relativos a alteraciones antes de ejecutar la simulación son:

\begin{itemize}
   \item Cambio de número de pilotos.
   \item Cambio de número de vueltas.
   \item Cambios de las aptitudes de los pilotos; es decir, el nivel máximo de cada condición que puede alcanzar.
   \item Cambio de la velocidad máxima de los vehículos.
   \item Permitir (opcionalmente) que haya variación de prestaciones entre vehículos.
   \item Cambio del tipo de vehículo.
\end{itemize}

\bigskip

En cuanto a cambios durante la carrera, debe cumplir estos objetivos:

\begin{itemize}
   \item Seleccionar piloto de la lista de pilotos.
   \item Obtener las condiciones actuales del piloto seleccionado.
   \item Modificar las condiciones actuales del piloto seleccionado.
   \item Cambiar la velocidad de simulación.
\end{itemize}

Conviene resaltar que todos estos objetivos y algunos opcionales serán explicados en detalle en apartados siguientes.

\section{Estado del arte}
Como ya se ha descrito anteriormente, este proyecto hará uso de \textit{Unreal Engine} como motor gráfico y \textit{framework} para la realización del simulador especificado. En cuanto a la forma en la que los pilotos se moverán por el circuito, he decidido usar el algoritmo de navegación \finalAlg.

\bigskip

A continuación, voy a explicar las distintas opciones que hay disponibles en motores gráficos y en algoritmos de navegación y explicar el motivo de la elección realizada. Separaré esto en dos subsecciones.

\subsection{Motor gráfico}
En el mercado hay gran variedad de motores gráficos de videojuegos, los más conocidos son los siguientes:

\subsubsection{Unreal Engine}

\begin{figure}[H]
   \centering
   \includegraphics[width=0.3\textwidth]{imagenes/UE_LOGO.png}
   \caption{Logotipo de \textit{Unreal Engine}\cite{unreal-logo}.}
   % \vspace{10pt}
   % \footnotesize{Fuente: \url{https://es.wikipedia.org/wiki/Unreal_Engine}}
\end{figure}

Unreal Engine \cite{unreal} es una plataforma de desarrollo que permite, entre otras cosas: el desarrollo de videojuegos, creación de contenido para programas y películas, simulación, etc. Posee un soporte para un gran número de plataformas, tanto consolas como móviles.

\bigskip

Unreal incluye \textit{Blueprint Visual Scripting}, el cual es un sistema de \textit{scripting} visual que hace uso de nodos para programar las distintas partes del sistema. Permite programar de manera visual sin la necesidad de conocer la \textit{API} de C++.

\bigskip

Además, a partir de la versión 5.0, Unreal incluye diversas tecnologías como: 

\begin{itemize}
   \item TSR \textit{(Temporal Super Resolution)}, que mejora el rendimiento haciendo que el juego se renderice internamente a una resolución menor y luego sea reescalado al tamaño deseado.
   \item Nanite, que permite tener objetos con una gran cantidad de polígonos en la pantalla, 
   \item Lumen, que genera una iluminación más realista.
\end{itemize}

Estas tecnologías permiten obtener un resultado gráfico muy convincente, partiendo de un modelo creado por profesionales. Asimismo, la inclusión de \textit{TSR} posibilita escalar la aplicación a resoluciones mayores de lo que el hardware es capaz de ejecutar.

\bigskip

Unreal también incluye una tienda con abundantes recursos, permitiendo simplificar mucho el desarrollo de videojuegos.

\bigskip

Sin embargo, no tiene una licencia de código abierto, pero su código fuente es accesible a través de un repositorio de GitHub.

\subsubsection{Unity}

\begin{figure}[H]
   \centering
   \includegraphics[width=0.4\textwidth]{imagenes/UNITY_LOGO.png}
   \caption{Logotipo de \textit{Unity}\cite{unity}.}
   % \vspace{10pt}
   % \footnotesize{Fuente: \url{https://es.wikipedia.org/wiki/Unity_(motor_de_videojuego)}}
\end{figure}

Unity \cite{unity} es una plataforma de desarrollo que permite: desarrollar videojuegos, visualizar construcciones en el ámbito de la arquitectura, para uso en la industria automotriz y para la creación de películas y series.

\bigskip

Entre sus características se encuentra:

\begin{itemize}
   \item Uso de C\# como lenguaje de programación.
   \item Motor de físicas 2D separado del de físicas en 3D.
   \item Posee un \textit{Scriptable Rendering Pipeline} altamente personalizable, permitiendo mediante el uso de scripts escritos en C\# personalizar sombras, iluminación, oclusión ambiental, entre otras.
   \item Soporte para un gran número de plataformas, incluyendo consolas y móviles, aparte de Windows, macOS y Linux.
   \item Incluye un lenguaje de programación visual, similar a \textit{Unreal} que permite programar sin necesitar conocer la \textit{API}.
\end{itemize}
   
Además, Unity tiene varias versiones del programa, incluyendo uno gratuito y los demás de pago, con prioridad de asistencia y un mayor abanico de herramientas.

\subsubsection{Godot}

\begin{figure}[H]
   \centering
   \includegraphics[width=0.4\textwidth]{imagenes/GODOT_LOGO.png}
   \caption{Logotipo de \textit{Godot}\cite{godot-logo}.}
   % \vspace{10pt}
   % \footnotesize{Fuente: \url{https://es.wikipedia.org/wiki/Godot}}
\end{figure}

Godot \cite{godot} es un motor gráfico gratuito y multiplataforma de código abierto que permite crear videojuegos en 2D y 3D. 

\bigskip

Entre las características más notables se encuentran las siguientes:

\begin{itemize}
   \item Motor gráfico 2D dedicado. 
   \item Programación usando GDScript, C\# o C++ de manera oficial. No obstante, se puede programar usando otro lenguaje como Rust, Python o JavaScript. 
   \item Soporte para un gran número de plataformas, incluyendo móviles y consolas.
   \item Incluye un motor de físicas (solo para el motor gráfico 3D), aunque es algo más simple que el de sus competidores.
\end{itemize}

\bigskip 

A partir de la versión 4, Godot ha dejado de incluir el lenguaje visual, debido a su falta de uso, por lo que solo es posible utilizar lenguajes de programación convencionales \cite{godot-no-visual}.

\bigskip

Además, Godot tiene una biblioteca con una gran cantidad de recursos para utilizar durante el desarrollo de un videojuego.


\subsubsection{Elección final}

Al final, he decidido utilizar Unreal Engine debido a que la versión gratuita incluye toda la funcionalidad, es el más potente en cuanto a programación visual y es con el que más familiarizado estoy de todas las opciones citadas anteriormente.

\subsection{Algoritmo de navegación}
En cuanto a algoritmos de navegación para los vehículos, los más destacados son:

\begin{itemize}
   \item \textbf{Aprendizaje por refuerzo:} Es una técnica de aprendizaje automático en la que el agente aprende a tomar decisiones en un entorno mediante la interacción a través de acciones
   y recibiendo una recompensa en caso de acercarse al objetivo deseado o de hacerlo bien. 
   
   Es importante obtener un equilibrio entre exploración y explotación, de forma que sea capaz de actuar ante situaciones desconocidas correctamente. En caso de estar desequilibrado, puede darse la situación en el que se produzca un máximo local y no sea capaz de descubrir la mejor estrategia.

   
   Para evitar esto, se suelen utilizar métodos de equilibrio. Uno de ellos es el denominado \textit{epsilon-greedy}, que permite controlar la cantidad de exploración mediante un parámetro \textit{epsilon}, que determina la aleatoriedad en la selección de acciones \cite{10.1007/978-3-642-24455-1_33}, permitiendo elegir en función de dicho parámetro una acción aleatoria o la mejor hasta el momento.

   Una desventaja que tiene es la necesidad de entrenar el modelo, lo cual puede ser un proceso largo y complicado, pero con el suficiente tiempo y con un conjunto de recompensas y castigos bien realizado, es una excelente técnica para resolver este tipo de problema.

   El algoritmo \textit{Q-Learning} es uno de los más utilizados para espacios discretos. Consiste en una función Q, la cual posibilita al agente estimar el valor esperado de una acción dado un estado. Dicha función se implementa mediante una tabla denominada \textit{Q-Values}, la cual almacena el valor esperado de las recompensas para cada estado y acción posible. Para evitar problemas de equilibrio, se utilizan métodos como el \textit{epsilon-greedy}, ampliamente utilizado y que ofrece buenos resultados.

   No obstante, si el entorno de aprendizaje es continuo, una tabla de \textit{Q-Values} no es adecuada debido a la gran cantidad de posibles valores y el alto costo computacional en procesamiento y memoria. En su lugar, se puede sustituir la tabla por una red neuronal, como el algoritmo \textit{Deep Q-Network} (DQN), para trabajar en este tipo de espacios \cite{coulom:tel-00003985}.

   Para este proyecto, se podría implementar DQN. La red neuronal tendría como entrada varios sensores de colisión y de límites de pista, y como salida el nivel de aceleración, frenado y giro del volante.

   \item \textbf{Mediante reglas:} Conjunto de acciones que se aplican cuando se cumple una determinada condición y suelen estar ordenadas por prioridad. Pueden ser representadas mediante un árbol, donde cada acción se encuentra en una hoja y las preguntas o condiciones que las activan son las ramas (aristas). La prioridad de las acciones se establece según su posición en el árbol, siendo las más importantes las de niveles superiores.
   
   Esto tiene la ventaja de ser más sencillo de implementar que el de aprendizaje por refuerzo, pero suele tener peores resultados, al poder darse el caso de no haber priorizado bien las reglas o no haber tenido en cuenta algún caso especial.


   Aplicado a este proyecto, se podría diseñar como un conjunto de funciones que manejen cada posible control del vehículo y se ejecuten de manera periódica. De esta forma, dependiendo del entorno y del estado del piloto, se aplicarán distintas salidas para guiar el vehículo por el circuito.

   \item \textbf{Máquina de estados:} Este algoritmo implementa una máquina de estados finitos mediante una estructura de datos, donde cada nodo representa la estrategia seguida y cada arco representa la transición a otro estado. 
   
   Una de las ventajas que tiene es que suele ser más simple describir el comportamiento en agentes más complejos y además, suelen ser más fáciles de modificar que los basados en reglas.

   En este proyecto se podrían implementar estados como: ``Seguir línea'', ``Adelantar'', ``Comienzo de carrera'' y ``Esquivar obstáculo''. Dependiendo del entorno y del estado del piloto, el agente puede pasar a otro estado más adecuado.

   \item \textbf{Algoritmo A* (A Estrella)}: Algoritmo de búsqueda de caminos en grafos que se utiliza para encontrar la ruta óptima. Utiliza una función heurística para guiar la búsqueda hacia la solución más eficiente y es más rápido en tiempo de ejecución que otros algoritmos similares, como el de Dijkstra.
   
   Si se desea aplicar este algoritmo al proyecto, es necesario discretizar el espacio, lo que puede provocar una pérdida de precisión. Además, puede resultar más costoso que otras alternativas anteriormente mencionadas.
   % Además, se debería aplicar periódicamente en todos los vehículos de la simulación, lo que puede ser computacionalmente costoso. %arreglar esta argumentacion

   No obstante, una ventaja significativa del algoritmo es que se puede incluir el estado del piloto en la función heurística, lo que facilita su integración en el simulador. También se puede aplicar una función de suavizado a la trayectoria resultante para mejorar su precisión. Dado que el proyecto no se centra en la eficiencia computacional, A* podría ser considerado como una opción viable para la navegación por el circuito.
\end{itemize}

% DUDAS
% LAS REFERENCIAS DEBERIAN IR EN LAS REFERENCIAS DEL DOCUMENTO EN GENERAL
% LA CALIDAD DEL DOCUMENTO ES PESIMA
% LOS REQUISITOS DE INFORMACION SON NO FUNCIONALES
% HE REPETIDO DEMASIADAS VECES LO QUE HACE MI SIMULADOR
% EN LA MAYORIA DE LAS SECCIONES HE RELLENADO MAS BIEN POCO
% ME PARECEN POCOS REQUISITOS EN GENERAL, PERO NO SE ME OCURREN MAS

% DUDAS TUTORIA
%no se si poner aqui lo de la pagina web. tipo, que no hay suficientes opciones software para el publico general.


% TODO
% PONER QUE NOS REFERIMOS A SOFTWARE, SISTEMA Y ESO A LO MISMO
\chapter{Especificación de requisitos}

\section{Introducción}

\subsection{Propósito}
%mejorar
El objetivo de esta especificación es definir de manera clara, concisa y precisa las funciones y restricciones que tendrá la aplicación que se desea desarrollar. Al ser un trabajo individual, irá dirigido principalmente a mí, que haré de equipo de desarrollo.

\subsection{Alcance}
%rescribir subsection entera
El proyecto será conocido de ahora en adelante como ``\myTitle''. Por lo tanto, al hacer referencia a ``proyecto'', ``software'', ``simulador'', ``aplicación'' y ``aplicación gráfica'', me referiré a lo mismo.

\bigskip

% repes: carrera
Este sistema se encargará de simular carreras de coches, donde los pilotos tendrán distintas características que variarán dependiendo de las condiciones de la carrera e influirán en su desempeño. El usuario podrá modificar el estado de cada piloto en tiempo real, observando como estos afectan al comportamiento del vehículo.

\bigskip

Además, el software permitirá modificar una serie de ajustes antes de simular la carrera, de acuerdo a las preferencias del usuario. Algunos parámetros son: el número de vehículos, las aptitudes de cada piloto, el tipo de coche y la velocidad máxima. De esta manera, se podrán crear carreras personalizadas de acuerdo a las necesidades del usuario.

\subsection{Definiciones, siglas y abreviaturas}

\begin{itemize}
    \item \textbf{Definiciones: }
        \begin{itemize}
            \item \textbf{Usuario: }Persona que utiliza el sistema con el objetivo de obtener los resultados de la carrera personalizada.
            \item \textbf{Simulación: }Se refiere a la ejecución de una carrera en la aplicación.
            \item \textbf{Piloto: }Conductor de un vehículo en la simulación. Será 
            \item \textbf{Aguante mental y físico: }Condiciones que tiene un piloto durante la carrera y que son modificadas durante la simulación, pudiendo afectar a su conducción.
            \item \textbf{Agresividad: }Factor que afectará al piloto durante la carrera, haciendo que si sube cometa más errores.
            \item \textbf{Experiencia: }Parámetro del piloto que no varía durante la carrera y que indica la habilidad que tiene para tomar decisiones.
        \end{itemize}
    \item \textbf{Siglas: }
    \begin{itemize}
        \item \textbf{FPS: }\textit{Frames Per Second}. Son los fotogramas (imágenes) que muestra el sistema cada segundo. Si son lo suficientemente altos, se genera la ilusión de movimiento.
    \end{itemize}
\end{itemize}

% \subsection{Referencias}

% Los siguientes documentos y enlaces se han consultado para crear este capítulo:

% \begin{itemize}
%     \item Ingeniería del Software. Ejercicio en clase, unidad 3: Requerimientos del Software. LSI (UGR). \url{https://lsi2.ugr.es/~mvega/docis/aluwork/colectivo/Ejercicio%20en%20clase%20version%202}
%     \item fusm calidad del software. \url{https://sites.google.com/site/fusmcalidaddelsoftware/proyecto/g-informe-de-especificacion-de-requerimientos/3-requisitos-especificos/3-5-atributos-del-sistema}
% \end{itemize}

\subsection{Visión general}
% rescribir
Este capítulo constará de tres secciones: %Introducción, descripción global, y requisitos específicos. 

\bigskip

En esta primera sección se muestra la introducción y la visión general de la especificación de requisitos.

\bigskip

En la sección 2 se proporcionará una descripción general del sistema a construir, con el fin de conocer las funciones principales, datos requeridos, restricciones y otros aspectos relevantes. 

\bigskip

En la sección 3 se definen detalladamente los requisitos que debe cumplir el sistema en el momento de su desarrollo.

\section{Descripción general}
\subsection{Perspectiva del producto}

\begin{figure}[H]
    \centering
    \includegraphics[width=0.8\textwidth]{imagenes/grafo-sistema.drawio.png}
    \caption{Grafo de los distintos componentes del sistema.}
 \end{figure}

El software se diseñará como una aplicación gráfica que permitirá a los usuarios simular carreras de coches, permitiendo configurar parámetros antes de la simulación y durante la carrera. El sistema recopilará información sobre la situación de carrera de cada piloto, y utilizará estos datos para simular el desempeño de cada uno en tiempo real.

\bigskip

Los usuarios interactuarán con el sistema a través de una interfaz gráfica con la que podrán modificar todos los ajustes que estimen convenientes.

\bigskip

Además, será un sistema independiente, al no necesitar interactuar con otros sistemas externos.

\subsection{Funciones del producto}

Las funciones principales de la aplicación son las siguientes:

\begin{itemize}
    \item 
    %Configuración de los parámetros antes de la carrera: 
    % Otorgará la posibilidad de 
    Configurar las distintas opciones antes de la carrera como: número de pilotos, aptitudes de cada uno, tipo de vehículo, velocidad máxima y número de vueltas. 
    
    \item Calcular el estado del piloto según las condiciones actuales de la carrera, de manera que si su estado empeora, cometerá más errores y si mejora, cometerá menos errores, todo en tiempo real.
    
    \item Modificar el estado de los pilotos durante la carrera, con independencia de las condiciones de la misma, pudiendo ver en tiempo real como afecta el cambio producido.
\end{itemize}

Y tendrá otras funciones como:

\begin{itemize}
    \item Pausar la simulación.
    \item Acelerar o reducir la velocidad de simulación.
    \item Cambiar la hora del día.
\end{itemize}

\subsection{Características de los usuarios}

Es recomendable que los usuarios tengan conocimientos básicos de informática, pero no es necesario que sepan como funcionan las carreras, ya que el proceso se encuentra automatizado. 

\subsection{Restricciones}

Las restricciones que tendrá la aplicación son las siguientes:

\begin{itemize}
    \item \textbf{RES1.-} Para el apartado gráfico y la interfaz, se utilizará el motor gráfico \textit{Unreal Engine}.
    \item \textbf{RES2.-} Una vez pausada la simulación, no se podrá modificar su velocidad.
    \item \textbf{RES3.-} El número de pilotos en la carrera no puede ser menor o igual a 1.
    \item \textbf{RES4.-} Las aptitudes de cada piloto antes de la carrera tienen que ser mayores que 0.
    % \item \textbf{RES5.-} La velocidad máxima no podrá ser menor de 150 km/h.
\end{itemize}

\subsection{Suposiciones y dependencias}

% Los requisitos descritos pueden estar sujetos a cambios en función de la evolución del proyecto y la posible adición de nuevas funcionalidades. 

% \bigskip

Este sistema funciona de forma independiente, por lo que no es necesario comunicarse con otros sistemas externos o instalar ningún programa adicional, excepto el propio simulador.

\bigskip

También se asume que el sistema operativo instalado es Windows en sus versiones 10 u 11.

\subsection{Requisitos específicos}

\subsubsection{Requisitos funcionales}

\begin{itemize}
    \item \textbf{RF1.- Modificación de parámetros antes de la carrera:} El sistema debe permitir modificar los parámetros antes de ejecutar la simulación.
    \item \textbf{RF2.- Visualización del estado de los pilotos:} El usuario podrá ver el estado de cada piloto durante la carrera, para poder realizar operaciones sobre el mismo, si lo ve necesario.
    \item \textbf{RF3.- Modificación del estado de los pilotos durante la carrera:} El sistema permitirá modificar el estado de los pilotos en tiempo real.
    \item \textbf{RF4.- Actualización de la velocidad de simulación:} El sistema permitirá actualizar la velocidad entre varias opciones predefinidas.
    \item \textbf{RF5.- Pausado y reanudado de la simulación:} El sistema permitirá pausar y reanudar la simulación que está en curso.
    \item \textbf{RF6.- Características del simulador:} El sistema debe simular la conducción de varios pilotos en una carrera de coches, como adelantamientos y sorteo de obstáculos, entre otros. Los pilotos tendrán un conjunto de condiciones que variarán durante la carrera y que afectarán a su rendimiento.
    \item \textbf{RF7.- Exportación de la configuración de la simulación:} El sistema permitirá almacenar la configuración de la carrera en un fichero para su posterior importación.
    \item \textbf{RF8.- Importación de la configuración de la simulación:} El sistema permitirá importar la configuración de la simulación.
    \item \textbf{RF9.- Visualización de la vuelta actual:} El sistema deberá realizar el cálculo y la visualización de la vuelta actual.
\end{itemize}


\subsubsection{Requisitos de soporte}

\begin{itemize}
    % \item RNF1.- El sistema debe ser ejecutado en un entorno Windows 10 o Windows 11.
    \item \textbf{RNF1.-} El sistema se ejecutará en los sistemas operativos Windows 10 y Windows 11.
\end{itemize}

\subsubsection{Requisitos de usabilidad}

\begin{itemize}
    \item \textbf{RNF2.-} La interfaz deberá ser fácil de usar e intuitiva para los usuarios, de manera que puedan navegar por la misma sin demasiada dificultad.
    \item \textbf{RNF3.-} El tamaño del texto y el estilo de fuente deben ser adecuados para facilitar su lectura.
\end{itemize}

\subsubsection{Requisitos de rendimiento}

\begin{itemize}
    \item \textbf{RNF4.-} El sistema deberá tardar, como máximo, 50 milisegundos (unos 20 FPS) en ejecutar cada paso de la simulación.
    \item \textbf{RNF5.-} El cambio manual del estado de un piloto debe ser reflejado en la simulación en menos de 10 segundos.
\end{itemize}

\subsubsection{Requisitos de información del sistema}

\begin{itemize}
    \item \textbf{RI1.- Datos de los pilotos:} Nombre, nacionalidad, aguante mental y físico, agresividad y experiencia.
    % \begin{itemize}
    %     \item Nombre
    %     \item Nacionalidad
    %     \item Aguante mental
    %     \item Aguante físico
    %     \item Agresividad
    %     \item Experiencia
    % \end{itemize}
    \item \textbf{RI2.- Datos del simulador:} Velocidad de simulación, hora actual, número de vueltas, vuelta actual y número de vehículos.
\end{itemize}

\subsubsection{Aspectos legales}

La simulación no va a guardar ningún tipo de dato privado ni asociable a un individuo, haciendo que todos los datos sean anónimos. Por tanto, no es necesario seguir ningún tipo de tratamiento especial con la información insertada en la aplicación.

\subsubsection{Interfaz de usuario}

La interfaz de usuario que se utilizará para configurar los parámetros previos a la carrera estará compuesta por varios deslizadores para ajustar el número de vueltas, de coches y la hora. Además, habrá otra ventana para modificar algunas características de los pilotos y dos botones para importar y exportar la configuración.

\bigskip

Durante la carrera, la interfaz de usuario tendrá un aspecto similar al de las carreras reales, con un contador de vueltas y la posición de los pilotos a la izquierda. Asimismo, se podrá pulsar sobre un piloto para ver y modificar su estado en tiempo real.

\newpage

Los bocetos de dichas interfaces son las siguientes:
% La estructura que tendrá la interfaz de usuario es la siguiente:

\begin{figure}[H]
    \centering
    \includegraphics[width=\textwidth]{imagenes/pag1.png}
    \caption{Interfaz del configurador antes de la carrera.}
 \end{figure}

 \begin{figure}[H]
    \centering
    \includegraphics[width=\textwidth]{imagenes/pag2.png}
    \caption{Interfaz de las opciones durante la carrera.}
 \end{figure}

% DUDAS
% PUSE EXPLICACION GENERAL, PERO NO SE EXACTAMENTE QUE ES
% esto son datos cambiantes, puedo poner algo inicial, pero seguramente haya que modificarlo al final
% no se ni siquiera el numero de sprints que va a haber!! (tendria que tener una velocidad de equipo)
% En el siguiente capitulo es cuando se hace la division en tareas, no entiendo si hayq ue poner otro diagrama de gantt especifico en el capitulo anterior
\chapter{Planificación}

\section{Introducción}

Se pretende adaptar una metodología ágil para el desarrollo del proyecto, más concretamente el marco de trabajo Scrum para que se ajuste a las condiciones del proyecto, y se complementará con otras herramientas de planificación.

\bigskip

Al ser solo una persona desarrollando el proyecto, es imposible seguir algunas fases del marco de trabajo, como los \textit{Daily Meetings}. No obstante, todos los artefactos se pueden generar.
%  y el proyecto puede ser dividido en Sprints.

\bigskip

Otros aspectos de Scrum que se van a aplicar son: la utilización de sprints para desarrollar el proyecto, su enfoque iterativo e incremental y las Historias de Usuario.

\bigskip

Además, se van a utilizar herramientas que no son estrictamente de metodologías ágiles, como el diagrama de Gantt, para mostrar de una manera más visual la temporización de las distintas tareas a realizar.
% para planificar la cantidad de tiempo que se piensa dedicar a cada sprint o Historia de Usuario, dependiendo del nivel de detalle. 

\bigskip

Por último, cabe recalcar que no se pretende seguir Scrum o una metodología ágil de manera precisa, al no cumplir todos los requisitos para poder usarlo. Solo se usarán algunos aspectos aplicables del marco, junto a otras herramientas que no son exactamente de metodologías ágiles.


\newpage

\section{Product Backlog}

El \textit{Product Backlog} con las Historias de Usuario, ordenadas por prioridad y con sus Puntos de Historia (PH), son:

\begin{table}[H]
    \centering
    \begin{tabularx}{\textwidth}{| >{\centering\arraybackslash}X | >{\centering\arraybackslash}m{5.5cm} | >{\centering\arraybackslash}X | >{\centering\arraybackslash}X |}
        \hline
        \textbf{Código} & \textbf{Descripción} & \textbf{Prioridad} & \textbf{PH} \\
        \hline
        % HU1 & Como usuario, quiero que el sistema simule la conducción de varios pilotos en una carrera de coches, incluyendo adelantamientos y sorteo de obstáculos. & 1 & 5 \\
        HU1 & Como usuario, quiero tener una representación en 3D de los coches y el circuito. & 1 & 3 \\
        \hline
        HU2 & Como usuario, quiero que los pilotos calculen la ruta más óptima en el circuito. & 1 & 8 \\
        \hline
        HU3 & Como usuario, quiero que los pilotos sean capaces de seguir una ruta de manera suave. & 1 & 5 \\
        \hline
        HU4 & Como usuario, quiero que los pilotos sean capaces de adelantarse entre ellos. & 1 & 5 \\
        \hline
        HU5 & Como usuario, quiero que los pilotos sean capaces de recuperarse después de un accidente. & 1 & 3 \\
        \hline
        HU6 & Como usuario, quiero modificar los ajustes de la simulación antes de comenzar la carrera. & 2 & 2 \\
        \hline
        HU7 & Como usuario, quiero modificar el estado de los pilotos durante la carrera. & 2 & 2 \\
        \hline
        HU8 & Como usuario, quiero ver la posición actual y el estado de los pilotos de la carrera en curso. & 3 & 3 \\
        \hline
        HU9 & Como usuario, quiero ver el número de vueltas de la carrera en curso. & 3 & 2 \\ 
        \hline
        HU10 & Como usuario, quiero guardar la configuración de una carrera para su posterior uso. & 4 & 2\\
        \hline
        HU11 & Como usuario, quiero cargar la configuración de una carrera almacenada en un archivo. & 4 & 2 \\
        \hline
        HU12 & Como usuario, quiero pausar y reanudar la simulación en curso. & 5 & 0,5 \\
        \hline
        HU13 & Como usuario, quiero modificar la velocidad de la simulación. & 5 & 0,5 \\
        \hline
    \end{tabularx}
\end{table}


\newpage

\section{Velocidad del equipo}

Esto solo es una estimación de la velocidad que se puede alcanzar para el primer Sprint, en función de los puntos de historia realizados al final del mismo se actualizará la velocidad del equipo.

\bigskip

Algunos aspectos a tener en cuenta son:

\begin{itemize}
    \item Duración de los sprints de \sprintLength, que equivalen a \actualSprintLength, quitando fines de semana.
    \item Se trabajará 6 horas al día en el proyecto.
    \item 1 PH representa un día de trabajo ideal a jornada completa (8 horas). Por tanto, usando las horas reales, 1 PH se completaría en 1,33 días.
    \item Hay un total de \projectph en el proyecto.
\end{itemize}

\bigskip

Entonces, la estimación de la velocidad del equipo para la primera iteración es: $\frac{10 \text{ días/sprint}}{1.33 \text{ días/PH}} = 7.5 \text{ PH/sprint} \simeq \mathbf{8\,PH/sprint}$
% $2 \text{ semanas/sprint} \times 20 \text{ horas/semana} \times 8 \text{ horas/PH} = $

\section{\textcolor{red}{Diseño del Sprint}}

% INCLUIR LOS SPRINTS REFERENTES A LA DOCUMENTACION
% Este proyecto se dividirá en \sprintNro sprints, cuya duración será de \sprintLength. A continuación se desglosará qué se realizará en cada uno de ellos.

La implementación de la aplicación final se dividirá en \sprintNro sprints, cuya duración será de \sprintLength. A esta cifra hay que sumarle los \docSprints sprints anteriores en los que se ha realizado la documentación y prototipos para encontrar el algoritmo que más se ajuste a las necesidades del proyecto. Entonces, hay un total de \totalSprints sprints.

\subsection{Sprint 1}
\begin{itemize}
    \item Explicación: En este sprint se ha creado la versión inicial de la introducción de la documentación, incluyendo las distintas opciones en motores gráficos y algoritmos de navegación.

    \item Temporización: Del 20 de febrero hasta el 8 de marzo.
\end{itemize}

\subsection{Sprint 2}
\begin{itemize}
    \item Explicación: En este sprint se han corregido todos los errores producidos en el capítulo de introducción de la documentación. Además, se incluyeron nuevos algoritmos de navegación y se detallaron algunos otros en la sección de estado del arte.
    \item Temporización: Del 8 de marzo hasta el 22 de marzo. 
\end{itemize}

\subsection{Sprint 3}
\begin{itemize}
    \item Explicación: En este sprint se han arreglado más fallos en el capítulo de introducción y se ha creado la especificación de requisitos del proyecto.
    
    Además, se ha realizado un prototipo del proyecto para probar distintos algoritmos de navegación y se ha programado una versión inicial del algoritmo de seguimiento de líneas de los coches.
    \item Temporización: Del 22 de marzo hasta el 29 de marzo.
\end{itemize}


\subsection{Sprint 4 - Comienzo de la aplicación final}
\begin{itemize}
    \item Explicación: En este sprint se han añadido más detalles a la especificación de requisitos, incluyendo algunos requisitos que faltaban, y se han redactado los capítulos de planificación y análisis.
    
    % Asimismo, se ha comenzado la implementación del proyecto, implementando el algoritmo de navegación inicial y la versión final del controlador del volante utilizando un algoritmo genético para su refinamiento.
    Asimismo, se ha comenzado con la aplicación final, creando la trazada del circuito y el modelado de los coches, con el objetivo de tener una base para los siguientes sprints. También se ha implementado el controlador para que los coches puedan girar de manera suave y correcta por el circuito. Para ello, se ha empleado un algoritmo genético con el objetivo de obtener los mejores valores para el controlador.

    \item Temporización: Del 29 de marzo hasta el 18 de abril.
    \item Historias de Usuario abordadas: HU1, HU3.
\end{itemize}

\subsection{Sprint 5}
\begin{itemize}
    \item Explicación: En este sprint se pretende implementar el algoritmo de navegación base para que los pilotos puedan saber la ruta más eficiente por el circuito.

    % En este sprint se pretende refinar el algoritmo de navegación para que sean capaces de realizar adelantamientos. Además, se implementará la lógica necesaria para que los pilotos que se han salido de pista, puedan recuperarse y volver, calculando una nueva ruta.

    En cuanto a la documentación, se arreglarán los fallos encontrados del sprint anterior y se redactarán los capítulos de diseño e implementación.

    \item Temporización: Del 24 de abril hasta el 7 de mayo.
    \item Historias de Usuario abordadas: HU2.
\end{itemize}

\newpage

\subsection{Sprint 6}

\begin{itemize}
    \item Explicación: En este sprint se pretende incluir la lógica adicional referente a los adelantamientos y recuperaciones en accidentes, para que los pilotos realicen acciones más acordes a la realidad.
    
    % En este sprint se pretende implementar la pantalla de configuración de la simulación, junto a la visualización de la lista de posiciones de los pilotos y el contador de vueltas. Asimismo, se implementará la modificación manual de estado de los pilotos durante la carrera.

    \item Temporización: Del 8 de mayo hasta el 21 de mayo.
    \item Historias de Usuario abordadas: HU4, HU5.
\end{itemize}

\subsection{Sprint 7}

\begin{itemize}
    \item Explicación: En este sprint se pretende implementar el configurador de la simulación, junto al listado de posiciones de los pilotos. Además, se implementará la lógica para poder modificar el estado de los pilotos en tiempo real.
    
    % En este último sprint se pretende implementar las opciones de importación y exportación de la configuración a un fichero externo, con el objetivo de tener una forma rápida y sencilla de almacenar la configuración.
    
    % Además, se implementará el botón de pausa y el botón de velocidad de la simulación.
    
    \item Temporización: Del 22 de mayo hasta el 4 de junio.
    \item Historias de Usuario abordadas: HU6, HU7, HU8.
\end{itemize}

\subsection{Sprint 8}

\begin{itemize}
    \item Explicación: En este último sprint se pretende implementar el contador de vueltas de la carrera, la lógica para la exportación e importación de ficheros de configuración al simulador, el botón de pausa y el botón de velocidad de simulación.
    \item Temporización: Del 4 de junio hasta el 18 de junio.
    \item Historias de Usuario abordadas: HU9, HU10, HU11, HU12, HU13.
\end{itemize}

% \subsection{Sprint 8}

% \begin{itemize}
%     \item Explicación: 
%     \item Temporización: Del 4 de junio hasta el 18 de junio.
%     \item Historias de Usuario abordadas: 
% \end{itemize}

\newpage
\section{Diagrama de Gantt}

El diagrama de Gantt general de todos los Sprints es:


% diagrama de Gantt general
% \begin{center}
\begin{adjustbox}{angle=90}
    \resizebox{0.9\textheight}{!}{
        \begin{ganttchart}[
                hgrid,
                vgrid,
                x unit=5mm,
                time slot format=isodate,
            ]{2023-04-24}{2023-06-04}
            % \gantttitle{Diagrama de Gantt general}{1}
            \gantttitlecalendar{month=name, day, week, weekday=letter} \\
            \ganttbar{Sprint 1}{2023-04-24}{2023-05-07} \\
            \ganttbar{Sprint 1}{2023-04-24}{2023-05-07} \\
            \ganttbar{Sprint 1}{2023-04-24}{2023-05-07} \\
            \ganttbar{Sprint 1}{2023-04-24}{2023-05-07} \\
            \ganttbar{Sprint 1}{2023-04-24}{2023-05-07} \\
            \ganttbar{Sprint 1}{2023-04-24}{2023-05-07} \\
            \ganttbar{Sprint 1}{2023-04-24}{2023-05-07} \\
            \ganttbar{Sprint 1}{2023-04-24}{2023-05-07} \\
            \ganttbar{Sprint 1}{2023-04-24}{2023-05-07} \\
            \ganttbar{Sprint 1}{2023-04-24}{2023-05-07} \\
            \ganttbar{Sprint 1}{2023-04-24}{2023-05-07} \\
            \ganttbar{Sprint 1}{2023-04-24}{2023-05-07} \\
            \ganttbar{Sprint 2}{2023-05-08}{2023-05-21} \\
            \ganttbar{Sprint 3}{2023-05-22}{2023-06-04}
        \end{ganttchart}
    }
\end{adjustbox}
% \end{center}

\newpage

Los diagramas de Gantt de cada Sprint los voy a dividir en distintas subsecciones a continuación:

\subsection{Sprint 1}

\begin{center}
    \resizebox{0.6\textwidth}{!}{
        \begin{ganttchart}[
                hgrid,
                vgrid,
                x unit=6mm,
                time slot format=isodate,
            ]{2023-04-24}{2023-05-07}
            % \gantttitle{Diagrama de Gantt general}{1}
            \gantttitlecalendar{month=name, day, week, weekday=letter} \\
            \ganttbar{HU1}{2023-04-24}{2023-04-28}
            \ganttbar{}{2023-05-01}{2023-05-03} \\
            \ganttbar{HU2}{2023-05-04}{2023-05-05}
        \end{ganttchart}
    }
\end{center}

\subsection{Sprint 2}

\begin{center}
    \resizebox{0.6\textwidth}{!}{
        \begin{ganttchart}[
                hgrid,
                vgrid,
                x unit=6mm,
                time slot format=isodate,
            ]{2023-05-08}{2023-05-21}
            % \gantttitle{Diagrama de Gantt general}{1}
            \gantttitlecalendar{month=name, day, week, weekday=letter} \\
            \ganttbar{HU3}{2023-05-08}{2023-05-11} \\
            \ganttbar{HU4}{2023-05-12}{2023-05-12}
            \ganttbar{}{2023-05-15}{2023-05-17} \\
            \ganttbar{HU5}{2023-05-18}{2023-05-19}
        \end{ganttchart}
    }
\end{center}

\subsection{Sprint 3}

\begin{center}
    \resizebox{0.6\textwidth}{!}{
        \begin{ganttchart}[
                hgrid,
                vgrid,
                x unit=6mm,
                time slot format=isodate,
            ]{2023-05-22}{2023-06-04}
            % \gantttitle{Diagrama de Gantt general}{1}
            \gantttitlecalendar{month=name, day, week, weekday=letter} \\
            \ganttbar{HU6}{2023-05-22}{2023-05-25} \\
            \ganttbar{HU7}{2023-05-29}{2023-06-01} \\
            \ganttbar{HU8}{2023-05-26}{2023-05-26} \\
            \ganttbar{HU9}{2023-06-02}{2023-06-02}
        \end{ganttchart}
    }
\end{center}

% diagrama de gantt

\chapter{Análisis}

En este capítulo se abordará en más detalle cada Historia de Usuario, indicando sus pruebas de validación y las tareas que pueden surgir de cada una de ellas.

\bigskip

Cada Historia de Usuario se desarrollará en una tabla a continuación:

% \section{Historias de Usuario divididas en subtareas}

\begin{table}[H]
    \begin{tabularx}{\textwidth}{| X | X | X |}
        \hline
        \multicolumn{1}{|l|}{\textbf{Identificador:} HU3} & \multicolumn{2}{l|}{Modificación del estado de los pilotos.}
        \\ \hline
        \multicolumn{3}{|>{\hsize=\dimexpr3\hsize+4\tabcolsep+2\arrayrulewidth\relax\linewidth=\hsize}X|}{\textbf{Descripción:} Como usuario, quiero modificar el estado de los pilotos durante la carrera.}
        \\ \hline
        \textbf{Estimación:} 2 & \textbf{Prioridad:} 2 & \textbf{Entrega:} 2
        \\ \hline
        \multicolumn{3}{|>{\hsize=\dimexpr3\hsize+4\tabcolsep+2\arrayrulewidth\relax\linewidth=\hsize}X|}{\textbf{Pruebas de aceptación:}
        \begin{itemize}
            \item No se podrá elegir un valor fuera del rango del deslizador.
            \item Los cambios en el estado deberán verse reflejados en el comportamiento durante la carrera.
        \end{itemize}
        }
        \\ \hline
        \multicolumn{3}{|>{\hsize=\dimexpr3\hsize+4\tabcolsep+2\arrayrulewidth\relax\linewidth=\hsize}X|}{\textbf{Requisitos relacionados:} RF3, RNF5.}
        \\ \hline
        \multicolumn{3}{|>{\hsize=\dimexpr3\hsize+4\tabcolsep+2\arrayrulewidth\relax\linewidth=\hsize}X|}{\textbf{Dependencias:} HU4.}        
        \\ \hline
        \multicolumn{3}{|>{\hsize=\dimexpr3\hsize+4\tabcolsep+2\arrayrulewidth\relax\linewidth=\hsize}X|}{\textbf{Tareas:}
        \begin{itemize}
            \item Implementar los deslizadores referentes al aguante físico y mental y la agresividad.
        \end{itemize}
        }
        \\ \hline
        \multicolumn{3}{|>{\hsize=\dimexpr3\hsize+4\tabcolsep+2\arrayrulewidth\relax\linewidth=\hsize}X|}{\textbf{Observaciones:}}
        \\ \hline
    \end{tabularx}
\end{table}

\newpage


\begin{table}[H]
    \begin{tabularx}{\textwidth}{| X | X | X |}
        \hline
        \multicolumn{1}{|l|}{\textbf{Identificador:} HU1} & \multicolumn{2}{l|}{Algoritmo de navegación.}
        \\ \hline
        \multicolumn{3}{|>{\hsize=\dimexpr3\hsize+4\tabcolsep+2\arrayrulewidth\relax\linewidth=\hsize}X|}{\textbf{Descripción:} Como usuario, quiero que el sistema simule la conducción de varios pilotos en una carrera de coches, simulando adelantamientos y sorteo de obstáculos.}
        \\ \hline
        \textbf{Estimación:} 5 & \textbf{Prioridad:} 1 & \textbf{Entrega:} 1
        \\ \hline
        \multicolumn{3}{|>{\hsize=\dimexpr3\hsize+4\tabcolsep+2\arrayrulewidth\relax\linewidth=\hsize}X|}{\textbf{Pruebas de aceptación:}
        \begin{itemize}
            \item Se debe poder diferenciar la conducción entre pilotos con estados distintos.
            \item Los pilotos deben ser capaces de esquivar obstáculos en la pista.
            \item La simulación debe poder ejecutarse en un tiempo razonable, de manera que se pueda seguir la carrera en tiempo real.
            \item Los pilotos deben ser capaces de adelantar a otros que sean más lentos sin provocar accidentes.
        \end{itemize}
        }
        \\ \hline
        \multicolumn{3}{|>{\hsize=\dimexpr3\hsize+4\tabcolsep+2\arrayrulewidth\relax\linewidth=\hsize}X|}{\textbf{Requisitos relacionados:} RF6, RNF4.}
        \\ \hline
        \multicolumn{3}{|>{\hsize=\dimexpr3\hsize+4\tabcolsep+2\arrayrulewidth\relax\linewidth=\hsize}X|}{\textbf{Tareas:}
        \begin{itemize}
            \item Implementar la malla de navegación.
            \item Implementar el algoritmo a partir de la malla.
            \item Implementar la lógica de los vehículos, de forma que sean capaces usar esta malla para tomar decisiones.
        \end{itemize}
        }
        \\ \hline
        \multicolumn{3}{|>{\hsize=\dimexpr3\hsize+4\tabcolsep+2\arrayrulewidth\relax\linewidth=\hsize}X|}{\textbf{Observaciones:}}
        \\ \hline
    \end{tabularx}
\end{table}

\begin{table}[H]
    \begin{tabularx}{\textwidth}{| X | X | X |}
        \hline
        \multicolumn{1}{|l|}{\textbf{Identificador:} HU2} & \multicolumn{2}{l|}{Modificación de ajustes antes de la carrera.}
        \\ \hline
        \multicolumn{3}{|>{\hsize=\dimexpr3\hsize+4\tabcolsep+2\arrayrulewidth\relax\linewidth=\hsize}X|}{\textbf{Descripción:} Como usuario, quiero modificar los ajustes de la simulación antes de comenzar la carrera.}
        \\ \hline
        \textbf{Estimación:} 2 & \textbf{Prioridad:} 2 & \textbf{Entrega:} 1
        \\ \hline
        \multicolumn{3}{|>{\hsize=\dimexpr3\hsize+4\tabcolsep+2\arrayrulewidth\relax\linewidth=\hsize}X|}{\textbf{Pruebas de aceptación:}
        \begin{itemize}
            \item No se podrá elegir un número menor de 1 para la cantidad de vueltas.
            \item No se podrá elegir un número menor de 2 coches.
            \item No se podrá dejar el campo del nombre y los apellidos vacíos.
        \end{itemize}
        }
        \\ \hline
        \multicolumn{3}{|>{\hsize=\dimexpr3\hsize+4\tabcolsep+2\arrayrulewidth\relax\linewidth=\hsize}X|}{\textbf{Requisitos relacionados:} RF1, RNF2, RNF3.}
        \\ \hline
        \multicolumn{3}{|>{\hsize=\dimexpr3\hsize+4\tabcolsep+2\arrayrulewidth\relax\linewidth=\hsize}X|}{\textbf{Tareas:}
        \begin{itemize}
            \item Implementar los deslizadores correspondientes al número de vueltas, de coches, hora, aguante y experiencia.
            \item Implementar los campos de texto para el nombre y los apellidos.
            \item Programar la lógica necesaria para que los coches aparezcan en las posiciones correctas.
            \item Programar la lógica de las flechas para elegir a los pilotos.
        \end{itemize}
        }
        \\ \hline
        \multicolumn{3}{|>{\hsize=\dimexpr3\hsize+4\tabcolsep+2\arrayrulewidth\relax\linewidth=\hsize}X|}{\textbf{Observaciones:}}
        \\ \hline
    \end{tabularx}
\end{table}


\begin{table}[H]
    \begin{tabularx}{\textwidth}{| X | X | X |}
        \hline
        \multicolumn{1}{|l|}{\textbf{Identificador:} HU4} & \multicolumn{2}{l|}{Visualización de la información de los pilotos.}
        \\ \hline
        \multicolumn{3}{|>{\hsize=\dimexpr3\hsize+4\tabcolsep+2\arrayrulewidth\relax\linewidth=\hsize}X|}{\textbf{Descripción:} Como usuario, quiero ver la posición actual y el estado de los pilotos de la carrera en curso.}
        \\ \hline
        \textbf{Estimación:} 3 & \textbf{Prioridad:} 3 & \textbf{Entrega:} 2
        \\ \hline
        \multicolumn{3}{|>{\hsize=\dimexpr3\hsize+4\tabcolsep+2\arrayrulewidth\relax\linewidth=\hsize}X|}{\textbf{Pruebas de aceptación:}
        \begin{itemize}
            \item Cuando un piloto cambie posiciones con otro, el marcador debe cambiar.
            \item El cambio del estado del piloto producido por la carrera será reflejado en el deslizador.
        \end{itemize}
        }
        \\ \hline
        \multicolumn{3}{|>{\hsize=\dimexpr3\hsize+4\tabcolsep+2\arrayrulewidth\relax\linewidth=\hsize}X|}{\textbf{Requisitos relacionados:} RF2, RNF3.}
        \\ \hline
        \multicolumn{3}{|>{\hsize=\dimexpr3\hsize+4\tabcolsep+2\arrayrulewidth\relax\linewidth=\hsize}X|}{\textbf{Tareas:}
        \begin{itemize}
            \item Implementar el panel de las posiciones y el estado.
            \item Implementar la lógica asociada al cálculo de la posición general de cada piloto.
            \item Implementar la funcionalidad de mostrar los deslizadores del piloto al pulsar.
        \end{itemize}
        }
        \\ \hline
        \multicolumn{3}{|>{\hsize=\dimexpr3\hsize+4\tabcolsep+2\arrayrulewidth\relax\linewidth=\hsize}X|}{\textbf{Observaciones:}}
        \\ \hline
    \end{tabularx}
\end{table}

\begin{table}[H]
    \begin{tabularx}{\textwidth}{| X | X | X |}
        \hline
        \multicolumn{1}{|l|}{\textbf{Identificador:} HU5} & \multicolumn{2}{l|}{Visualización de las vueltas restantes.}
        \\ \hline
        \multicolumn{3}{|>{\hsize=\dimexpr3\hsize+4\tabcolsep+2\arrayrulewidth\relax\linewidth=\hsize}X|}{\textbf{Descripción:} Como usuario, quiero ver el número de vueltas de la carrera en curso.}
        \\ \hline
        \textbf{Estimación:} 2 & \textbf{Prioridad:} 3 & \textbf{Entrega:} 1
        \\ \hline
        \multicolumn{3}{|>{\hsize=\dimexpr3\hsize+4\tabcolsep+2\arrayrulewidth\relax\linewidth=\hsize}X|}{\textbf{Pruebas de aceptación:}
        \begin{itemize}
            \item El marcador no deberá mostrar como vuelta actual el número 0 ni un número superior al número máximo.
        \end{itemize}
        }
        \\ \hline
        \multicolumn{3}{|>{\hsize=\dimexpr3\hsize+4\tabcolsep+2\arrayrulewidth\relax\linewidth=\hsize}X|}{\textbf{Requisitos relacionados:} RF9, RNF3.}
        \\ \hline
        \multicolumn{3}{|>{\hsize=\dimexpr3\hsize+4\tabcolsep+2\arrayrulewidth\relax\linewidth=\hsize}X|}{\textbf{Tareas:}
        \begin{itemize}
            \item Implementar la lógica para calcular la vuelta actual.
        \end{itemize}
        }
        \\ \hline
        \multicolumn{3}{|>{\hsize=\dimexpr3\hsize+4\tabcolsep+2\arrayrulewidth\relax\linewidth=\hsize}X|}{\textbf{Observaciones:}}
        \\ \hline
    \end{tabularx}
\end{table}


\begin{table}[H]
    \begin{tabularx}{\textwidth}{| X | X | X |}
        \hline
        \multicolumn{1}{|l|}{\textbf{Identificador:} HU6} & \multicolumn{2}{l|}{Exportación de la configuración.}
        \\ \hline
        \multicolumn{3}{|>{\hsize=\dimexpr3\hsize+4\tabcolsep+2\arrayrulewidth\relax\linewidth=\hsize}X|}{\textbf{Descripción:} Como usuario, quiero guardar la configuración de una carrera para su posterior uso.}
        \\ \hline
        \textbf{Estimación:} 2 & \textbf{Prioridad:} 4 & \textbf{Entrega:} 3
        \\ \hline
        \multicolumn{3}{|>{\hsize=\dimexpr3\hsize+4\tabcolsep+2\arrayrulewidth\relax\linewidth=\hsize}X|}{\textbf{Pruebas de aceptación:}}
        \\ \hline
        \multicolumn{3}{|>{\hsize=\dimexpr3\hsize+4\tabcolsep+2\arrayrulewidth\relax\linewidth=\hsize}X|}{\textbf{Requisitos relacionados:} RF7.}
        \\ \hline
        \multicolumn{3}{|>{\hsize=\dimexpr3\hsize+4\tabcolsep+2\arrayrulewidth\relax\linewidth=\hsize}X|}{\textbf{Tareas:}
        \begin{itemize}
            \item Implementar la lógica referente a la escritura de la configuración en un archivo externo.
        \end{itemize}
        }
        \\ \hline
        \multicolumn{3}{|>{\hsize=\dimexpr3\hsize+4\tabcolsep+2\arrayrulewidth\relax\linewidth=\hsize}X|}{\textbf{Observaciones:}}
        \\ \hline
    \end{tabularx}
\end{table}

\begin{table}[H]
    \begin{tabularx}{\textwidth}{| X | X | X |}
        \hline
        \multicolumn{1}{|l|}{\textbf{Identificador:} HU7} & \multicolumn{2}{l|}{Importación de la configuración.}
        \\ \hline
        \multicolumn{3}{|>{\hsize=\dimexpr3\hsize+4\tabcolsep+2\arrayrulewidth\relax\linewidth=\hsize}X|}{\textbf{Descripción:} Como usuario, quiero cargar la configuración de una carrera almacenada en un archivo.}
        \\ \hline
        \textbf{Estimación:} 2 & \textbf{Prioridad:} 4 & \textbf{Entrega:} 3
        \\ \hline
        \multicolumn{3}{|>{\hsize=\dimexpr3\hsize+4\tabcolsep+2\arrayrulewidth\relax\linewidth=\hsize}X|}{\textbf{Pruebas de aceptación:}
        \begin{itemize}
            \item El fichero generado en la fase de exportación debe ser legible por la aplicación.
        \end{itemize}
        }
        \\ \hline
        \multicolumn{3}{|>{\hsize=\dimexpr3\hsize+4\tabcolsep+2\arrayrulewidth\relax\linewidth=\hsize}X|}{\textbf{Requisitos relacionados:} RF8.}
        \\ \hline
        \multicolumn{3}{|>{\hsize=\dimexpr3\hsize+4\tabcolsep+2\arrayrulewidth\relax\linewidth=\hsize}X|}{\textbf{Tareas:}
        \begin{itemize}
            \item Implementar la lógica referente a la modificación de los parámetros por los valores que se encuentren en el archivo.
        \end{itemize}
        }
        \\ \hline
        \multicolumn{3}{|>{\hsize=\dimexpr3\hsize+4\tabcolsep+2\arrayrulewidth\relax\linewidth=\hsize}X|}{\textbf{Observaciones:}}
        \\ \hline
    \end{tabularx}
\end{table}

\newpage

\begin{table}[H]
    \begin{tabularx}{\textwidth}{| X | X | X |}
        \hline
        \multicolumn{1}{|l|}{\textbf{Identificador:} HU8} & \multicolumn{2}{l|}{Pausado y reanudación de la simulación.}
        \\ \hline
        \multicolumn{3}{|>{\hsize=\dimexpr3\hsize+4\tabcolsep+2\arrayrulewidth\relax\linewidth=\hsize}X|}{\textbf{Descripción:} Como usuario, quiero pausar y reanudar la simulación en curso.}
        \\ \hline
        \textbf{Estimación:} 0,5 & \textbf{Prioridad:} 5 & \textbf{Entrega:} 3
        \\ \hline
        \multicolumn{3}{|>{\hsize=\dimexpr3\hsize+4\tabcolsep+2\arrayrulewidth\relax\linewidth=\hsize}X|}{\textbf{Pruebas de aceptación:}
        \begin{itemize}
            \item Al pulsar el botón, debe alternar entre pausado y reanudado.
            \item Al encontrarse la simulación pausada, no se podrá modificar su velocidad.
        \end{itemize}
        }
        \\ \hline
        \multicolumn{3}{|>{\hsize=\dimexpr3\hsize+4\tabcolsep+2\arrayrulewidth\relax\linewidth=\hsize}X|}{\textbf{Requisitos relacionados:} RF5, RNF2.}
        \\ \hline
        \multicolumn{3}{|>{\hsize=\dimexpr3\hsize+4\tabcolsep+2\arrayrulewidth\relax\linewidth=\hsize}X|}{\textbf{Tareas:}
        \begin{itemize}
            \item Incluir el botón en la interfaz.
            \item Implementar la lógica relacionada con el botón y el pausado de la simulación.
        \end{itemize}
        }
        \\ \hline
        \multicolumn{3}{|>{\hsize=\dimexpr3\hsize+4\tabcolsep+2\arrayrulewidth\relax\linewidth=\hsize}X|}{\textbf{Observaciones:}}
        \\ \hline
    \end{tabularx}
\end{table}

\begin{table}[H]
    \begin{tabularx}{\textwidth}{| X | X | X |}
        \hline
        \multicolumn{1}{|l|}{\textbf{Identificador:} HU9} & \multicolumn{2}{l|}{Modificación de la velocidad de la simulación.}
        \\ \hline
        \multicolumn{3}{|>{\hsize=\dimexpr3\hsize+4\tabcolsep+2\arrayrulewidth\relax\linewidth=\hsize}X|}{\textbf{Descripción:} Como usuario, quiero modificar la velocidad de la simulación.}
        \\ \hline
        \textbf{Estimación:} 0,5 & \textbf{Prioridad:} 5 & \textbf{Entrega:} 3
        \\ \hline
        \multicolumn{3}{|>{\hsize=\dimexpr3\hsize+4\tabcolsep+2\arrayrulewidth\relax\linewidth=\hsize}X|}{\textbf{Pruebas de aceptación:}
        \begin{itemize}
            \item Al pulsar el botón, debe alternar entre distintas velocidades de simulación.
            \item Este botón no deberá funcionar si la simulación se encuentra pausada.
        \end{itemize}
        }
        \\ \hline
        \multicolumn{3}{|>{\hsize=\dimexpr3\hsize+4\tabcolsep+2\arrayrulewidth\relax\linewidth=\hsize}X|}{\textbf{Requisitos relacionados:} RF4, RNF2.}
        \\ \hline
        \multicolumn{3}{|>{\hsize=\dimexpr3\hsize+4\tabcolsep+2\arrayrulewidth\relax\linewidth=\hsize}X|}{\textbf{Tareas:}
        \begin{itemize}
            \item Incluir el botón en la interfaz.
            \item Implementar la lógica relacionada con el botón y la modificación de la velocidad de simulación.
        \end{itemize}
        }
        \\ \hline
        \multicolumn{3}{|>{\hsize=\dimexpr3\hsize+4\tabcolsep+2\arrayrulewidth\relax\linewidth=\hsize}X|}{\textbf{Observaciones:}}
        \\ \hline
    \end{tabularx}
\end{table}

\chapter{Diseño}

En este capítulo se abordarán las cuestiones de diseño la interfaz de usuario, explicando de manera detallada cada elemento que la compone, junto a algunos bocetos. Además, se incluirá un diagrama de navegación para saber a qué pantallas lleva cada opción.

\bigskip

Cada cuestión anteriormente mencionada se dividirá en secciones a continuación:

\section{Diseño de la Interfaz de Usuario}

La interfaz de usuario del configurador de la simulación está dividida en dos secciones principales: a la izquierda se encuentra la sección para el número de vehículos y de pilotos, la disciplina y las opciones de importación y exportación de la configuración. A la derecha está la segunda sección, compuesta por las características de cada piloto y el botón para comenzar la simulación.

\bigskip

Durante la carrera, se mostrará un listado de posiciones de los pilotos y un contador de vueltas en la parte izquierda de la pantalla. Este listado permitirá visualizar y modificar el estado de cada piloto en tiempo real. La disposición de los elementos será similar a la de disciplinas como la Fórmula 1, donde el ranking de pilotos y el contador de vueltas aparecen a la izquierda, y en cada celda de posición aparecen las tres primeras letras del apellido. En este proyecto, he decidido añadir también la primera letra del nombre, en caso de que haya varios pilotos con el mismo apellido.

\bigskip

A continuación se encuentran algunos bocetos de ambas pantallas:

\begin{figure}[H]
    \centering
    \includegraphics[width=\textwidth]{imagenes/pag1.png}
    \caption{Interfaz del configurador antes de la carrera.}
\end{figure}

\begin{figure}[H]
    \centering
    \includegraphics[width=\textwidth]{imagenes/pag2.png}
    \caption{Interfaz de las opciones durante la carrera.}
\end{figure}

\section{Diagrama de navegación}

La navegación por la interfaz de usuario en la aplicación es relativamente sencilla, al estar formada por dos pantallas: el configurador de simulación y la carrera en curso.

\bigskip
\newpage
El diagrama de navegación de la aplicación se encuentra a continuación:

% foto del diagrama de navegacion
\begin{figure}[H]
    \centering
    \includegraphics[width=0.9\textwidth]{imagenes/nav.png}
    \caption{Diagrama de navegación de la interfaz de usuario de la aplicación.}
 \end{figure}

Como se puede observar en el diagrama de navegación, al pulsar el botón de ejecutar aparecerá la nueva pantalla para ejecutar la simulación y dar comienzo a la carrera.

% Como se puede observar, la navegación por la aplicación es sencilla, al solo tener dos pantallas y un solo botón para ir a la siguiente. Si se pulsa sobre el botón de ejecutar en el configurador de la carrera, se procederá a cargar la simulación y dará comienzo la carrera.

\section{Entidades y atributos}

En esta sección se detallarán las distintas entidades que componen la aplicación, junto a los atributos necesarios para implementar la lógica. Cada entidad y sus atributos estarán en una subsección a continuación:

\subsection{AStarNode}
\textbf{Descripción: }Representa una celda de la matriz utilizada para ejecutar el algoritmo A*.

\bigskip

\textbf{Atributos: }
\begin{itemize}
    \item \textbf{neighbors} : AStarNode\verb|[]|
    \begin{itemize}
        \item \textbf{Descripción: }Almacena las celdas con las que colinda.
    \end{itemize}

    \item \textbf{hasObstacle} : boolean
    \begin{itemize}
        \item \textbf{Descripción: }Almacena si la celda actual está ocupada por un coche.
    \end{itemize}

    \item \textbf{isOptimal} : boolean
    \begin{itemize}
        \item \textbf{Descripción: }Indica si la celda forma parte de la ruta óptima del circuito. Puede estar ocupada.
    \end{itemize}

    \item \textbf{isDelimiter} : boolean
    \begin{itemize}
        \item \textbf{Descripción: }Indica si la celda forma parte de los límites de pista. Esta celda estará siempre ocupada y tendrá prioridad frente a las demás.
    \end{itemize}

    \item \textbf{isCheckpoint} : boolean
    \begin{itemize}
        \item \textbf{Descripción: }Indica si la celda forma parte de un checkpoint. Esta celda estará siempre libre, con independencia de si hay un coche o no.
    \end{itemize}
\end{itemize}
\subsection{AStarGrid}
\textbf{Descripción: }Representa la matriz navegación utilizada para ejecutar A*.

\bigskip

\textbf{Atributos: }
\begin{itemize}
    \item \textbf{nodes} : AStarNode\verb|[]|
    \begin{itemize}
        \item \textbf{Descripción: }Almacena todos los nodos que forman la malla de navegación. El primero es el de más arriba a la izquierda y el último es el de más abajo a la derecha.
    \end{itemize}
    \item \textbf{gridSize} : Integer
    \begin{itemize}
        \item \textbf{Descripción: }Cantidad de cubos que va a haber a lo largo y a lo ancho.
    \end{itemize}
    \item \textbf{nodeSize} : Integer
    \begin{itemize}
        \item \textbf{Descripción: }Tamaño que tendrá la celda. Por defecto es 100.
    \end{itemize}
    \item \textbf{startingPosition} : Vector
    \begin{itemize}
        \item \textbf{Descripción: }Posición por la que se debe comenzar a generar la malla. Esta coordenada representa la posición más a la izquierda y arriba de la malla.
    \end{itemize}
    % \item \textbf{openList} : PriorityQueue
\end{itemize}

\subsection{PriorityQueue}
\textbf{Descripción: }Representa una cola con prioridad ascendente, utilizada para A*.

\bigskip

\textbf{Atributos: }
\begin{itemize}
    \item \textbf{queue} : QueueElement\verb|[]|
    \begin{itemize}
        \item \textbf{Descripción: }Lista con todos los nodos visitados y sus costes. Se encuentran ordenados.
    \end{itemize}
\end{itemize}


\subsection{QueueElement}
\textbf{Descripción: }Estructura de datos que es utilizada por PriorityQueue para realizar los cálculos.

\bigskip

\textbf{Atributos: }
\begin{itemize}
    \item \textbf{node} : AStarNode
    \begin{itemize}
        \item \textbf{Descripción: }Representa el nodo que se va a insertar en la cola.
    \end{itemize}

    \item \textbf{cost} : Float
    \begin{itemize}
        \item \textbf{Descripción: }Representa el coste total del nodo, incluyendo la heurística.
    \end{itemize}
\end{itemize}
\subsection{AStarPathfinder}
\textbf{Descripción: }Implementa el algoritmo A* para ser utilizado por los coches.


\subsection{SportsCarPawn\_AI}
\textbf{Descripción: }Representa a los coches que compiten por el circuito.

\bigskip

\textbf{Atributos: }

\begin{itemize}
    % id no va aqui por el modo entrenamiento
    \item \textbf{PID} : PID
    \item \textbf{Kp} : Float
    \item \textbf{Ki} : Float
    \item \textbf{Kd} : Float
    \item \textbf{nombre} : String
    \item \textbf{apellidos} : String
    \item \textbf{agresividad} : Float
    \item \textbf{aguante} : Float
    \item \textbf{position} : Integer
    \item \textbf{currentCheckpoint} : Integer
    \item \textbf{currentLap} : Integer
    \item \textbf{AGRESIVIDAD\_MAX\_BRAKING\_DISTANCE} : Float
    \begin{itemize}
        \item Indica la distancia máxima de frenada que se le puede substraer debido a la agresividad.
    \end{itemize}
    \item \textbf{AGRESIVIDAD\_MAX\_SPEED} : Float
    \begin{itemize}
        \item Indica la velocidad extra que se le puede añadir debido a la agresividad.
    \end{itemize}
    \item \textbf{AGUANTE\_MIN\_SPEED} : Float
    \begin{itemize}
        \item Indica el decremento en la velocidad que sufre el piloto por el desgaste.
    \end{itemize}
    \item \textbf{REVERSE\_TIME} : Float
    \begin{itemize}
        \item Tiempo máximo en segundos que el vehículo da marcha atrás para recuperarse de un accidente.
    \end{itemize}
    \item \textbf{stuckTime} : Float
    \begin{itemize}
        \item Tiempo que permanece atascado el vehículo (en segundos).
    \end{itemize}
    \item \textbf{isReversing} : boolean
    \begin{itemize}
        \item Indica si el vehículo está dando marcha atrás.
    \end{itemize}
    \item \textbf{goSlowTimer} : Float
    \begin{itemize}
        \item Indica la cantidad de segundo que el vehículo está yendo lento para volver a la pista después de un accidente.
    \end{itemize}
    \item \textbf{MAX\_GO\_SLOW\_TIME} : Float       % AQUI DEBERIA IR TAMBIEN SLOW_TIMER, QUE ESTA REPETIDO
    \begin{itemize}
        \item Indica el número de segundos que debe ir lento para volver a pista
    \end{itemize}
    \item \textbf{goSlowAfterCrash} : boolean
    \begin{itemize}
        \item Indica si el coche debe ir lento para volver a pista después de un accidente.
    \end{itemize}
    \item \textbf{reverseTime} : Float
    \begin{itemize}
        \item Cantidad de segundos que lleva el coche dando marcha atrás.
    \end{itemize}
    \item \textbf{currentOvertakeTime} : Float
    \begin{itemize}
        \item Cantidad de segundos que el vehículo lleva intentando un adelantamiento.
    \end{itemize}
    \item \textbf{isOvertaking} : boolean
    \begin{itemize}
        \item Indica si el vehículo está intentando adelantar en el momento.
    \end{itemize}
    \item \textbf{OVERTAKE\_TIME} : Float
    \begin{itemize}
        \item Tiempo máximo que debe intentar un adelantamiento.
    \end{itemize}
    \item \textbf{Spline} : SplinePath
    \begin{itemize}
        \item Almacena la ruta actual calculada por el algoritmo de navegación.
    \end{itemize}
    \item \textbf{brakingG} : Float
    \item \textbf{checkpoints} : Checkpoint\verb|[]|
    \item \textbf{AStarGridRef} : AStarGrid
    \item \textbf{currentPath} : Vector\verb|[]|
    \begin{itemize}
        \item Puntos en coordenadas del mundo obtenidos del algoritmo de navegación.
    \end{itemize}
    \item \textbf{marcadorRef} : Marcador
    \item \textbf{pathfinderRef} : AStarPathfinder
    \item \textbf{color} : Linear Color
    \begin{itemize}
        \item Representa el color del vehículo.
    \end{itemize}
\end{itemize}

\subsection{Marcador}
\textbf{Descripción: }Entidad utilizada para generar la lista de posiciones y el contador de vueltas, así como la actualización de ambos.

\bigskip

\textbf{Atributos: }
\begin{itemize}
    \item \textbf{uiRef} : Panel
    \begin{itemize}
        \item \textbf{Descripción: }Referencia al panel con el contador de vueltas y el ranking.
    \end{itemize}
    
    \item \textbf{cars} : SportsCarPawn\_AI\verb|[]|
    \begin{itemize}
        \item \textbf{Descripción: }Almacena una referencia de todos los coches de la carrera. Se encuentran siempre ordenados, estando en posiciones menores los vehículos que mejor van.
    \end{itemize}

    \item \textbf{checkpoints} : Checkpoint\verb|[]|
    \begin{itemize}
        \item \textbf{Descripción: }Almacena todos los checkpoints que hay en la escena. Están ordenados de manera ascendente, de forma que los índices superiores indican checkpoints posteriores del circuito.
    \end{itemize}

    \item \textbf{currentLap} : Integer
    \begin{itemize}
        \item \textbf{Descripción: }Almacena la vuelta actual de la carrera.
    \end{itemize}

    \item \textbf{MAX\_LAPS} : Integer 
    \begin{itemize}
        \item \textbf{Descripción: }Indica el número de vueltas que tiene la carrera.
    \end{itemize}
\end{itemize}

\subsection{Panel}
\textbf{Descripción: }Interfaz de usuario en la que se muestra el contador de vueltas y el ranking.

\bigskip

\textbf{Atributos: }
\begin{itemize}
    \item \textbf{LAPS} : Text Widget
    \begin{itemize}
        \item \textbf{Descripción: }Es el texto que aparece en la propia interfaz de usuario.
    \end{itemize}

    \item \textbf{VerticalBox\_0} : Vertical Box
    \begin{itemize}
        \item \textbf{Descripción: }Marco donde se almacena cada celda de la clase Posicion con la información de cada piloto.
    \end{itemize}
    % \item cars
\end{itemize}

\subsection{Posicion}
\textbf{Descripción: }Cuadro básico que representa una celda en el ranking.

\bigskip

\textbf{Atributos: }
\begin{itemize}
    \item \textbf{Agresividad} : Sliding Bar
    \begin{itemize}
        \item \textbf{Descripción: }Barra deslizadora que indica la agresividad del conductor.
    \end{itemize}

    \item \textbf{Aguante} : Sliding Bar
    \begin{itemize}
        \item \textbf{Descripción: }Barra deslizadora que indica el aguante físico y mental del conductor.
    \end{itemize}

    % \item TextBlock_47
\end{itemize}

\subsection{ArrayOfCars}
\textbf{Descripción: }Estructura utilizada para almacenar el número de coches que hay en un checkpoint concreto.

\bigskip

\textbf{Atributos: }
\begin{itemize}
    \item \textbf{cars} : SportsCarPawn\_AI\verb|[]|
    \begin{itemize}
        \item \textbf{Descripción: }Almacena todos los coches para un checkpoint dado.
    \end{itemize}
\end{itemize}

\subsection{ArrayOfArrayOfCars}
\textbf{Descripción: }Estructura que almacena en que vuelta y en que checkpoint se encuentra cada vehículo.

\bigskip

\textbf{Atributos: }
\begin{itemize}
    \item \textbf{laps} : Map$<$Integer, ArrayOfCars$>$
    \begin{itemize}
        \item \textbf{Descripción: }Almacena todos los coches y sus checkpoints que se encuentran en una vuelta dada.
    \end{itemize}
\end{itemize}


\subsection{PID}
\textbf{Descripción: }Entidad encargada de implementar la lógica del PID, exponiendo funciones para su uso.

\bigskip

\textbf{Atributos: }
\begin{itemize}
    \item \textbf{Kp} : Float
    \begin{itemize}
        \item \textbf{Descripción: }Constante proporcional del PID.
    \end{itemize}
    
    \item \textbf{Ki} : Float
    \begin{itemize}
        \item \textbf{Descripción: }Constante integral del PID.
    \end{itemize}
    
    \item \textbf{Kd} : Float
    \begin{itemize}
        \item \textbf{Descripción: }Constante derivativa del PID.
    \end{itemize}        

    % \item minoutput, maxoutput, erroractualfuera

    \item \textbf{errorPasado} : Float
    \begin{itemize}
        \item \textbf{Descripción: }Error en el instante anterior al actual. Necesario para implementar la componente integral.
    \end{itemize}

    \item \textbf{errorAcumulado} : Float
    \begin{itemize}
        \item \textbf{Descripción: }Error acumulado con el tiempo. Esta variable es utilizada por la componente integral.
    \end{itemize}
\end{itemize}

\subsection{Checkpoint}
\textbf{Descripción: }Entidad encargada de poner un checkpoint en el mundo, para poder realizar cálculos.

\bigskip

\textbf{Atributos: }
\begin{itemize}
    \item \textbf{id} : Integer
    \begin{itemize}
        \item \textbf{Descripción: }Identificador del checkpoint. Este valor debe ir creciendo en el mismo orden en el que los coches pasan los checkpoints en el circuito.
    \end{itemize}
    
    \item \textbf{speed} : Float
    \begin{itemize}
        \item \textbf{Descripción: }Velocidad aconsejable a la que debe pasar el coche el tramo hasta el siguiente checkpoint.
    \end{itemize}

    % \item \textbf{apexType}
\end{itemize}

\subsection{SplinePath}
\textbf{Descripción: }Entidad que almacena un spline.

\subsection{ChangeSpline}
\textbf{Descripción: }Entidad referente al algoritmo genético. Utilizada para cambiar de spline prefijado en el modo entrenamiento.

\subsection{Genome}
\textbf{Descripción: }Estructura de datos que almacena las constantes del PID de un coche, así como su identificador y \textit{fitness} para el algoritmo genético.

\bigskip

\textbf{Atributos: }
\begin{itemize}
    \item \textbf{Kp} : Float
    \begin{itemize}
        \item \textbf{Descripción: }Constante proporcional del PID.
    \end{itemize}
    
    \item \textbf{Ki} : Float
    \begin{itemize}
        \item \textbf{Descripción: }Constante integral del PID.
    \end{itemize}    

    \item \textbf{Kd} : Float
    \begin{itemize}
        \item \textbf{Descripción: }Constante derivativa del PID.
    \end{itemize}    

    \item \textbf{fitness} : Float
    \begin{itemize}
        \item \textbf{Descripción: }Almacena la probabilidad de ser escogido para la siguiente generación.
    \end{itemize}

    \item \textbf{id} : Integer
    \begin{itemize}
        \item \textbf{Descripción: }Almacena el identificador del vehículo.
    \end{itemize}
\end{itemize}

\subsection{TrainingStop}
\textbf{Descripción: }Utilizado en el algoritmo genético. Marca el punto final de los coches.

\bigskip

\textbf{Atributos: }
\begin{itemize}
    \item \textbf{currentCars} : Integer
    \begin{itemize}
        \item \textbf{Descripción: }Almacena el número de vehículos que aún siguen en pie. Esto es debido a que aquellos coches que se chocan o se caen son eliminados.
    \end{itemize}

    \item \textbf{MAX\_GENOMES} : Integer
    \begin{itemize}
        \item \textbf{Descripción: }Indica el número de coches que se lanzan por generación.
    \end{itemize}

    \item \textbf{values} : genome\verb|[]|
    \begin{itemize}
        \item \textbf{Descripción: }Almacena todas las constantes de los hijos de una generación.
    \end{itemize}

    \item \textbf{identificadores} : Integer
    \begin{itemize}
        \item \textbf{Descripción: }Contador para asignar el identificador a todos los coches.
    \end{itemize}

    \item \textbf{probabilidadMutacion} : Float
    \begin{itemize}
        \item \textbf{Descripción: }Indica la probabilidad con la que un coche pueda llegar a tener una mutación.
    \end{itemize}

    % isEnabled (hay que eliminarlo en el proyecto final), prueba
\end{itemize}

\chapter{Implementación}

Este capítulo tratará sobre las herramientas utilizadas para la implementación y planificación, así como los distintos algoritmos que se emplearán para realizar la simulación. Además, se detallará el diseño de los \textit{blueprints} y se describirán las reglas que seguirá el algoritmo de navegación.

\section{Herramientas utilizadas}

En el desarrollo del proyecto se utilizará \verb|git| para el control de versiones del proyecto y se almacenará en un repositorio de GitHub. Se hará uso de ramas 
% y \textit{Pull Requests} 
para implementar cada una de las historias de usuario, con el objetivo de seguir una metodología de trabajo similar a las que se realizan en metodologías ágiles.


% \bigskip

% En cuanto a la implementación de la aplicación, se utilizará C++ en las partes más críticas, en lugar de \textit{blueprints}. Para ello, se empleará Visual Studio como entorno de desarrollo.

\bigskip

% En cuanto a herramientas para la planificación, se hará uso de la herramienta \planApp para llevar un control de las Historias de Usuario que se realizan y pendientes en cada Sprint.
En lo que respecta a herramientas de planificación, se utilizará \planApp para llevar un control de las historias de usuario realizadas y pendientes en cada sprint.

% foto de \planApp
\begin{figure}[H]
    \centering
    \includegraphics[width=0.75\textwidth]{imagenes/trello.png}
    \caption{Tablero de ejemplo de Trello\cite{tablero-trello}.}
\end{figure}

\section{Algoritmos utilizados}

Voy a dividir en subsecciones los diversos algoritmos que componen la simulación:

\subsection{\textcolor{red}{Algoritmo para el giro del volante del coche}}

En el control del volante de los vehículos, correspondiente a HU3 y realizada en el sprint 4, se ha utilizado un controlador PID (\textit{Proportional, Integral} y \textit{Derivative}). Este controlador se utiliza para regular el error de un sistema de manera suave y evitando producir muchas desviaciones mediante el ajuste de tres constantes para la componente proporcional, integral y derivativa. Cabe destacar que la variable a la que el sistema debe llegar se llama SP (\textit{Set Point}) y el valor actual del sistema se denomina PV (\textit{Process Variable}). 

% foto de una grafica de esas, quitar el bigskip si se encuentra alguna
% \caption{Diferencias de respuestas de un control estandar frente al controlador PID.}
\bigskip

% esta parte mejor en una lista
Para que el controlador funcione bien, es necesario modificar sus tres constantes: 
% Proporcional, Integral y Derivativa. 

\begin{itemize}
    \item \textbf{Proporcional: }Actuará de manera proporcional al error actual, multiplicada por la constante asociada. Si solo se utiliza este componente, el sistema se pasará del SP, haciendo que tenga que volver a corregir, produciendo así demasiadas oscilaciones para llegar al SP. 
    
    Esta constante la he implementado de la siguiente forma en \textit{Unreal}:

    % foto de la constante
\begin{figure}[H]
    \centering
    \includegraphics[width=\textwidth]{example-image-a}
    \caption{Implementación usando \textit{blueprints} de la componente Proporcional.}
\end{figure}

    \item \textbf{Integral: }Se encarga de derivar el error con respecto al tiempo, con el objetivo de anular errores residuales e incluso del propio sensor. Por normal general, seguirá pasándose del SP, ya que de esto se encarga la componente derivativa.
    
    En la aplicación, la constante se ha implementado de la siguiente forma: 

    % foto de la constante
    \begin{figure}[H]
        \centering
        \includegraphics[width=\textwidth]{example-image-b}
        \caption{Implementación usando \textit{blueprints} de la componente Integral.}
    \end{figure}

    \item \textbf{Derivativa: }Se utiliza para estimar como evolucionará la curva con el tiempo, basándose en su pendiente. Esta componente es la que se encarga de minimizar las oscilaciones, al reducir el efecto del actuador si se acerca demasiado rápido al SP.
    
    La constante se ha programado de la siguiente forma:

    % foto de la constante
    \begin{figure}[H]
        \centering
        \includegraphics[width=\textwidth]{example-image-c}
        \caption{Implementación usando \textit{blueprints} de la componente Derivativa.}
    \end{figure}    
\end{itemize}

\bigskip

El rango de valores que pueden tener las variables dependen del sistema al que se aplique. No obstante, todos los valores son números reales positivos, incluyendo el 0, que deshabilita la componente.

\bigskip

Cabe destacar que, en mi caso, el SP el punto más cercano a la ruta generada, y el error es la distancia del coche a dicho punto.

\bigskip

% rescribir la ultima parte ... y utilizando tambien el valor...
La calibración del PID se realiza modificando las constantes asociadas a cada componente. Existen diversos métodos como el de Ziegler-Nichols \cite{enwiki:1140258750}, que consiste en modificar solo la parte proporcional, dejando las demás a 0, hasta que el sistema comience a oscilar de manera estable, en ese momento se debe calcular la frecuencia a la que oscila y utilizando también el valor de la constante proporcional, se pueden obtener las demás mediante un cálculo matemático.

\bigskip

Dado que este método no me dio los resultados deseados, decidí implementar un algoritmo genético para obtener las constantes del controlador PID. Consiste en lanzar un conjunto de coches con valores de las constantes del PID aleatorios positivos al principio, con el objetivo de que intenten llegar a la meta con el menor error posible (por error se sigue entendiendo como la distancia del punto más cercano a la ruta con el coche). Aquellos con menos error, tienen más posibilidades de ser seleccionados para generaciones futuras. Una vez obtenido el error, se calcula su inversa y se obtiene la probabilidad de ser elegido mediante el nuevo valor entre la suma de todos. Esto dará como resultado unas probabilidades que al ser sumadas darán 1 (100\%). 

\bigskip

La selección se realiza lanzando una ruleta \cite{enwiki:1141636554}, cuyos sectores son divididos en función de la probabilidad calculada anteriormente.
% la cual se divide según las probabilidades de cada coche. 
Una vez que se han elegido los coches que van a ser padres, se emparejan y se mezclan sus constantes para obtener dos hijos de cada pareja. Después, una de las tres constantes del PID en algunos de los hijos es mutada con un valor aleatorio de rango [-0.1,0.1] que se le añade al valor exsitente. Finalmente se sustituyen los coches no elegidos para la siguiente generación por los hijos con las constantes mezcladas y mutadas, manteniendo a los padres, y se vuelve a ejecutar de nuevo el algoritmo.
% no elegidos por los hijos, manteniendo también a los padres.

\bigskip

Utilizando este algoritmo, obtuve unos valores que funcionaban bien para el circuito. No obstante, tras ejecutar varias veces el algoritmo, me he dado cuenta de que la constante integral siempre tiende a 0. Para solucionarlo, he decidido obligarle a que tenga un valor mínimo de 0,01.

\subsection{Algoritmo de navegación}

En cuanto al algoritmo de navegación, correspondiente a HU2, HU4 y HU5 e implementadas en los sprints 5 y 6, he utilizado \finalAlg para obtener la ruta más óptima. Este algoritmo debe ejecutarse en determinados momentos, ya que está pensado principalmente para entornos estáticos.


\subsection{Algoritmo para las posiciones de los pilotos}

En lo que se refiere al cálculo de las posiciones de cada piloto, correspondientes a HU7 y HU8 e implementadas en el sprint 7, he utilizado una estructura de datos algo más compleja. Los vehículos almacenan la vuelta en la que están y el checkpoint; es decir, el sector del circuito en el que se encuentran. Con la información anterior, he creado un map que almacene como clave la vuelta y como valor otro map, que a su vez tiene como clave el checkpoint y como valor un array con todos los vehículos que se encuentran en ese estado. Para mostrarlo, solo hace falta iterar primero por aquellas claves mayores, ya que son los que más vueltas y más lejos del circuito están.

% foto de la estructura de datos
% \usepackage{multirow}

\begin{table}[H]
    \centering
    \begin{tblr}{
        cells = {c},
        row{1} = {Silver},
        cell{1}{2} = {c=2}{},
        cell{2}{1} = {r=4}{},
        cell{2}{2} = {Alto},
        cell{2}{3} = {Alto},
        cell{7}{1} = {r=2}{},
        cell{7}{2} = {Alto},
        cell{7}{3} = {Alto},
        vlines,
        hline{1-2,6-7,9} = {-}{},
        hline{3-5,8} = {2-3}{},
            }
        \textbf{Clave (vuelta actual)} & \textbf{Valor }                    &                \\
        0                              & \textbf{Clave (checkpoint actual)} & \textbf{Valor} \\
                                       & 0                                  & {[}c1]         \\
                                       & ...                                &                \\
                                       & i                                  & {[}c2, c3]     \\
        ...                            & ...                                & ...            \\
        i                              & \textbf{Clave (checkpoint actual)} & \textbf{Valor} \\
                                       & 3                                  & {[}c4]
    \end{tblr}
    \caption{Representación de la estructura de datos que almacena las posiciones de los pilotos durante la carrera.}
\end{table}

Para saber cuando lanzar la actualización de la lista de posiciones, cada coche calcula continuamente si el que está por delante de él sigue así, en caso contrario lanza el evento de actualización.


\subsection{Contador de vueltas}

En cuanto a la actualización del contador de vueltas, correspondiente a HU9 y realizado en el sprint 8, he hecho que todos los coches cuando acaben una vuelta intenten actualizarlo, en caso de que la vuelta actual sea mayor o igual, no se actualiza. De esta forma, es más fácil llevar un control más preciso de las vueltas.

% Capitulo conclusiones.

% Que se ha hecho en el proyecto, logros conseguidos. Posibles vias de futuro para trabajo: mejoras o añadir mas funcionalidad.

\chapter{Conclusiones y trabajos futuros}
% Como conclusión, he logrado alcanzar todos los objetivos que había planteado de manera satisfactoria. Aunque he tenido problemas que ralentizaron el desarrollo, como el ajuste correcto de las constantes del controlador PID, problemas de rendimiento en el algoritmo de navegación, fallos al calcular el cambio de posición en el marcador, colisiones extrañas durante la carrera y errores en la apertura de cuadros de diálogo, he podido resolverlos y continuar con la implementación de la aplicación, cumpliendo con los plazos planificados.

% \section{Competencias adquiridas}

Gracias al desarrollo del proyecto he adquirido un mayor conocimiento sobre diversos algoritmos de navegación, especialmente en el algoritmo A*, junto a sus diversas heurísticas y optimizaciones. También he conseguido entender como funcionan y se implementan las colas con prioridad utilizando \textit{Binary Heaps}.

\bigskip
 
He comprendido el funcionamiento de los algoritmos genéticos y he realizado una implementación utilizando \textit{blueprints} para el entrenamiento de las constantes del controlador del volante, obteniendo mejores resultados en cada generación.

\bigskip


% He ampliado mis conocimientos en el manejo de \textit{Unreal Engine} y en la programación mediante \textit{blueprints}. 
En cuanto al manejo de \textit{Unreal Engine} y la programación con \textit{blueprints}, he ampliado mis conocimientos ya existentes.
Asimismo, he aprendido a programar utilizando la API en C++, cuya gestión de memoria es distinta a como se trabaja en C++ nativo. El uso de C++ me ha permitido crear algoritmos más eficientes, así como actores, objetos y la posibilidad de compartir su funcionalidad con otros \textit{blueprints}.

\bigskip

Además, he aplicado conceptos matemáticos y físicos en el desarrollo de funciones para el seguimiento de rutas, cálculo de posiciones entre los vehículos y determinación de la distancia de frenado.

\bigskip

Por último, aunque no era uno de los objetivos, he profundizado mis conocimientos en el programa Blender, especialmente en el modelado de objetos, mapeado UV, generación de texturas normales y \textit{rigging} de los vehículos.  


\section{Trabajos futuros}

He conseguido implementar todas las historias de usuario planificadas, pero en el futuro se podría incluir las siguientes funcionalidades en la aplicación:

\begin{itemize}
    \item Ciclo día/noche modificable de manera similar al estado de los pilotos.
    \item Variabilidad de rendimiento entre vehículos de una misma disciplina.
    \item Nuevas características de los pilotos que afecten a su rendimiento en la carrera.
    \item Posibilidad de conducir uno de los vehículos.
    \item Inclusión de sonido de motor en los vehículos.
\end{itemize}

\bigskip

Además, se podrían mejorar algunas de las siguientes características:

\begin{itemize}
    \item Mejora de rendimiento en términos de fotogramas por segundo durante la simulación.
    \item Reducción del tiempo de lanzamiento de la simulación.
    \item Toma de decisiones basadas en más factores.
    \item Reimplementación en C++ de objetos con gran cantidad de operaciones.
    \item Arreglo de giros erróneos del volante que pueden ocurrir ocasionalmente.
\end{itemize}

\chapter{Obtención del código fuente}

Dado que la aplicación tiene un tamaño demasiado grande, no es posible subirlo a ningún repositorio de manera completa. Por tanto, he decidido subir a GitHub una versión donde solo se incluye el código fuente en \textit{blueprints} y C++, sin los \textit{assets} que hacen que ocupen demasiado espacio. Evidentemente, esta versión no es funcional, al no satisfacer todas las referencias.

\bigskip

Adicionalmente, he subido en una carpeta de Drive dos archivos comprimidos: uno primero con todo el código fuente y los \textit{assets}, por lo que debería ser funcional, y una versión compilada para Windows 10 y 11 en 64 bits.

\bigskip

Los enlaces al repositorio y a la carpeta de Drive son los siguientes:

\begin{itemize}
    \item \url{https://github.com/amt911/TFG-Unreal}
    \item \url{https://drive.google.com/drive/folders/1dwZoSL7Hh45YKErOJHgY3C89RtYpUBlG?usp=sharing}
\end{itemize}

\bigskip

Cabe destacar que los ficheros de C++ se encuentran en el directorio \texttt{Source}, mientras que los blueprints se encuentran en el directorio \texttt{Content}.

% ################################ FIN MIS CAPITULOS ################################
%\input{capitulos/01_Introduccion}
%
%\input{capitulos/02_EspecificacionRequisitos}
%
%\input{capitulos/03_Planificacion}
%
%\input{capitulos/04_Analisis}
%
%\input{capitulos/05_Diseno}
%
%\input{capitulos/06_Implementacion}
%
%\input{capitulos/07_Pruebas}
%
%\input{capitulos/08_Conclusiones}
%
%%\chapter{Conclusiones y Trabajos Futuros}
%
%
%%\nocite{*}
\bibliography{bibliografia/bibliografia}\addcontentsline{toc}{chapter}{Bibliografía}
% \bibliographystyle{miunsrturl}
\bibliographystyle{plainurl}
%
%\appendix
%\input{apendices/manual_usuario/manual_usuario}
%%\input{apendices/paper/paper}
%\input{glosario/entradas_glosario}
% \addcontentsline{toc}{chapter}{Glosario}
% \printglossary
\chapter*{}
\thispagestyle{empty}

\end{document}
